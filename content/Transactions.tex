\chapter{Monero Ring Confidential Transactions (RingCT)}
\label{chapter:transactions}

Throughout Chapters \ref{chapter:addresses} and \ref{chapter:pedersen-commitments} we built up several aspects of Monero transactions. At this point a simple one-input, one-output transaction from some unknown author to some unknown recipient sounds like:

``My transaction uses transaction public key $r G$. I will spend old output $X$ (note that it has a hidden amount $A_X$, committed to in $C_X$). I will give it a pseudo output commitment $C'_X$. I will make one output $Y$, which may be spent by the one-time address $K^o_Y$. It has a hidden amount $A_Y$ committed to in $C_Y$, encrypted for the recipient, and proven in range with a Bulletproofs-style range proof. Please note that $C'_X - C_Y = 0$."

Some questions remain. Did the author actually own $X$? Does the pseudo output commitment $C'_X$ actually correspond to $C_X$, such that $A_X = A'_X = A_Y$? Has someone tampered with the transaction, and perhaps directed the output at a recipient unintended by the original author?
\\

As mentioned in Section \ref{sec:one-time-addresses}, we can prove ownership of an output by signing a message with its one-time address (whoever has the address's key owns the output). We can also prove it has the same amount as a pseudo output commitment by proving knowledge of the commitment to zero's private key ($C_X - C'_X = z_X G$). Moreover, if that message is {\em all the transaction data} (except the signature itself), then verifiers can be assured everything is as the author intended (the signature only works with the original message). MLSAG signatures allow us to do all of this while also obscuring the actual spent output amongst other outputs from the blockchain, so observers can't be sure which one is being spent.



\section{Transaction types}
\label{sec:transaction_types}

Monero\marginnote{src/crypto- note\_core/ cryptonote\_ tx\_utils.cpp {\tt construct\_ tx\_with\_ tx\_key()}} is a cryptocurrency under steady development. Transaction structures, protocols, and cryptographic schemes are always prone to evolving as new innovations, objectives, or threats are found.

In this report we have focused our attention on {\em Ring Confidential Transactions}, a.k.a. {\em RingCT}, as they are implemented in the current version of Monero. RingCT is mandatory for all new Monero transactions, so we will not describe any deprecated transaction schemes, even if they are still partially supported.\footnote{RingCT was first implemented in January 2017 (v4 of the protocol). It was made mandatory for all new transactions in September 2017 (v6 of the protocol) \cite{ringct-dates}. RingCT is version 2 of Monero's transaction protocol.} The transaction type we have discussed so far, and which will be further elaborated in this chapter, is {\tt RCTTypeBulletproof2}.\footnote{Within the RingCT era there are three deprecated transaction types: {\tt RCTTypeFull}, {\tt RCTTypeSimple}, and {\tt RCTTypeBulletproof}. The former two coexisted in the first iteration of RingCT and are explored in the first edition of this report \cite{ztm-1}, then with the advent of Bulletproofs (protocol v8) {\tt RCTTypeFull} was deprecated, and {\tt RCTTypeSimple} was upgraded to {\tt RCTTypeBulletproof}. {\tt RCTTypeBulletproof2} arrived due to a minor improvement in encrypting output commitments' masks and amounts (v10).}

We present a conceptual summary of transactions in Section \ref{sec:transaction_summary}.



\section{Ring Confidential Transactions of type {\tt RCTTypeBulletproof2}}
\label{sec:RCTTypeBulletproof2}

Currently (protocol v12) all new transactions must use this transaction type, in which each input is signed separately. An actual example of an {\tt RCTTypeBulletproof2} transaction, with all its components, can be inspected in Appendix \ref{appendix:RCTTypeBulletproof2}.


\subsection{Amount commitments and transaction fees}
\label{sec:commitments-and-fees}

Assume a transaction sender has previously received various outputs with amounts $a_1, ..., a_m$ addressed to one-time addresses $K^o_{\pi,1}, ..., K^o_{\pi,m}$ and with amount commitments $C^a_{\pi,1}, ..., C^a_{\pi,m}$.

This sender knows the private keys $k^o_{\pi,1}, ..., k^o_{\pi,m}$ corresponding to the one-time addresses (Section \ref{sec:one-time-addresses}). The sender also knows the blinding factors $x_j$ used in commitments $C^a_{\pi,j}$ (Section \ref{sec:pedersen_monero}).

Typically transaction outputs are {\em lower} in total than transaction inputs, in order to provide a fee that will incentivize miners to include the transaction in the blockchain.\footnote{In Monero there is a minimum base fee that scales with transaction weight. It is semi-mandatory because while you can create new blocks containing tiny-fee transactions, most Monero nodes won't relay such transactions to other nodes. The result is few if any miners will try to include them in blocks. Transaction authors can provide miner fees above the minimum if they want. We go into more detail on this in Section \ref{subsec:dynamic-minimum-fee}.} Transaction fee amounts $f$ are stored in clear text in the transaction data transmitted to the network. Miners can create an additional output for themselves with the fee (see Section \ref{subsec:miner-transaction}).

A transaction consists of inputs \(a_1, ..., a_m\) and outputs \(b_0, ..., b_{p-1}\) such that \(\sum\limits_{j=1}^m a_j - \sum\limits_{t=0}^{p-1} b_t - f = 0\).\footnote{Outputs\marginnote{src/crypto- note\_core/ cryptonote\_ tx\_utils.cpp {\tt construct\_ tx\_with\_ tx\_key()}}[-1.9cm] are randomly shuffled by the core implementation before getting assigned an index, so observers can't build heuristics around their order. Inputs are sorted by key image within transaction data.}

The sender calculates pseudo output commitments for the input amounts, $C'^a_{\pi,1}, ..., C'^a_{\pi,m}$, and creates commitments for intended output amounts $b_0, ..., b_{p-1}$. Let these new commitments be $C^b_0, ..., C^b_{p-1}$.

He knows the private keys $z_1,...,z_m$ to the commitments to zero $(C^a_{\pi,1} - C'^a_{\pi,1}),...,(C^a_{\pi,m} - C'^a_{\pi,m})$.

For verifiers to confirm transaction amounts sum to zero, the fee amount must be converted into a commitment. The solution is to calculate the commitment of the fee $f$ without the masking effect of any blinding factor. That is, $C(f) = f H$.

Now\marginnote{src/ringct/ rctSigs.cpp verRct- Semantics- Simple()} we can prove input amounts equal output amounts:\\
\[(\sum_j C'^a_{j} - \sum_t C^b_{t}) - f H = 0\]


\subsection{Signature}
\label{full-signature}

The sender selects $m$ sets of size $v$, of additional unrelated one-time addresses and their commitments from the blockchain, corresponding to apparently unspent outputs.\footnote{\label{input-selection}In Monero it is standard for the sets\marginnote{src/wallet/ wallet2.cpp {\tt get\_outs()}} of `additional unrelated addresses' to be selected from a gamma\marginnote{{\tt gamma\_picker ::pick()}}[1.2cm] distribution across the range of historical outputs (RingCT outputs only, a triangle distribution is used for pre-RingCT outputs). This method uses a binning procedure to smooth out differences in block densities. First calculate the average time between transaction outputs over up to a year ago of RingCT outputs (avg time = [\#outputs/\#blocks]*blocktime). Select an output via the gamma distribution, then look inside its block and grab a random output to be a member of the set. \cite{AnalysisOfLinkability}}\footnote{As of protocol v12, all transaction inputs must be at least 10 blocks old ({\tt CRYPTONOTE\_DEFAULT\_TX \_SPENDABLE\_AGE}). Prior to v12 the core implementation used 10 blocks by default, but it was not required so an alternate wallet could make different choices, and some apparently did \cite{visualizing-monero-vid}.} To sign input $j$, she stirs a set of size $v$ into a {\em ring} with her own $j$\nth unspent one-time address (placed at unique index $\pi$), along with false commitments to zero, as follows:\footnote{In Monero each of a transaction's rings must have the same size, and the protocol controls how many ring members there can be per output-to-be-spent. It has changed with different protocol versions: v2 March 2016 $\geq$ 3, v6 Sept. 2017 $\geq$ 5, v7 April 2018 $\geq$ 7, v8 Oct. 2018 11-only. Since\marginnote{src/crypto- note\_core/ cryptonote\_ core.cpp {\tt check\_ tx\_inputs\_ ring\_members\_ diff()}} v6 each individual ring can not contain duplicate members, though there may be duplicates between rings to allow multiple inputs when there are insufficient total outputs in the transaction history (i.e. to assemble rings without cross-over) \cite{duplicate-ring-members}.}
%functions get_random_rct_outs for rct outputs mixed into an rct spend, and get_random_outputs for pre-rct outputs mixed into a pre-rct spend. These are only used for light wallets -> instead it is wallet2::get_outs for most situations.
%--ring member selection changing to gamma distribution in fall 2018
%CRYPTONOTE_DEFAULT_TX_SPENDABLE_AGE = 10 blocks

\begin{align*}
    \mathcal{R}_j = \{&\{K^o_{1, j}, (C_{1, j} - C'^a_{\pi, j})\}, \\
    &... \\
    &\{ K^o_{\pi, j}, (C^a_{\pi, j} - C'^a_{\pi, j})\}, \\
    &... \\
    &\{ K^o_{v+1, j}, (C_{v+1, j} - C'^a_{\pi, j})\}\}
\end{align*}

Alice\marginnote{src/ringct/ rctSigs.cpp proveRct- MGSimple()} uses an MLSAG signature (Section \ref{sec:MLSAG}) to sign this ring, where she knows the private keys $k^o_{\pi,j}$ for $K^o_{\pi,j}$, and $z_j$ for the commitment to zero $(C^a_{\pi,j}$ - $C'^a_{\pi,j})$. Since no key image is needed for the commitments to zero, there is consequently no corresponding key image component in the signature’s construction.\footnote{Building and verifying the signature excludes the term $r_{i,2} \mathcal{H}_p(C_{i, j} - C'^a_{\pi, j}) + c_i \tilde{K}_{z_j}$.}

Each input in a transaction is signed individually using rings like \(\mathcal{R}_j\) as defined above, thereby obscuring the real outputs being spent, ($K^o_{\pi,1},...,K^o_{\pi,m}$), amongst other unspent outputs.\footnote{The advantage of signing inputs individually is that the set of real inputs and commitments to zero need not be placed at the same index $\pi$, as they would be in the aggregated case. This means even if one input's origin becomes identifiable, the other inputs' origins would not. The old transaction type {\tt RCTTypeFull} used aggregated ring signatures, combining all rings into one, and as we understand it was deprecated for that reason.} Since part of each ring includes a commitment to zero, the pseudo output commitment used must contain an amount equal to the real input being spent. This ties input amounts to the proof that amounts balance, without compromising which ring member is the real input.

The message $\mathfrak{m}$ signed by each input is essentially a hash of all transaction data {\em except} for the MLSAG signatures.\footnote{The\marginnote{src/ringct/ rctSigs.cpp {\tt get\_pre\_ mlsag\_hash()}} actual message is $\mathfrak{m} = \mathcal{H}(\mathcal{H}(tx\textunderscore prefix),\mathcal{H}(ss),\mathcal{H}(\text{range proofs}))$ where:\par
$tx\textunderscore prefix = $\{transaction era version (i.e. RingCT = 2), inputs \{ring member key offsets, key images\}, outputs \{one-time addresses\}, extra \{transaction public key, payment ID or encoded payment ID, misc.\}\}\par
$ss = $\{transaction type ({\tt RCTTypeBulletproof2} = `4'), transaction fee, pseudo output commitments for inputs, ecdhInfo (encrypted amounts), output commitments\}.\par
See Appendix \ref{appendix:RCTTypeBulletproof2} regarding this terminology.} This ensures transactions are tamper-proof from the perspective of both transaction authors and verifiers. Only one message is produced, and each input MLSAG signs it.

One-time private key $k^o$ is the essence of Monero's transaction model. Signing $\mathfrak{m}$ with $k^o$ proves you are the owner of the amount committed to in $C^a$. Verifiers can be confident that transaction authors are spending their own funds, without even knowing which funds are being spent, how much is being spent, or what other funds they might own!


\subsection{Avoiding double-spending}

An\marginnote{src/crypto- note\_core/ block- chain.cpp {\tt have\_tx\_ keyimges\_ as\_spent()}} MLSAG signature (Section \ref{sec:MLSAG}) contains images \(\tilde{K}_{j}\) of private keys \(k_{\pi, j}\). An important property for any cryptographic signature scheme is that it should be unforgeable with non-negligible probability. Therefore, to all practical effects, we can assume a signature’s key images must have been deterministically produced from legitimate private keys.

The network only needs to verify that key images included with MLSAG signatures (corresponding to inputs and calculated as $\tilde{K}^o_{j} = k^o_{\pi,j} \mathcal{H}_p(K^o_{\pi,j})$) have not appeared before in other transactions.\footnote{Verifiers must also check the key image is a member of the generator's subgroup (recall Section \ref{blsag_note}).} If they have, then we can be sure we are witnessing an attempt to re-spend an output $(C^a_{\pi,j}, K_{\pi,j}^o)$.


\subsection{Space requirements}
\label{subsec:space-and-ver-rcttypefull}

\subsubsection*{MLSAG signature (inputs)}

From Section \ref{sec:MLSAG} we recall that an MLSAG signature in this context would be expressed as

\hfill \(\sigma_j(\mathfrak{m}) = (c_1, r_{1, 1}, r_{1, 2}, ..., r_{v+1, 1}, r_{v+1, 2}) \textrm{ with } \tilde{K}^o_j \) \hfill \phantom{.}

As a legacy of CryptoNote, the values \(\tilde{K}^o_j\) are not referred to as part of the signature, but rather as {\em images} of the private keys $k^o_{\pi,j}$. These {\em key images} are normally stored separately in the transaction structure, as they are used to detect double-spending attacks.

With this in mind and assuming point compression (Section \ref{point_compression_section}), since each ring \(\mathcal{R}_j\) contains \((v+1) \cdot 2\) keys, an input signature $\sigma_j$ will require \( (2(v+1) + 1) \cdot 32  \) bytes. On top of this, the key image $\tilde{K}^o_{\pi,j}$ and the pseudo output commitment $C'^a_{\pi,j}$ leave a total of $(2(v+1)+3) \cdot 32$ bytes per input.

To this value we would need additional space to store the ring member offsets in the blockchain (see Appendix \ref{appendix:RCTTypeBulletproof2}). These offsets are used by verifiers to find each MLSAG signature's ring members' output keys and commitments in the blockchain, and are stored as variable length integers, hence we can not exactly quantify the space needed.\footnote{See \cite{varint-description} or \cite{varint-spec} for an explanation of Monero's varint data type\marginnote{src/common/ varint.h}. It is an integer type that uses up to 9 bytes, and stores up to 63 bits of information.}\footnote{Imagine\marginnote{src/crypto- note\_basic/ cryptonote\_ format\_ utils.cpp {\tt absolute\_out- put\_offsets\_ to\_relative()}} the blockchain contains a long list of transaction outputs. We report the indices of outputs we want to use in rings. Now, bigger indexes require more storage space. We just need the `absolute' position of {\em one} index from each ring, and the `relative' positions of the other ring members' indices. For example, with real indices \{7,11,15,20\} we just need to report \{7,4,4,5\}. Verifiers can compute the last index with (7+4+4+5 = 20). Ring members are organized in ascending order of blockchain index within rings.}\footnote{A transaction with 10 inputs using rings with 11 total members will need \(((11 \cdot 2 + 3) \cdot 32) \cdot 10 = 8000 \) bytes for its inputs, with around 110 to 330 bytes for the offsets (there are 110 ring members).}%src/cryptonote_basic/cryptonote_format_utils.cpp absolute_output_offsets_to_relative

Verifying\marginnote{src/ringct/ rctSigs.cpp verRctMG- Simple() verRct- Semantics- Simple()}[-1cm] all of an {\tt RCTTypeBulletproof2} transaction's MLSAGs includes the computation of \( (C_{i, j} - C'^a_{\pi, j}) \) and \( (\sum_j C'^a_{j} \stackrel{?}{=} \sum_t C^b_{t} + f H)\), and verifying key images are in $G$'s subgroup with $l \tilde{K} \stackrel{?}{=} 0$.

\subsubsection*{Range proofs (outputs)}

An\marginnote{src/ringct/ bullet- proofs.cpp {\tt bullet- proof\_ VERIFY()}} aggregate Bulletproof range proof will require $(2 \cdot \lceil \textrm{log}_2(64 \cdot p) \rceil + 9) \cdot 32$ total bytes.



\newpage
\section{Concept summary: Monero transactions}
\label{sec:transaction_summary}

To summarize this chapter, and the previous two chapters, we present the main content of a transaction, organized for conceptual clarity. A real example can be found in Appendix \ref{appendix:RCTTypeBulletproof2}.

\begin{itemize}
    \item \underline{Type}: `0' is {\tt RCTTypeNull} (for miners), `4' is {\tt RCTTypeBulletproof2} %see chapter 7, blockchain, about type 0 transactions
    \item \underline{Inputs}: for each input $j \in \{1,...,m\}$ spent by the transaction author
    \begin{itemize}
        \item \textbf{Ring member offsets}: a list of `offsets' indicating where a verifier can find input $j$'s ring members $i \in \{1,...,v+1\}$ in the blockchain (includes the real input)
        \item \textbf{MLSAG Signature}: $\sigma_j$ terms $c_1$, and $r_{i,1}$ \& $r_{i,2}$ for $i \in \{1,...,v+1\}$
        \item \textbf{Key image}: the key image $\tilde{K}^{o,a}_j$ for input $j$
        \item \textbf{Pseudo output commitment}: $C'^{a}_j$ for input $j$
    \end{itemize}
    
    \item \underline{Outputs}: for each output $t \in \{0,...,p-1\}$ to address or subaddress $(K^v_t,K^s_t)$
    \begin{itemize}
        \item \textbf{One-time address}: $K^{o,b}_t$ for output $t$
        \item \textbf{Output commitment}: $C^{b}_t$ for output $t$
        \item \textbf{Encoded amount}: so output owners can compute $b_t$ for output $t$
        \begin{itemize}
            \item \textit{Amount}: $b_t \oplus_8 \mathcal{H}_n(``amount”, \mathcal{H}_n(r K_B^v, t))$
        \end{itemize}
        \item \textbf{Range proof}: an aggregate Bulletproof for all output amounts $b_t$
        \begin{itemize}
            \item \textit{Proof}: $\Pi_{BP} = (A, S, T_1, T_2, \tau_x, \mu, \mathbb{L}, \mathbb{R}, a, b, t)$
        \end{itemize}
    \end{itemize}
    \item \underline{Transaction fee}: communicated in clear text multiplied by $10^{12}$ (i.e. atomic units, see Section \ref{subsec:block-reward}), so a fee of 1.0 would be recorded as 1000000000000
    \item \underline{Extra}: includes the transaction public key $r G$, or, if at least one output is directed to a subaddress, $r_t K^{s,i}_t$ for each subaddress'd output $t$ and $r_t G$ for each normal address'd output $t$, and maybe an encoded payment ID (should be at most one per transaction)\footnote{No information stored in the `extra' field is verified, though it {\em is} signed by input MLSAGs, so no tampering is possible (except with negligible probability). The field has no limit on how much data it can store, so long as the maximum transaction weight is respected. See \cite{extra-field-stackexchange} for more details.}
\end{itemize}

Our final one-input/one-output example transaction sounds like this: ``My transaction uses transaction public key $r G$. I will spend one of the outputs in set $\mathbb{X}$ (note that it has a hidden amount $A_X$, committed to in $C_X$). The output being spent is owned by me (I made a MSLAG signature on the one-time addresses in $\mathbb{X}$), and hasn't been spent before (its key image $\tilde{K}$ has not yet appeared in the blockchain). I will give it a pseudo output commitment $C'_X$. I will make one output $Y$, which may be spent by the one-time address $K^o_Y$. It has a hidden amount $A_Y$ committed to in $C_Y$, encrypted for the recipient, and proven in range with a Bulletproofs-style range proof. My transaction includes a transaction fee $f$. Please note that $C'_X - (C_Y + C_f) = 0$, and that I have signed the commitment to zero $C'_X - C_X = z G$ which means the input amount equals the output amount ($A_X = A'_X = A_Y + f$). My MLSAG signed all transaction data, so observers can be sure it hasn't been tampered with."


\newpage
\subsection{Storage requirements}

For {\tt RCTTypeBulletproof2} we need $(2(v+1)+2) \cdot m \cdot 32$ bytes of storage, and the aggregate Bulletproof range proof needs $(2 \cdot \lceil \textrm{log}_2(64 \cdot p) \rceil + 9) \cdot 32$ bytes of storage.\footnote{The amount of transaction content is limited by a maximum so-called `transaction weight'.\marginnote{src/crypto- note\_core/ tx\_pool.cpp {\tt get\_trans- action\_weight \_limit()}} Before Bulletproofs were added in protocol v8 (and indeed currently when transactions have only two outputs) the transaction weight and size in bytes were equivalent. The maximum weight is (0.5*300kB - {\tt CRYPTONOTE\_COINBASE\_BLOB\_RESERVED\_SIZE}), where the blob reserve (600 bytes) is intended to leave room for the miner transaction within blocks. Before v8 the 0.5x multiplier was not included, and the 300kB term was smaller in earlier protocol versions (20kB v1, 60kB v2, 300kB v5). We elaborate on these topics in Section \ref{subsec:dynamic-block-weight}.}

Miscellaneous requirements:
\begin{itemize}
    \setlength\itemsep{\listspace}
    \item Input key images: $m*32$ bytes
    \item One-time output addresses: $p*32$ bytes
    \item Output commitments: $p*32$ bytes
    \item Encoded output amounts: $p*8$ bytes
    \item Transaction public key: 32 bytes normally, $p*32$ bytes if sending to at least one subaddress.
    \item Payment ID: 8 bytes for an integrated address. There should be no more than one per tx.
    \item Transaction fee: stored as a variable length integer, so $\leq 9$ bytes
    \item Input offsets: stored as variable length integers, so $\leq 9$ bytes per offset, for $m*(v+1)$ ring members
    \item Unlock time: stored as a variable length integer, so $\leq 9$ bytes\footnote{Any transaction's author can lock its outputs\marginnote{src/crypto- note\_core/ block- chain.cpp {\tt is\_tx\_ spendtime\_ unlocked()}}, rendering them unspendable until a specified block height where it may be spent (or until a UNIX timestamp, at which time its outputs may be included in a block's transaction's ring members). He only has the option to lock all outputs to the same block height. It is not clear if this offers any meaningful utility to transaction authors (perhaps smart contracts). Miner transactions have a mandatory 60-block lock time. As of protocol v12 normal outputs can't be spent until after the default spendable age (10 blocks) which is functionally equivalent to a mandatory minimum 10-block lock time. If a transaction is published in the 10\nth block with an unlock time of 25, it may be spent in the 25\nth block or later. Unlock time is probably the least used of all transaction features for normal transactions.}
    \item `Extra' tags: each piece of data in the `extra' field (e.g. a transaction public key) begins with a 1 byte `tag', and some pieces also have a 1+ byte `length'; see Appendix \ref{appendix:RCTTypeBulletproof2} for more details
\end{itemize}