% this file is called up by main.tex
% content in this file will be fed into the main document

% ---------------------------------------------------------------------------

Cryptography. It may seem like only mathematicians and computer scientists have access to this obscure, esoteric, powerful, elegant topic. In fact, many kinds of cryptography are simple enough that anyone can learn their fundamental concepts.
\\ \newline
Many people know cryptography is used to secure communications, whether they be coded letters or private digital interactions. Another application is in so-called cryptocurrencies. These digital moneys use cryptography to ensure, first and foremost, that no piece of money can be duplicated or created at will. To that end, cryptocurrencies typically rely on `blockchains', creating public, distributed ledgers containing records of currency transactions that can be verified by third parties \cite{Nakamoto_bitcoin}.
\\ \newline
It might seem at first glance that transactions need to be sent and stored in plain text format in order to make them publicly verifiable. In fact, it is possible to conceal participants of transactions, as well as the amounts involved, using cryptographic tools that nevertheless allow transactions to be verified and agreed upon by observers \cite{cryptoNoteWhitePaper}. This is exemplified in the cryptocurrency Monero.
\\ \newline
We endeavor here to teach anyone who knows basic algebra and simple computer science concepts like the `bit representation' of a number not only how Monero works at a deep and comprehensive level, but also how useful and beautiful cryptography can be.
\\ \newline
For our experienced readers: Monero is a standard one-dimensional distributed acyclic graph (DAG) cryptocurrency blockchain \cite{Nakamoto_bitcoin} where transactions are based on elliptic curve cryptography using curve Ed25519 \cite{Bernstein2008}, transaction inputs are signed with Schnorr-style multilayered linkable spontaneous anonymous group signatures (MLSAG) \cite{MRL-0005}, and output amounts (communicated to recipients via ECDH \cite{Diffie-Hellman}) are concealed with Pedersen commitments \cite{maxwell-ct} and Schnorr-style Borromean ring signatures \cite{Signatures2015BorromeanRS}. Much of this report is spent explaining these ideas.