% this file is called up by main.tex
% content in this file will be fed into the main document

% ---------------------------------------------------------------------------

Криптография. Может показаться, что эта во многом непонятная, эзотерическая и в чём-то элегантная тема доступна для понимания только математикам и учёным. На самом же деле многие виды криптографии достаточно просты, а их фундаментальные концепции доступны буквально каждому.
\\ \newline
Общеизвестно, что криптография используется для защиты информации, будь то шифровка текстовых сообщений или обеспечение приватности цифрового взаимодействия. Ещё одним способом применения криптографии являются так называемые криптовалюты - цифровые деньги. Чтобы гарантировано исключить возможность дублирования или создания какой-либо части денег по собственному желанию. В своей основе криптовалюты обычно полагаются на «блокчейны», являющиеся публичными распределенными реестрами, которые содержат записи валютных транзакций, которые могут быть верифицированы третьими лицами \cite{Nakamoto_bitcoin}.
\\ \newline
Может показаться, что транзакции передаются и хранятся в простом текстовом формате, чтобы обеспечить возможность их публичной верификации. На самом же деле криптографические инструменты позволяют скрыть участников транзакции, а также передаваемые суммы, но при этом наблюдатели всё равно смогут верифицировать и согласовать их \cite{cryptoNoteWhitePaper}.
\\ \newline
Мы стремимся научить любого, кто знаком с базовой алгеброй и простыми понятиями информатики, такими как «битовое представление» числа, не только тому, как работает Monero, но также и тому, насколько полезной и красивой может быть криптография.
\\ \newline
Monero является криптовалютным блокчейном на базе одномерного распределённого ациклического графа (DAG) \cite{Nakamoto_bitcoin}, где транзакции используют криптографию на эллиптических кривых, а именно кривой Ed25519 \cite{Bernstein2008}, входы транзакций подписываются при помощи многоуровневых связных подписей спонтанной анонимной группы (MLSAG) в стиле Шнорра \cite{MRL-0005-ringct}, а суммы выходов (передаваемые получателям посредством ECDH \cite{Diffie-Hellman}) скрыты при помощи обязательств Педерсена \cite{maxwell-ct} и доказываются в допустимом диапазоне с использованием Bulletproofs \cite{Bulletproofs_paper}.