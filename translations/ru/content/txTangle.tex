\chapter{Объединённые транзакции Monero (TxTangle)}
\label{chapter:txtangle}

Существует целый ряд неизбежных способов эвристического анализа графов транзакций, используемых в зависимости от объектов, по отношению к которым они применяются, и сценариев.  В частности, поведение майнеров, пулов (см. подраздел 5.1 в работе \cite{AnalysisOfLinkability}), торговых площадок, использующих эскроу-счета, а также бирж соответствует определённым шаблонам, которые уязвимы для анализа даже в случае с основанным на применении кольцевых подписей протоколом Monero.

В данном разделе мы описываем TxTangle, аналогичный CoinJoin \cite{coinjoin-wiki}, который используется Bitcoin, метод, позволяющий избежать такого эвристического анализа.\footnote{В этой главе предлагается протокол создания объединённых транзакций. На момент написания документа такого протокола ещё не существовало. Предыдущее предложение под названием MoJoin было создано исследователем из Исследовательской лаборатории Monero под псевдонимом Sarang Noether (Саранг Ноезер) и требовало участия доверенного лица. Это противоречит базовому принципу проекта Monero, связанному с соблюдением анонимности и обеспечением взаимозаменяемости, и поэтому работа над MoJoin не была продолжена.} По сути, происходит слияние нескольких транзакций в одну, что смешивает поведенческие шаблоны каждого из участников.

Чтобы обеспечить эффективность такой обфускации, необходимо сделать так, чтобы внешним наблюдателям было неоправданно сложно использовать данные объединённых транзакций с целью определения групп входов и выходов для их дальнейшей привязки к отдельным участникам, а также нужно, чтобы эти наблюдатели в принципе не могли узнать, сколько пользователей участвует в создании транзакции.\footnote{Поскольку любой может увидеть суммы транзакций Bitcoin, как правило, существует возможность объединения входов и выходов CoinJoin, исходя из соответствующих сумм. \cite{coinjoin-sudoku}} Более того, даже самим участникам не должно быть известно их точное количество, и они не должны иметь возможности объедине\-ния входов и выходов других участников в группы, если только не контролируются все группы входов и выходов, кроме группы одного участника.\footnote{В случае с данным методом существует риск проведения атаки путём умышленного «засорения» объединённых транзакций, что впервые было обнаружено в CoinJoin. \cite{coinjoin-pollution}} Наконец должна иметься воз\-можность строить объединённые транзакции без участия какой-либо централизованной кон\-тролирующей процесс стороны \cite{exa-blockchain-analysis}. К счастью, Monero обеспечивает соответствие всем этим требованиям.



\section{Построение объединённых транзакций}
\label{sec:building-txtangle}

В обычной транзакции входы и выходы связаны между собой при помощи доказательства того, что суммы сбалансированы. Как было указано в подпункте \ref{sec:commitments-and-fees}, сумма обязательств по псевдовыходам равна сумме обязательств по выходам (плюс обязательство по комиссии).
\[\sum_j C'^a_{j} - (\sum_t C^b_{t} + f H) = 0\]

Простая объединённая транзакция может включать в себя всё содержимое множества транзак\-ций и позволяет объединить их в одну. Сообщения MLSAG могли бы использоваться для подписания всех данных подтранзакций, и сбалансированность сумм очевидно бы работала (0 + 0 + 0 = 0). Простая объединённая транзакция может включать в себя всё содержимое множества транзакций и позволяет объединить их в одну. Сообщения MLSAG могли бы использоваться для подписания всех данных подтранзакций, и сбалансированность сумм очевидно бы работала (0 + 0 + 0 = 0).\footnote{Поскольку доказательства в стиле Bulletproofs, по сути, объединяются в одно (см. подпункт \ref{sec:range_proofs}), даже в самом простом случае от участников потребуется некоторое сотрудничество.} Однако, тогда группу входов и выходов можно было бы выявить путём определения соответствия поднаборов входов/выходов суммам.\footnote{Участники могут хитрым образом разделить комиссию, чтобы запутать наблюдателей. Но это не сработает в случае атаки методом грубого перебора, так как размер комиссий недостаточно велик (примерно 32 бита или менее).}

Эта проблема легко решается путём вычисления общих секретов каждой пары участников и последующего добавления этих смещений масок обязательств по псевдовыходам (см. под\-пункт \ref{sec:ringct-introduction}). В каждой паре один из участников добавляет общий секрет к одному из {\em своих} обязательств по псевдовыходу. После того как всё будет просуммировано, секреты будут отменены, и поскольку у каждой пары участников имеется общий секрет, сумма будет сбалан\-сирована только после того, как все обязательства будут объединены.\footnote{Вместо обязательств по выходам мы смещаем обязательства по псевдовыходам, так как маски обязательств по выходам строятся на основе адреса получателя (см. подпункт \ref{sec:pedersen_monero}).}

Общие секреты позволяют скрыть группу входов/выходов в прямом смысле, но участникам необходимо как-то узнать все входы и выходы, и проще всего это сделать, если они сами сообщат свои группы входов/выходов. Очевидно, это нарушит изначальную задумку и будет подразумевать, что участникам будет известно их общее количество.


\subsection{$n$-направленный канал передачи данных}
\label{subsec:n-way-channel}

Максимальное количество участников транзакции TxTangle определяется либо количеством выходов, либо количеством входов (в зависимости от того, какое значение будет ниже). В рамках нашей модели каждый реальный участник притворяется, что он является другим человеком в случае с каждым отправляемым им выходом. В первую очередь это делается с целью создания группового канала передачи данных с другими потенциальными участниками, но общее количество участников при этом не раскрывается.

Представьте, что $n$ ($2 \leq n \leq 16$, хотя рекомендуется наличие по крайней мере 3)\footnote{В настоящее время транзакция может содержать 16 выходов максимум.} предположи\-тельно не связанных друг с другом людей собирается со случайными интервалами в чатруме, который откроется во время $t_0$, близкое к $t_1$ (одновременно в комнате может присутствовать только 16 человек, и приоритет в ней отдаётся на основе комиссий, базовых комиссий на байт [для простоты согласования текущей средней величины и вознаграждения за блок], а также диапазона допустимых типов транзакций, поскольку, например, транзакции, которые были проведены в начале существования Monero, нельзя напрямую потратить в транзакции RingCT \cite{pre-ringct-outputs-like-coinbase-research-issue-59}). С наступлением $t_1$ все фиктивные участники заявляют о своём желании продолжить процедуру, публикуя публичные ключи, и комната превращается в $n$-направленный канал передачи данных, когда создаётся общий секрет для всех её фиктивных участников.\footnote{Метод мультиподписи, о котором говорится в подпункте \ref{sec:m-of-n}, является однонаправленным, и схема «M-из-N» будет безоговорочно расширяться до «1-из-N».} Этот общий секрет используется для шифрования содержимого сообщений, в то время как фиктив\-ные участники подписывают связанные со входами сообщения, используя подписи SAG (см. подпункт \ref{SAG_section}). Поэтому остаётся совершенно неясным, кто отправил отдельно взятое сообще\-ние. Сообщения, связанные с выходами, подписываются при помощи bLSAG (см. подпункт \ref{blsag_note}) в соответствии с набором публичных ключей фиктивных участников. Таким образом, фактические выходы не получится связать с ними.\footnote{В каждом отдельном наборе bLSAG транзакций TxTangle должны использоваться одни и те же образы ключей, чтобы всё относящееся к конкретному выходу можно было связать.}% can create their key images using a different hash-to-point algorithm, or more simply by tagging the hash with a string, e.g. $\tilde{K}_t = \mathcal{H}_p(``TxTangle\_bLSAG\_2",K_t)$.}


\subsection{Количество раундов обмена сообщениями при построении\linebreak объединённой транзакции}
\label{subsec:message-rounds-txtangle}

После того как канал будет настроен, можно строить транзакции TxTangle. Это делается за пять раундов обмена данными, где очередной раунд может начаться только после завершения предыдущего, и для выполнения каждого раунда даётся определенный временной интервал, в течение которого сообщения публикуются случайным образом. Эти интервалы необходимы во избежание создания кластеров сообщений, которые могут раскрыть группы входов/выходов.
\begin{enumerate}
    \item Каждый фиктивный участник анонимно генерирует случайную скалярную величину для каждого предполагаемого выхода и подписывает их при помощи bLSAG. Отсортиро\-ванный список этих скалярных величин используется для определения индексов выходов (см. подпункт \ref{sec:multi_out_transactions}; наименьшая скалярная величина получает индекс $t = 0$).\footnote{Выбор индексов выходов должен соответствовать другим вариантам реализации конструкций транзакций. Это позволит избежать «маркировки» транзакций другим программным обеспечением. Мы используем этот случайный подход, чтобы обеспечить соответствие основному варианту реализации, в котором индексы выходов также выбираются случайным образом.} Они публикуют эти bLSAG, а также SAG, которыми подписываются номера версии транзак\-ции для заданных входов. После прохождения этого раунда участники могут вычислить вес транзакции, исходя из количества входов и выходов, а также точно определить размер необходимой комиссии.\footnote{Если в результате сравнения количества входов и выходов с чьи-то собственным набором входов/выходов выяснится, что в создании транзакции TxTangle участвуют всего два человека, то создание такой транзакции лучше прекратить. Рекомендуется, чтобы у каждого участника было по крайней мере два входа и два выхода на тот случай, если какой-либо злоумышленник решит не прекращать создание транзакции TxTangle, даже зная о наличии всего двух участников. Данная рекомендация подлежит дальнейшему обсуждению, поскольку использование большего количества входов и выходов не является нейтральным с точки зрения проведения эвристического анализа.}\footnote{В дополнительном поле транзакций TxTangle не должно присутствовать какой-либо внешней информации (например, незашифрованных идентификаторов, если только это не транзакция TxTangle с 2 выходами, где идентификатор платежа должен быть хотя бы фиктивно зашифрован).}\footnote{При оценке размера комиссии должен использоваться стандартизированный подход, чтобы у каждого из участников получался один и тот же результат. В противном случае могут образоваться кластеры выходов, основанные на методе вычисления. Тот же стандарт вычисления размера комиссии должен использоваться и в случае с транзакциями, не являющимися TxTangle, чтобы транзакции TxTangle выглядели так же, как и обычные транзакции.}
    \item Каждый фиктивный участник использует ряд публичных ключей, чтобы создать общий секрет с другим участником с целью смещения их обязательств по псевдовыходам, а также решает, кто будет добавлять, а кто вычитать в зависимости от того, какой из публичных ключей в каждой паре будет меньше.\footnote{Так как точки сжаты (см. подпункт \ref{point_compression_section}), ключи интерпретируются как целые 32-байтовые числа. По определению владелец самого малого ключа складывает, а владелец самого большого — вычитает.} Каждый из фиктивных участников должен заплатить 1/$n$ от вычисленной комиссии (при помощи целочисленного деления). Фиктивный участник с самым низким значением индекса выхода несёт ответственность за выплату остатка после деления (это будет действительно ничтожная сумма, но её необходимо учитывать во избежание «маркировки» транзакций TxTangle). Участники анонимно генерируют публичные ключи транзакций для каждого из своих выходов (пока что не для отправки другим участникам) и создают обязательства по своим выходам, зашифрованные суммы, частичные доказательства Части A, которые будут использоваться для создания совокупного доказательства диапазона Bulletproof. Всё это подписывается при помощи bLSAG (одно обязательство, одна зашифрованная сумма и одно частичное доказательство на сообщение bLSAG, а образ ключа связывает этот оригинальный список случайных скалярных величин, которые использовались для обо\-значения индексов выходов). Обязательства по псевдовыходам генерируются обычным образом (см. подпункт \ref{sec:ringct-introduction}), а затем смещаются при помощи общих секретов и подписы\-ваются с использованием SAG. После того как bLSAG и SAG будут опубликованы, а общая комиссия в одинаковой степени верно будет рассчитана всеми участниками, можно верифицировать общую сбалансированность сумм.\footnote{Мы не публикуем отдельные суммы комиссий, выплачиваемых в том случае, если участник вычислил её неверно, что может раскрыть кластер выходов из-за накопления ряда нестандартных сумм комиссии. Если суммы не сбалансированы надлежащим образом, создание транзакции TxTangle можно прервать.}
    \item Если суммы сбалансированы надлежащим образом, можно начинать дополнительный раунд построения совокупного доказательства Bulletproof, которое докажет, что все суммы в выходах находятся в пределах допустимого диапазона. Каждый фиктивный участник использует частичные доказательства Части A предыдущего раунда и соответ\-ствующие обязательства по выходам и анонимно вычисляет совокупный запрос A. Они используются для построения собственного частичного доказательства Части B, которое отправляется по каналу вместе с bLSAG.
    \item Участники начинают заполнять сообщение, которое будет подписано при помощи MLSAG (см. сноску в подпункте \ref{full-signature}). Два вида сообщений публикуются в случайном порядке через интервал связи. Каждое смещение участника кольца с выходом и каждый образ ключа подписывается при помощи SAG и связывается с правильным обязатель\-ством по псевдовыходу. Одноразовый адрес каждого выхода, публичный ключ транзак\-ции и частичное доказательство Части C (вычисляется на основе частичных доказа\-тельств Части B и совокупного запроса B) подписываются при помощи bLSAG (сюда также может входить случайный компонент публичного ключа базовой транзакции, который, как мы увидим, можно задействовать во избежание подделки путём проведе\-ния атаки Януса).
    \item Участники используют все частичные доказательства, чтобы создать совокупное дока\-зательство Bulletproof, и анонимно используют метод логарифмического скалярного произведения, чтобы сжать его до окончательного доказательства, которое будет вклю\-чено в данные транзакции. Как только вся информация, подлежащая подписанию MLSAG, будет собрана, каждый из участников должен создать MLSAG для своих входов и случайным образом отправить их (применив SAG к каждому) по каналу с интервалом связи. Любой из участников может передать свою транзакцию, как только будут собра\-ны все её части.
\end{enumerate}{}

\subsubsection*{Публичные ключи транзакций и способ избежать атаки Януса}

Если каждому из участников транзакции TxTangle известен приватный ключ транзакции $r$ (см. подпункт \ref{sec:one-time-addresses}), то любой из них может сравнить одноразовые адреса выходов со списком известных адресов. Поэтому существует необходимость в построении транзакций TxTangle, как если бы был получатель с подадресом (см. подпункт \ref{sec:subaddresses}), включая использование различных публичных ключей транзакции для каждого выхода.

В целях соответствия возможным вариантам предотвращения возможности проведения атаки Януса, связанной с подадресами, когда в дополнительное поле включается дополнительный «базовый» публичный ключ транзакции \cite{janus-mitigation-issue-62}, TxTangle также должна содержать фальшивый «базовый» ключ, состоящий из суммы случайных ключей, сгенерированных каждым из фик\-тивных участников.\footnote{Отмена ключа (см. подпункт \ref{subsec:drawbacks-naive-aggregation-cancellation}) не должна быть проблемой, поскольку это всего лишь фальшивый ключ, и в идеале его следует случайным образом проиндексировать в списке публичных ключей транзакции.}

У множества участников TxTangle, отправляющих деньги на подадрес, вероятнее всего, будет, по крайней мере, по два выхода, один из которых будет использоваться, чтобы вернуть сдачу участнику. Это означает, что любой из участников TxTangle может обеспечить защиту от атаки Януса, также сделав публичный ключ транзакции сдачи «базовым» ключом для подадреса получателя.\footnote{При отправке средств на ваш собственный подадрес нет никакой необходимости в предотвращении атаки Януса. Кошельки с включенной защитой от такого вида атаки должны распознавать, что сумма, которая тратится в транзакции TxTangle, равна сумме, полученной на ваш подадрес, и поэтому они ошибочно не уведомляют пользователя о возможной проблеме.} Получатель, использующий подадрес, может понять, что транзак\-ция является TxTangle и что «базовый» ключ, вероятно, соответствует выходу сдачи отправи\-теля.\footnote{Предполагается, что количество публичных ключей транзакции совпадает в соотношении 1:1 с выходами, как, очевидно, и делается сегодня. Если бы по стандарту публичные ключи транзакции располагались в дополнительном поле в случайном или отсортированном порядке, то транзакции TxTangle и транзакции другого вида были бы в значительной степени неотличимы для получателей, использующих подадреса. Существуют особые случаи, когда участники TxTangle не могут включить «базовый» ключ (например, когда все их выходы относятся к подадресам), или когда транзакция явно не относится к виду TxTangle, поскольку получатель, использующий подадрес, получает большую часть или выходов, или всех их. Следует отметить, что, поскольку транзакции TxTangle обычно имеют намного больше выходов, чем типичная транзакция, для того чтобы отличить TxTangles от обычных транзакций с подадресами, можно прибегнуть к эвристическому анализу.}


\subsection{Слабые стороны}
\label{subsec:weaknesses-txtangle}

У злоумышленников есть два основных способа убить смысл TxTangle, то есть обойти меха\-низм сокрытия групп входов/выходов от потенциального анализа. Они могут «засорить» транзакции так, что ряд честных участников станет предельно мал (или их вообще не будет) \cite{coinjoin-pollution}. Они также могут воспрепятствовать попытке создания TxTangle и использовать после\-дующие попытки тех же участников, чтобы оценить группы входов/выходов.

В первом случае этого будет сложно избежать, особенно при децентрализованном сценарии, где ни один из участников не будет обладать достаточной репутацией. Одним из способов применения транзакций TxTangle является их использование в совместных пулах, которые скрывают, к какому именно пулу из целого набора принадлежит майнер. Таким пулам будут известны группы входов/выходов, но так как их целью является помощь подключившимся к ним майнерам, это будет подталкивать их к тому, чтобы хранить эту информацию в секрете. Более того, такие транзакции TxTangle исключат из процесса злоумышленников, если допустить, что поведение пулов будет честным.

В последнем случае защиту можно обеспечить, только попытавшись провести TxTangle не\-сколько раз перед тем, как прервать её, и при этом в случае с каждой транзакцией всегда необходимо генерировать большее количество случайных элементов. Среди этих элементов публичные ключи транзакции, маски обязательств по псевдовыходам, скалярные величины доказательств диапазона и скалярные величины MLSAG. В частности, набор ложных выходов в кольцах должен оставаться тем же, что позволит избежать перекрёстного сравнения, поз\-воляющего выявить действительный вход.  Если это возможно, то при разных попытках создания TxTangle следует использовать и разные действительные входы. Поскольку такая уязвимость неизбежна, она делает концепцию, изложенную в следующем подпункте, более важной.



\section{Организованная транзакция TxTangle}
\label{sec:hosted-txtangle}

В случае с действительно децентрализованными транзакциями TxTangle остаётся несколько открытых вопросов. Как запускаются и реализуются рассчитанные по времени раунды? Как вообще создаются чатрумы, чтобы участники могли найти друг друга? Самый простой способ — организовать хост TxTangle, который будет генерирует эти чатрумы и управлять ими.

Такой хост, казалось бы, сводит на нет цель, заключающуюся в сокрытии участия, поскольку каждый человек должен будет подключиться к нему и отправлять сообщения, которые можно затем использовать для корреляции групп входов/выходов (особенно в том случае, если хост участвует в создании транзакции и знает содержимое сообщения). Мы могли бы использовать такую сеть, как I2P\footnote{The Invisible Internet Project - Проект «Невидимый интернет» (\url{https://geti2p.net/en/}).}, чтобы каждое сообщение, полученное хостом, выглядело так, как если бы оно было отправлено уникальным пользователем.


\subsection{Базовые принципы обмена данными с хостом через I2P и другие функции}
\label{subsec:txtangle-host-communication}

С помощью I2P пользователи создают так называемые «туннели», по которым передаются зашифрованные сообщения.  Перед тем как достичь адресата, сообщения проходят через клиентов других пользователей. Из чего можно понять, что по этим туннелям может быть передано множество сообщений, прежде чем они будут уничтожены и воссозданы (например, для таких туннелей может быть установлен 10-минутный таймер). В нашей ситуации важно тщательно контролировать, когда создаются новые туннели и какие сообщения могут посту\-пать из одного и того же туннеля.\footnote{В I2P есть «исходящие» и «входящие» туннели (см. \url{https://geti2p.net/en/docs/how/tunnel-routing}). Всё, что принимается по входящим туннелям, выглядит так, как будто получено из одного и того же источника, даже если на самом деле их несколько. Поэтому на первый взгляд может показаться, что пользователям TxTangle не нужно создавать разные туннели для всех сценариев использования. Однако если хост TxTangle делает себя точкой входа для собственного входящего туннеля, то он получает прямой доступ к исходящим туннелям участников TxTangle.}%\footnote{For the sake of absolutely minimal information leaks, what we describe here is probably incredibly inefficient, especially since I2P is already very inefficient compared to the `clear' web.} is it?
\begin{enumerate}
    \item {\em Применительно к TxTangles}: в нашем первоначальном $n$-направленном варианте (см. подпункт \ref{subsec:n-way-channel}) фиктивные участники постепенно добавляются в доступные комнаты TxTangle до того, как они будут закрыты. Однако если достаточно большое количество пользователей одновременно попытается создать TxTangle, высока вероятность сбоя, поскольку пользователи попытаются случайным образом поместить все свои выходы в одну и ту же «комнату» TxTangle, но тогда комнаты будут заполняться слишком быстро, и пользователям придётся отступить. Это вызвало бы самую настоящую путаницу.

    Мы можем провести эффективную оптимизацию, сообщив хосту, сколько выходов у нас имеется (например, предоставив ему список наших публичных ключей фиктивного участника), и позволив ему собрать всех участников TxTangle. Поскольку мы по-прежнему следуем протоколу обмена сообщениями bLSAG и SAG, хост не сможет иден\-тифицировать группы выходов в окончательной транзакции. Всё, что ему будет извест\-но — это количество участников и количество выходов, имеющихся у каждого из них. Более того, при таком сценарии наблюдатели не смогут отслеживать открытые комнаты TxTangle с целью получения информации об участниках, что является важным улучше\-нием с точки зрения анонимности. Обратите внимание, что способность хоста «засо\-рять» TxTangles не отличается существенно от решения без привлечения хоста, поэтому это изменение будет нейтральным для данного вектора атаки.
    \item {\em Метод передачи данных}: поскольку хост уже действует как локус передачи сообщений, он без труда может управлять передачей данных TxTangle. Во время каждого раунда хост собирает сообщения от фиктивных участников (всё ещё случайным образом в течение определенного временного интервала связи), а в конце раунда имеется короткая фаза распределения данных, когда он отправляет все собранные данные каждому участ\-нику с определённым периодом буферизации. Это делается перед следующим раундом, чтобы убедиться в том, что сообщения получены и у участников есть время для их обработки.% If someone controls multiple mock-members, it's possible that some of those messages just fizzle out, or perhaps that person receives duplicate messages (e.g. the host is given a different destination for each mock-member). The host should not realize how many real people he is distributing to, nor which mock-member has a given output.
    \item {\em Туннели и группы входов/выходов}: после того как создание транзакции TxTangle будет инициировано, пользователям будет необходимо отделить свои фиктивные личности от фактических создаваемых выходов. Это означает необходимость в создании новых туннелей для передачи сообщений, подписанных bLSAG, и по каждому такому туннелю могут передаваться только сообщения, относящиеся к определённому выходу (можно передавать несколько таких сообщений через один и тот же туннель, поскольку очевидно, что информация об одном и том же выходе поступает из одного и того же источника). Также следует создать новые туннели для подписанных SAG сообщений, относящихся к определённым входам.
    \item {\em Угроза проведения атаки посредника (MITM) хостом}: хост может обмануть участника, притворившись другими участниками, поскольку он контролирует рассылку списка фиктивных участников для создания bLSAG и SAG. Другими словами, список, который он отправляет пользователю A, может содержать фиктивных участников пользователя A, а все остальные будут его собственными. Сообщения, получаемые пользователем B, отклоняются при помощи списка A перед повторной передачей пользователю A. Поскольку все сообщения, подписанные с использованием списка A, принадлежат A, хост будет иметь непосредственное представление о группах входов/выходов A!
    
    Мы можем запретить хосту действовать в качестве MITM при взаимодействии с честны\-ми участниками, изменив способ создания публичных ключей транзакции. Участники отправляют друг другу свои предполагаемые публичные ключи транзакции обычным образом (с помощью bLSAG), а затем, как и в случае надёжной агрегации ключей, описанной в подпункте \ref{sec:robust-key-aggregation}, к фактическим ключам, которые включаются в данные транзакции (и используются для создания масок обязательств по выходам и т. д.) в начале добавляется хеш списка фиктивных участников. Другими словами,\linebreak $\mathcal{H}_n(T_{agg},\mathbb{S}_{mock},r_t G)*r_t G$ является публичным ключом $t$-й транзакции. Включение списка фиктивных участников в саму транзакцию затрудняет завершение TxTangles без прямого взаимодействия между всеми фактическими участниками.\footnote{Если реализуется защита от атаки Януса, вместо этого должна выполняться данная защита от MITM, что делается с помощью фальшивого базового ключа, используемого для предотвращения атаки Януса. Каждый фиктивный участник выдаёт случайный ключ $r_{mock} G$, и тогда фактическим базовым ключом будет $\sum_{mock} \mathcal{H}_n(T_{agg},\mathbb{S}_{mock},r_{mock} G)*r_{mock} G$.}%Users must access the host's `eepsite/service' to discover available TxTangles, without revealing to the host how many prospective participants there are. If he sees five requests for available TxTangles, and then five outputs/mock-members appear, he shouldn't be able to conclude there are five actual participants, nor easily associate any given request with a specific mock-member. To accomplish this, while engaged in a TxTangle a user's client should at random intervals create a new tunnel to the host and through it send a (false) request for available TxTangles. Users should also create new tunnels for each intended output/mock-member, which are used to actually apply for a TxTangle.\footnote{In I2P there are `outbound' and `inbound' tunnels (see \url{https://geti2p.net/en/docs/how/tunnel-routing}). Everything received through a inbound tunnel looks like it's from the same source even if from multiple sources, so on the surface it would appear TxTangle users don't need to create different tunnels for all their usecases. However, if the TxTangle host makes himself the entry point for his own inbound tunnel, then he gains direct access to the outbound tunnels of TxTangle participants.}\footnote{If a large enough volume of users try to TxTangle concurrently, there is likely to be a high rate of failure as users try to randomly put all their intended outputs in the same TxTangle `room', but then the rooms get full too soon so they have to back out. It would be quite a mess. We can make an impactful optimization by telling the host how many outputs we have (e.g. giving him a list of our mock-member public keys), and letting him assemble each TxTangle's participants. Since we still retain the bLSAG and SAG messaging protocol, the host won't be able to identify output groupings in the final transaction. All he knows is the number of participants, and how many outputs each had. Moreover, in this scenario observers can't monitor open TxTangle rooms to deduce information about the participants, an important privacy improvement. Note that the host's power to pollute TxTangles isn't significantly affected, so this change is neutral to that attack vector. We can prevent the host from acting as MITM (`man in the middle') of honest participant interactions (he could do this by serving them fake mock-member lists where he pretends to be all the other participants) by modifying how transaction public keys are made. Participants send each other their intended transaction public keys as normal (with a bLSAG), then, much like robust key aggregation from Section \ref{sec:robust-key-aggregation}, the actual keys that get included in transaction data are prefixed with a hash of the mock-member list and list of original transaction public keys. In other words, $\mathcal{H}_n(T_{agg},\mathbb{S}_{mock},\mathbb{S}_{r-original},r_t*G)*r_t*G$ is the $t$\nth transaction public key. Baking the mock-member list into the transaction itself makes it is very difficult to complete TxTangles without direct communication between all actual participants.}
\end{enumerate}{}


\subsection{Использование хоста в качестве сервиса}
\label{subsec:txtangle-host-service}

Для обеспечения надёжности и постоянного улучшения важно, чтобы сервис TxTangle исполь\-зовался с целью извлечения прибыли.\footnote{Несмотря на то, что развёрнутый сервис TxTangle может использоваться для извлечения прибыли, сам код может быть открытым. Этот аспект важен с точки зрения аудита программного обеспечения кошелька, которое взаимодействует с сервисом TxTangle.} Вместо того чтобы ставить под угрозу идентификаци\-онные данные пользователей, получаемые с помощью модели на основе учётных записей, хост может участвовать в каждой транзакции TxTangle, давая единственный выход, и требовать от участников, чтобы те его финансировали. При получении доступа к сайту eepsite/сервису хоста для создания TxTangles пользователи также получают и уведомление о текущей стои\-мости хостинга, которую им будет необходимо уплатить за каждый выход.

Участники будут нести ответственность за оплату части комиссии {\em и} сбора за организацию хоста. На этот раз самый малый ключ фиктивного участника (за исключением ключа хоста) будет должен оплатить оставшуюся часть комиссии и сбора за организацию хоста.\footnote{В данном случае мы должны использовать ключи фиктивного участника, поскольку хост не платит комиссию, а индекс его выхода неизвестен.} Посколь\-ку у хоста нет входов, у него нет и обязательств по псевдовыходам, чтобы отменить маску обязательства по своему выходу. Вместо этого он, как обычно, вместе с другими фиктивными участниками создает общие секреты, а затем разделяет свою настоящую маску обязательства на части произвольного размера для каждого фиктивного участника и делит их на общие секреты. Он публикует список этих скалярных величин (соотнося их с каждым фиктивным участником на основе их публичных ключей), подписываясь своим ключом фиктивного участ\-ника, чтобы остальные пользователи узнали его от хоста. Появление этого списка сигнализи\-рует о начале раунда 1, описанного в подпункте \ref{subsec:message-rounds-txtangle} (например, об окончании раунда настройки «0»). Фиктивные участники умножают свою скалярную величину, полученную от хоста, на соответствующий общий секрет и добавляют её к своей маске обязательства по псевдовыходу. Таким образом, даже выход хоста не может быть идентифицирован каким-либо из участников окончательной транзакции, если только все они  не объединятся против него.

Чтобы упростить расчёт комиссии, хост может распределить общую комиссию, которая будет использоваться в рамках транзакции, в конце первого раунда, так как он узнает вес транзак\-ции раньше. Участники могут убедиться, что сумма соответствует ожидаемой, и оплатить свою долю.

Если участники объединятся, чтобы обмануть хост, и не внесут плату за хостинг, то хост может завершить проведение TxTangle в 3-м раунде. Он также может завершить процесс, если в канале появляются сообщения, которых там не должно быть или же которые окажутся недействительными.

В конце 5-го раунда хост завершает транзакцию и отправляет её в сеть для верификации, что является частью предоставляемых им услуг. Он включает хеш транзакции в окончательное сообщение о распространении.



\section{Пользование услугами проверенного посредника}
\label{sec:dealer-txtangle}

У децентрализованной модели TxTangle есть некоторые недостатки. Она требует, чтобы все участники активно общались в рамках строгого графика (и для начала находили друг друга), что сложно реализовать.

Привлечение центрального посредника, который бы отвечал за сбор информации о транзак\-циях каждого участника и производил бы обфускацию групп входов/выходов, может упро\-стить процедуру. Ценой в данном случае является более высокий уровень доверия, поскольку посредник должен (как минимум) знать эти группы.\footnote{На написание данного подпункта нас вдохновила схема протокола MoJoin.}


\subsection{Процедура с привлечением посредника}
\label{subsec:dealer-procedure-txtangle}

Посредник должен активно предлагать свои услуги по управлению созданием TxTangles и собирать заявки от потенциальных участников (состоящие из количества предполагаемых входов [с их типами] и выходов). При желании он может участвовать в процессе, используя свой собственный набор входов/выходов.

После того как группа, состоящая почти из 16 выходов, будет собрана (должно быть два или большее количество участников, и ни один участник не может обладать всеми наборами, кроме одного выхода или входа), посредник запускает первый из пяти раундов. При проведе\-нии каждого раунда он собирает информацию, получаемую от каждого участника, принимает некоторые решения и рассылает сообщения, которые означают начало нового раунда.
\begin{enumerate}
    \item Сначала для каждой пары участников посредник генерирует случайную скалярную величину и решает, кому в каждой паре должна принадлежать положительная или отрицательная версия. Чтобы оценить размер общей требуемой комиссии, он использует количество и тип входов и выходов. Он суммирует скалярные величины каждого поль\-зователя и анонимно отправляет каждому их сумму вместе с указанием той части комиссии, которую они должны заплатить, и (случайно выбранными) индексами их выходов. Эти сообщения представляют собой сигнал участникам о начале проведения TxTangle.
    \item Каждый участник строит свою подтранзакцию, как обычно, генерируя отдельные пуб\-личные ключи транзакции для своих выходов (с уменьшением риска проведения атаки Януса по мере необходимости), вычисляя одноразовые адреса выходов и кодируя суммы в выходах, создавая обязательства по псевдовыходам, которые сбалансированы с обяза\-тельствами по выходам и частью комиссии. Так составляется список смещений участни\-ков кольца для использования в подписях MLSAG наряду с соответствующими образами ключей. Кроме того, к одному из обязательств по псевдовыходу добавляется скалярная величина, отправленная посредником (умноженная на $G$). Пользователи создают для своих выходов частичные доказательства Части А и отправляют всю эту информацию посреднику. Посредник проверяет сбалансированность сумм во входах и выходах и отправляет полный список частичных доказательств Части А каждому участнику.
    \item Каждый участник вычисляет совокупный запрос A и генерирует частичные доказатель\-ства Части B, которые отправляются посреднику. Посредник собирает частичные дока\-зательства и распределяет их среди всех остальных участников.
    \item Каждый участник вычисляет совокупный запрос B и генерирует частичные доказатель\-ства Части C, которые отправляются посреднику. Посредник собирает их и применяет метод логарифмического внутреннего произведения, чтобы получить окончательное до\-казательство. Если доказательство проходит проверку должным образом, он генерирует случайный фальшивый публичный ключ «базовой» транзакции, снижающий риск атаки Януса, и отправляет каждому участнику сообщение для подписания MLSAG.
    \item Каждый участник завершает подписание MLSAG и отправляет всё посреднику. Как только посредник получит все части, он сможет завершить построение транзакции и отправить её для включения в блокчейн. Он также может отправить идентификатор транзакции каждому из участников, чтобы они могли подтвердить, что она была опубли\-кована.
\end{enumerate}{}