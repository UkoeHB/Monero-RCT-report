\chapter{Продвинутые подписи Шнорра}
\label{chapter:advanced-schnorr}

В своём базовом варианте подпись Шнорра имеет один подписывающий ключ. Как оказалось, эту ключевую концепцию можно применить для создания целого множества всё более слож\-ных схем подписи. Одна из этих схем, MLSAG, занимает центральное по важности место в протоколе транзакций Monero.



%-same signing key across multiple bases
\section{Доказательство знания дискретного логарифма на основе множества «базовых» ключей}
\label{sec:proofs-discrete-logarithm-multiple-bases}

Зачастую бывает полезно доказать, что один и тот же приватный ключ использовался для создания различных публичных ключей на основе различных «базовых» ключей. Например, у нас может иметься обычный публичный ключ $k G$ и общий секрет Диффи-Хеллмана $k R$ с публичным ключом какого-либо другого человека (см. подпункт \ref{DH_exchange_section}), где базовыми ключами являются $G$ и $R$. Как можно будет скоро увидеть, мы можем доказать знание дискретного логарифма $k$ в $k G$, доказать знание $k$ в $k R$, {\em и} доказать, что $k$ является одним и тем же в обоих случаях (при этом $k$ никак не раскрывается).


\subsection*{Неинтерактивное доказательство}

Допустим, у нас есть приватный ключ $k$ и $d$ базовых ключей $\mathcal{J} = \{J_1,...,J_d\}$. Соответствую\-щими публичными ключами будут $\mathcal{K} = \{K_1,...,K_d\}$. Создаём доказательство в стиле Шнорра (см. подпункт \ref{sec:schnorr-fiat-shamir}) по всем базовым ключам.\footnote{Несмотря на то, что мы говорим «доказательство», можно просто создать подпись, включив сообщение $\mathfrak{m}$ в хеш запроса. В этом контексте термины являются взаимозаменимыми.} Допустим, существует хеш-функция \(\mathcal{H}_n\), которая распределяется по целым числам в диапазоне от 0 до $l-1$.\footnote{В случае с Monero хеш-функция $\mathcal{H}_n(x) = \textrm{sc\textunderscore reduce32}(\mathit{Keccak}(x))$, где $\mathit{Keccak}$ будет основой SHA3, а sc\textunderscore reduce32() определит 256-битный результат в диапазоне от 0 до $l-1$ (хотя, по сути, результат должен находиться в диапазоне от 1 до $l-1$).} 

\begin{enumerate}
	\item Генерируем случайное число $\alpha \in_R \mathbb{Z}_l$ и вычисляем для всех $i \in (1,...,d)$, $\alpha J_i$.
	\item Вычисляем запрос:\vspace{.175cm}
	\[c = \mathcal{H}_n(\mathcal{J},\mathcal{K},[\alpha J_1],[\alpha J_2],...,[\alpha J_d])\]
	\item Определяем ответ: $r = \alpha - c*k$.
	\item Публикуем подпись $(c, r)$.
\end{enumerate}


\subsection*{Верификация}

Допустим, верификатору известны $\mathcal{J}$ и $\mathcal{K}$, и он делает следующее.

\begin{enumerate}
	\item Вычисляет запрос:\vspace{.175cm}
	\[c' = \mathcal{H}(\mathcal{J},\mathcal{K},[r J_1 + c*K_1],[r J_2 + c*K_2],...,[r J_d + c*K_d])\]
	\item Если $c = c'$, значит, подписанту должен быть известен дискретный логарифм по всем базовым ключам, и в каждом случае это будет один и тот же дискретный логарифм (как и всегда, за исключением ничтожной вероятности).
\end{enumerate}


\subsection*{Почему это работает}

Если бы вместо $d$ базовых ключей был всего один, очевидно, что доказательство было бы тем же, что и наше оригинальное доказательство Шнорра (см. подпункт \ref{sec:schnorr-fiat-shamir}). Мы можем представить каждый базовый ключ отдельно, и это позволит нам увидеть, что доказательство по множеству базовых ключей является просто совокупностью доказательств Шнорра, объеди\-нённых в одну группу. Более того, при использовании только одного запроса и ответа для всех этих доказательств, у них будет один и тот же дискретный логарифм $k$. Чтобы получить отдельный ответ, который будет работать в случае с множеством ключей, запрос должен быть известен до того, как $\alpha$ будет определена для каждого ключа, но $c$ при этом должна являться функцией $\alpha$!



%-multiple signing keys on their own unique bases (or they can be the same e.g. G)
\section{Множество приватных ключей в одном доказательстве}
\label{sec:multiple_private_keys_in_one_proof}

Во многом, как и в случае с доказательствами по множеству базовых ключей, мы можем объединить доказательства Шнорра, использующие различные приватные ключи. Это позво\-ляет доказать, что нам известны все приватные ключи для набора публичных ключей, и сокращает место, необходимое для хранения, так как для всех доказательств требуется всего один запрос.


\subsection*{Неинтерактивное доказательство}

Допустим, у нас есть $d$ базовых ключей $k_1,...,k_d$ и базовые ключи $\mathcal{J} = \{J_1,...,J_d\}$.\footnote{В данном случае нет никакой причины, по которой $\mathcal{J}$ не мог бы содержать дублирующие друг друга базовые ключи или же был одинаковым для всех базовых ключей (например, $G$). В случае с доказательствами по множеству базовых ключей «дубликаты» будут избыточными, но здесь мы имеем дело с другими приватными ключами.} Соответ\-ствующими публичными ключами будут $\mathcal{K} = \{K_1,...,K_d\}$. Создаём доказательство в стиле Шнорра одновременно для всех ключей.

\begin{enumerate}
	\item Генерируем случайные числа $\alpha_i \in_R \mathbb{Z}_l$ для всех $i \in (1,...,d)$ и вычисляем все $\alpha_i J_i$.
	\item Вычисляем запрос:\vspace{.175cm}
	\[c = \mathcal{H}_n(\mathcal{J},\mathcal{K},[\alpha_1 J_1],[\alpha_2 J_2],...,[\alpha_d J_d])\]
	\item Определяем каждый ответ: $r_i = \alpha_i - c*k_i$.
	\item Публикуем подпись $(c, r_1,...,r_d)$.
\end{enumerate}


\subsection*{Верификация}

Допустим, верификатору известны $\mathcal{J}$ и $\mathcal{K}$, и он делает следующее.

\begin{enumerate}
	\item Вычисляет запрос:\vspace{.175cm}
	\[c' = \mathcal{H}(\mathcal{J},\mathcal{K},[r_1 J_1 + c*K_1],[r_2 J_2 + c*K_2],...,[r_d J_d + c*K_d])\]
	\item Если $c = c'$, значит, подписанту должны быть известны приватные ключи для всех публичных ключей в $\mathcal{K}$ (за исключением ничтожной вероятности).
\end{enumerate}



\section{Подписи спонтанной анонимной группы (SAG)}
\label{SAG_section}

Групповые подписи являются способом доказательства того, что подписант принадлежит к группе, без его идентификации. Изначально (см. Чаум (Chaum) \cite{Chaum:1991:GS:1754868.1754897}) схемы групповых подписей требовали настройки системы, а в некоторых случаях и управления доверенным лицом с целью предотвращения использования незаконных подписей и, в случае с некоторыми схемами, разрешения спорных ситуаций. Использование {\em группового секрета} не желательно, так как создается риск раскрытия, что ставит под угрозу анонимность. Кроме того, необходи\-мость в координации между членами группы (то есть в настройке и управлении) не масштаби\-руется, независимо от того, происходит это в пределах небольших групп или внутри компаний.

Лю {\em и др.} (Liu et al.) в работе \cite{Liu2004} была представлена более интересная схема, основанная на исследованиях Ривеста {\em и др.} (Rivest et al.) \cite{rivest-leak-secret}. Авторами был подробно изложен алгоритм формирования групповых подписей под названием LSAG, характеризуемый тремя свойствами: {\em анонимность, связываемость} и {\em спонтанность}. Для простоты понимания кон\-цепции в данном случае нами рассматривается SAG, несвязываемая версия LSAG. Рассмотре\-ние идеи связываемости мы оставляем для последующих разделов.
\\

Схемы, обладающие свойствами анонимности и спонтанности, как правило, называются «коль\-цевыми подписями». В контексте Monero в конечном счёте они обеспечивают возможность проведения транзакций с сокрытием подписанта и невозможностью подделки, что делает потоки валюты по большей части неотслеживаемыми.


\subsection*{Подпись}

Кольцевые подписи состоят из кольца и подписи. Каждое {\em кольцо} является набором публич\-ных ключей, один из которых принадлежит подписанту, а остальные не связаны между собой. {\em Подпись} генерируется при помощи этого кольца, состоящего из ключей, и верификатор нико\-гда не сможет точно сказать, кто из участников кольца является действительным подписан\-том.

Нашу схему в стиле Шнорра, описанную в подпункте \ref{sec:signing-messages}, можно рассматривать в качестве кольцевой подписи с одним ключом. Мы приходим к использованию двух ключей путём создания ложного $r$ (вместо определения $r'$) и создания нового запроса для определения $r$.
\\

Допустим, \(\mathfrak{m}\) является подписываемым сообщением, \(\mathcal{R} = \{K_1, K_2, ..., K_n\}\) является набором определённых публичных ключей (группой/кольцом), а \(k_\pi\) является приватным ключом под\-писанта, соответствующим его публичному ключу \(K_\pi \in \mathcal{R}\), где $\pi$ является секретным индек\-сом.

\begin{enumerate}
	\item Генерируем случайное число \(\alpha \in_R \mathbb{Z}_l\) и ложные ответы \(r_i \in_R \mathbb{Z}_l\) для \(i \in \{1, 2, ..., n\}\), но за исключением \(i = \pi\).

	\item Вычисляем
	\[c_{\pi+1} = \mathcal{H}_n(\mathcal{R}, \mathfrak{m}, [\alpha G])\]

	\item Для \(i = \pi+1, \pi+2, ..., n, 1, 2, ..., \pi-1\), заменив \(n + 1 \rightarrow 1\),  вычисляем\vspace{.175cm}
	\[  c_{i+1} = \mathcal{H}_n(\mathcal{R}, \mathfrak{m}, [r_i G + c_i K_i])\] 

	\item Определяем такой реальный ответ $r_\pi$, чтобы \(\alpha = r_\pi + c_\pi k_\pi \pmod l\).
\end{enumerate}

Кольцевая подпись содержит подпись \(\sigma(\mathfrak{m}) = (c_1, r_1, ..., r_n) \) и кольцо $\mathcal{R}$.


\subsection*{Верификация}

Процесс верификации означает доказательство того, что $\sigma(\mathfrak{m})$ является действительной под\-писью, созданной с использованием приватного ключа, соответствующего публичному ключу в кольце $\mathcal{R}$ (при этом не обязательно известно, какому именно), и производится следующим образом:

\begin{enumerate}
	\item Для \(i = 1, 2, ..., n\), заменив \(n + 1 \rightarrow 1\), итеративно вычисляем\vspace{.175cm}
	\begin{align*}
	c'_{i+1}   = \mathcal{H}_n(\mathcal{R}, \mathfrak{m}, [r_i G + c_i {K_i}])
	\end{align*}

	\item Если \(c_1 = c'_1\), то подпись является действительной. Следует отметить, что $c'_1$ вычисляет\-ся в последнюю очередь.
\end{enumerate}

В рамках этой схемы мы сохраняем (1+$n$) целых чисел и используем $n$ публичных ключей.


\subsection*{Почему это работает}

Мы можем убедиться в том, что алгоритм работает, рассмотрев следующий пример. Возьмём кольцо $R = \{K_1, K_2, K_3\}$, в котором $k_\pi = k_2$. Сначала идёт подпись:
\begin{enumerate}
    \item Генерируем случайные числа: $\alpha$, $r_1$, $r_3$
\begin{align*}
    \intertext{\item Вводим цикл подписи:}	c_3 &= \mathcal{H}_n(\mathcal{R}, \mathfrak{m}, [\alpha G])
    \intertext{\item Производим итерацию: \vspace{-.2cm}}
        c_1 &= \mathcal{H}_n(\mathcal{R}, \mathfrak{m}, [r_3 G + c_3 K_3])\\
        c_2 &= \mathcal{H}_n(\mathcal{R}, \mathfrak{m}, [r_1 G + c_1 K_1])
\end{align*}
    \item Замыкаем цикл ответом: $r_2 = \alpha - c_2 k_2 \pmod{l}$
\end{enumerate}

Мы можем заменить $\alpha$ на $c_3$, чтобы понять, откуда взялось слово «кольцо»:\vspace{.175cm}
\begin{alignat*}{3}
    c_3 &= \mathcal{H}_n(\mathcal{R}, \mathfrak{m}, [(r_2 + c_2 k_2) G &&])\\
    c_3 &= \mathcal{H}_n(\mathcal{R}, \mathfrak{m}, [r_2 G + c_2 K_2 &&])
\end{alignat*}\vspace{.05cm}

Затем производится верификация с использованием $\mathcal{R}$ и $\sigma(\mathfrak{m}) = (c_1, r_1, r_2, r_3)$:
\begin{enumerate}
    \item Используем $r_1$ и $c_1$ для вычисления\vspace{.175cm}
    \begin{align*}
c'_2 &= \mathcal{H}_n(\mathcal{R}, \mathfrak{m}, [r_1 G + c_1 K_1])
    \intertext{\item С момента создания подписи мы видим, что $c'_2 = c_2$. Используя $r_2$ и $c'_2$ , вычисляем\vspace{.175cm}}
c'_3 &= \mathcal{H}_n(\mathcal{R}, \mathfrak{m}, [r_2 G + c'_2 K_2])
    \intertext{\item Заменив $c_2$ на $c'_2$, мы увидим, что $c'_3 = c_3$. Используя $r_3$ и $c'_3$ , получаем\vspace{.175cm}}
c'_1 &= \mathcal{H}_n(\mathcal{R}, \mathfrak{m}, [r_3 G + c'_3 K_3])
    \end{align*}
\end{enumerate}
\quad В данном случае ничего удивительного: если заменить $c_3$ на $c'_3$, то $c'_1 = c_1$.\vspace{-.3cm}



\section{Связываемые подписи спонтанной анонимной группы Бэка (bLSAG)}
\label{blsag_note}

Кольцевые подписи, рассматриваемые здесь, демонстрируют несколько свойств, которые будут полезны при создании конфиденциальных транзакций.\footnote{Следует помнить о том, что все устойчивые схемы подписей имеют модели безопасности, обладающие различными свойствами. Свойства, упомянутые здесь, вероятно, будут наиболее важными с точки зрения понимания предназначения кольцевых подписей Monero, но они не могут служить для полного описания свойств связываемой кольцевой подписи.} Необходимо отметить, что как понятие «сокрытия подписанта», так и «невозможности подделки» также относятся к подписям SAG.

\begin{description}
	\item[Сокрытие подписанта]
	Наблюдатель должен быть в состоянии определить, что подписант является участником кольца, но не должен знать, каким именно\footnote{\label{anonymity_note}Анонимность в случае совершения какого-либо действия с точки зрения «анонимной группы», которая включает в себя всех людей, которые могли совершить такое действие. Самой большой анонимной группой является «человечество», а в случае с Monero это размер кольца или, например, так называемый «уровень смешивания» $v$ плюс реальный подписант. Термин «смешивание» подразумевает количество ложных участников в каждой кольцевой подписи. Если значение смешивания составляет $v$ = 4, значит, присутствует 5 возможных подписантов. Расширение анонимной группы усложняет процесс определения реальных участников транзакции.} Monero использует это для маскировки источника средств при проведении каждой транзакции.

	\item[Связываемость]
	Если приватный ключ используется для того, чтобы подписать два различ\-ных сообщения, то эти сообщения становятся связанными.\footnote{\label{linkability_note}Свойство связываемости не имеет отношения к публичным ключам, не используемым для подписания. То есть участник кольца, чей публичный ключ был смешан в других подписях, не будет связан.} Как будет показано далее, это свойство помогает предотвратить атаки, связанные с двойной тратой Monero (за исключением ничтожной вероятности).

	\item[Невозможность подделки]
    Никакой злоумышленник не сможет сгенерировать фальшивую подпись, кроме как с ничтожной вероятностью.\footnote{\label{unforgeability_note}Определенные схемы кольцевых подписей, включая ту, что используется Monero, довольно устойчивы к атакам, осуществляемой путём адаптивной выборки сообщений и адаптивной выборки публичного ключа. Злоумышленник, который может получить действительные подписи для сообщений и соответствующие публичные ключи в случае с такими кольцами, не сможет определить, как сформировать подпись даже для одного сообщения. Это называется {\em экзистенциальной невозможностью подделки}, см. \cite{MRL-0005-ringct} и \cite{Liu2004}.} Это не позволяет похитить Monero тем, кто не владеет соответствующими приватными ключами.
\end{description}

В соответствии со схемой подписи LSAG \cite{Liu2004} владелец приватного ключа может создавать одну анонимную несвязанную подпись на кольцо.\footnote{\label{lsag_linkability_note}В рамках схемы LSAG связываемость реализуется в случае только с теми подписями, которые используют кольца с теми же участниками и в том же самом порядке, то есть это должно быть «абсолютно то же самое кольцо». Фактически это «одна анонимная подпись на кольцо». Подписи могут быть связаны, даже если были созданы для разных сообщений.} В этом разделе нами предлагается улуч\-шенная версия алгоритма LSAG, где свойство связываемости не зависит от ложных участни\-ков кольца.\footnote{Схема LSAG рассматривалась в первой редакции данного отчёта. \cite{ztm-1}}

Модификация была предложена в работе \cite{MRL-0005-ringct}, основанной на публикации Адама Бэка (Adam Back) \cite{AdamBack-ring-efficiency}, касающейся алгоритма кольцевой подписи CryptoNote \cite{cryptoNoteWhitePaper} (ранее использовался Monero, но теперь от него отказались; см. подпункт \ref{subsec:proofs-input-creation-spendproof}), в основу которого, в свою очередь, легла работа Фуджисаки (Fujisaki) и Сузуки (Suzuki) \cite{Fujisaki2007}.


\subsection*{Подпись}

Как и в случае с SAG, допустим, что \(\mathfrak{m}\) является подписываемым сообщением, \linebreak \(\mathcal{R} = \{K_1, K_2, ..., K_n\}\) является набором определённых публичных ключей, а \(k_\pi\) является приватным ключом подписанта, соответствующим его публичному ключу \(K_\pi \in \mathcal{R}\), где $\pi$ является секретным индексом. Также допустим наличие хеш-функции \(\mathcal{H}_p\), отвечающей за распределение точек на кривой EC.\footnote{Совершенно неважно, буду точки $\mathcal{H}_p$ сжатыми или нет. Они всегда могут быть развёрнуты.}\footnote{Monero использует хеш-функцию\marginnote{src/ringct/ rctOps.cpp {\tt hash\_to\_p3()}}, которая выдаёт точки кривой напрямую, а не путём вычисления какого-либо целого числа, которое затем умножается на $G$. $\mathcal{H}_p$ была бы разбита, если бы кто-то нашёл способ найти такое $n_x$, чтобы $n_x G = \mathcal{H}_p(x)$. Описание алгоритма приводится в работе \cite{hashtopoint-writeup}. В соответствии с белой книгой CryptoNote \cite{cryptoNoteWhitePaper} изначально он был представлен в работе \cite{hashtopoint-original-paper}.}

\begin{enumerate}
	\item Вычисляем образ ключа \(\tilde{K} = k_\pi \mathcal{H}_p(K_\pi)\).\footnote{В случае с Monero для образов ключей важно использовать функцию преобразования хеша в точку вместо другой базовой точки. Таким образом, линейность не приведёт к связыванию подписей, созданных по одному и тому же адресу (даже в том случае, если они были созданы для разных одноразовых адресов). См. \cite{cryptoNoteWhitePaper}, стр. 18.}

	\item Генерируем случайное число \(\alpha \in_R \mathbb{Z}_l\) и случайные числа \(r_i \in_R \mathbb{Z}_l\) для \(i \in \{1, 2, ..., n\}\), но за исключением \(i = \pi\).

	\item Вычисляем
	\[c_{\pi+1} = \mathcal{H}_n(\mathfrak{m}, [\alpha G], [\alpha \mathcal{H}_p(K_\pi)])\]

	\item Для \(i = \pi+1, \pi+2, ..., n, 1, 2, ..., \pi-1\) заменив \(n + 1 \rightarrow 1\), вычисляем\vspace{.175cm}
	\[c_{i+1} = \mathcal{H}_n(\mathfrak{m}, [r_i G + c_i K_i], [r_i \mathcal{H}_p(K_i) + c_i \tilde{K}])\]

	\item Определяем \(r_\pi = \alpha - c_\pi k_\pi \pmod l\).
\end{enumerate}

Кольцевой подписью будет \(\sigma(\mathfrak{m}) = (c_1, r_1, ..., r_n)\) с образом ключа $\tilde{K}$ и кольцом $\mathcal{R}$.


\subsection*{Верификация}

Процесс верификации означает доказательство того, что $\sigma(\mathfrak{m})$ является действительной под\-писью, созданной с использованием приватного ключа, соответствующего публичному ключу в кольце $\mathcal{R}$, и производится следующим образом:

\begin{enumerate}
    \item Проверяем $l \tilde{K} \stackrel{?}{=} 0$.
	\item Для \(i = 1, 2, ..., n\), заменив \(n + 1 \rightarrow 1\), итеративно вычисляем\vspace{.175cm}
	\begin{align*}
	c'_{i+1} = \mathcal{H}_n(\mathfrak{m}, [r_i G + c_i {K_i}], [r_i \mathcal{H}_p(K_i) + c_i \tilde{K}])
	\end{align*}

	\item Если \(c_1 = c'_1\), то подпись является действительной.
\end{enumerate}

В рамках этой схемы мы сохраняем (1+$n$) целых чисел, имеем один образ ключа EC и используем $n$ публичных ключей.

Нам\marginnote{src/crypto- note\_core/ cryptonote\_ core.cpp {\tt check\_tx\_ inputs\_key- images\_do- main()}} необходимо проверить $l \tilde{K} \stackrel{?}{=} 0$, поскольку есть возможность добавить точку EC из подгруппы с размером $h$ (кофактор) в $\tilde{K}$ и, если все $c_i$ будут кратными $h$ (чего можно достичь при помощи автоматизированного последовательного приближения с использованием различных значений $\alpha$ и $r_i$), создать $h$ несвязанных действительных подписей при помощи того же кольца и подписывающего ключа.\footnote{Нас не интересуют точки из других подгрупп, поскольку выход $\mathcal{H}_n$ ограничивается до $\mathbb{Z}_l$. Для EC порядка $N = h l$ все делители $N$ (а следовательно, и возможные подгруппы) будут либо кратными $l$ (простому числу), либо будут делителями $h$.} Причина заключается в том, что точка EC, помноженная на порядок своей подгруппы, будет равна нулю.\footnote{На ранних этапах истории Monero это не проверялось. К счастью, этим никто не воспользовался до того момента, когда в апреле 2017 были внесены соответствующие исправления (была реализована версия v5 протокола) \cite{key-image-bug}.}

Для ясности: при наличии точки $K$ в подгруппе порядка $l$, некоторой точки $K^h$ порядка $h$ и целого числа $c$, делимого на $h$:
\begin{align*}
    c*(K + K^h) &= cK + cK^h\\
                &= cK + 0
\end{align*}

Мы демонстрируем правильность (то есть «как это работает») подобно тому, как это делали в случае с более простой схемой подписи SAG.

В нашем описании мы пытаемся придерживаться оригинального объяснения bLSAG, которое не включает $\mathcal{R}$ в хеш, используемый для вычисления $c_i$. Включение ключей в хеш известно как «добавление префикса ключа». Недавние исследования \cite{key-prefix-paper} позволяют предположить, что в этом нет необходимости, хотя добавление префикса является стандартной практикой при использовании подобных схем подписи (LSAG использует добавление префикса ключа).


\subsection*{Связываемость}

При наличии двух действительных подписей, отличающихся некоторым образом (например, имеющих разные ложные ответы, различные сообщения, различное общее количество участни\-ков кольца)\vspace{.1cm}
\begin{align*}
	\sigma(\mathfrak{m})   &= (c_1, r_1, ..., r_n)\textrm{ with } \tilde{K}\textrm{, and}\\
	\sigma'(\mathfrak{m}')  &= (c_1', r'_1, ..., r'_{n'})\textrm{ with } \tilde{K}'\textrm{,}
\end{align*}
\quad и если \(\tilde{K} =  \tilde{K}'\), значит, в основе обеих подписей лежит один и тот же приватный ключ.% because $\tilde{K}= k_{\pi} \mathcal{H}_p(\mathcal{R})$.

Несмотря на то, что внешний наблюдатель сможет связать $\sigma$ и $\sigma'$, совсем не обязательно ему будет известно, какое $K_i$ в $\mathcal{R}$ или $\mathcal{R}'$ было виновным, если между ними был только один общий ключ. При наличии более одного общего участника кольца наблюдателю останется только решить DLP или каким-либо образом провести аудит колец (например, узнать все $k_i$, где $i \neq \pi$, или узнать $k_\pi$).\footnote{\label{lsag_unforgeable_note}LSAG, как и bLSAG, не поддаётся поделке, поэтому никакой злоумышленник не сможет сгенерировать действительную кольцевую подпись, не зная приватного ключа. Если он придумает фальшивый образ $\tilde{K}$ и запустит вычисление подписи, используя $c_{\pi+1}$, то, не зная $k_\pi$, он не сможет вычислить число $r_\pi = \alpha - c_\pi k_\pi$, которое позволит получить $[r_\pi G + c_\pi K_\pi] = \alpha G$. Верификатор отклонит его подпись. В работе Лю {\em и др.} доказано, что создание подделок, которые могли бы пройти верификацию, крайне маловероятно \cite{Liu2004}.}



\section{Многоуровневые связываемые подписи спонтанной аноним\-ной группы (MLSAG)}
\label{sec:MLSAG}

Для того чтобы подписать транзакцию, понадобится множество приватных ключей. В работе \cite{MRL-0005-ringct}, Шена Нётера {\em и др.} описывают многоуровневое обобщение схемы подписей bLSAG, которое может применяться в случае наличия ряда \(n \cdot m\) ключей, то есть набора\vspace{.175cm}
\[\mathcal{R} = \{K_{i,j}\}  \quad \textrm{для} \quad  i \in \{1, 2, ..., n\} \quad \textrm{и} \quad j \in \{1, 2, ..., m\}\]

для которого нам известно $m$ приватных ключей \(\{k_{\pi, j}\}\), соответствующих поднабору \(\{K_{\pi, j}\}\), для некоторого индекса \(i = \pi\). Такой алгоритм удовлетворил бы наши потребности, если бы мы обобщили понятие связываемости.
\begin{description}
	\item[Связываемость] Если какой-либо из приватных ключей \(k_{\pi, j}\) используется в двух разных подписях, то эти подписи будут связаны автоматически.
\end{description}


\subsection*{Подпись}

\begin{enumerate}
	\item Вычисляем образы ключей \(\tilde{K_j} = k_{\pi, j} \mathcal{H}_p(K_{\pi, j})\) для всех \(j \in \{1, 2, ..., m\}\).

	\item Генерируем случайные числа \(\alpha_j \in_R \mathbb{Z}_l\) и \(r_{i, j} \in_R \mathbb{Z}_l\) для \(i \in \{1, 2, ..., n\}\) (за исключением \(i = \pi\)) и \(j \in \{1, 2, ..., m\}\).

	\item Вычисляем\footnote{MLSAG-подпись Monero\marginnote{src/ringct/ rctSigs.cpp {\tt MLSAG\_Gen()}} использует добавление префикса ключа. Каждый запрос содержит чётко заданные публичные ключи, подобные этому (добавление элементов $K$ отсутствует в bLSAG; в подписанные сообщения включаются образы ключей):\vspace{-.25cm}
	\[c_{\pi+1} = \mathcal{H}_n(\mathfrak{m}, K_{\pi, 1}, [\alpha_1 G], [\alpha_1 \mathcal{H}_p(K_{\pi, 1})], ..., K_{\pi, m}, [\alpha_m G], [\alpha_m \mathcal{H}_p(K_{\pi, m})])
	\]}
	\[c_{\pi+1} = \mathcal{H}_n(\mathfrak{m}, [\alpha_1 G], [\alpha_1 \mathcal{H}_p(K_{\pi, 1})], ..., [\alpha_m G], [\alpha_m \mathcal{H}_p(K_{\pi, m})])\]

	\item Для \(i = \pi+1, \pi+2, ..., n, 1, 2, ..., \pi-1\), заменив \(n + 1 \rightarrow 1\), вычисляем\vspace{.175cm}
	\[ c_{i+1} = \mathcal{H}_n(\mathfrak{m}, [r_{i, 1} G + c_i K_{i, 1}], [r_{i, 1} \mathcal{H}_p(K_{i, 1}) + c_i \tilde{K}_1], 
	..., [r_{i, m} G + c_i K_{i, m}], [r_{i, m} \mathcal{H}_p(K_{i, m}) + c_i \tilde{K}_m])\]

	\item Определяем все \(r_{\pi, j} = \alpha_j - c_\pi k_{\pi, j} \pmod l\).
\end{enumerate}

Кольцевой подписью будет \(\sigma(\mathfrak{m}) = (c_1, r_{1, 1}, ..., r_{1, m}, ..., r_{n, 1}, ..., r_{n, m}) \) с образами ключей \linebreak $(\tilde{K}_1, ...,  \tilde{K}_m)$.

%One way to think about MLSAG is that there are $m$ sub-loops of size $n$, and in each sub-loop we know a private key at index $i = \pi$ ($m \cdot n$ total public keys). The signature algorithm ties together a ‘stack’ of keys at each stage $c$, composed of one key from each sub-loop. bLSAG is the special case where $m = 1$.


\subsection*{Верификация}

Верификация подписи производится следующим образом:

\begin{enumerate}
    \item Для всех $j \in \{1,...,m\}$ проверяем $l \tilde{K}_j \stackrel{?}{=} 0$.
	\item Для \(i = 1, ..., n\), заменив \(n + 1 \rightarrow 1\), вычисляем\vspace{.175cm}
	\begin{align*}
	c'_{i+1} = \mathcal{H}_n(\mathfrak{m}, [r_{i, 1} G + c_i K_{i, 1}], [r_{i, 1} \mathcal{H}_p(K_{i, 1}) + c_i \tilde{K}_1], 
	..., [r_{i, m} G + c_i K_{i, m}], [r_{i, m} \mathcal{H}_p(K_{i, m}) + c_i \tilde{K}_m])
	\end{align*}

	\item Если \(c_1 = c'_1\), то подпись является действительной.
\end{enumerate}


\subsection*{Почему это работает}

Как и в случае с оригинальным алгоритмом SAG, мы видим, что:

\begin{itemize}
    \item если \(i \ne \pi \), то очевидно значения \(c'_{i + 1}\) вычисляются, как описано в алгоритме подписи;

    \item если \(i = \pi\), то, \(r_{\pi, j} = \alpha_j - c_\pi k_{\pi, j} \) закрывает цикл,\vspace{.175cm}
    \begin{alignat*}{6}
        r_{\pi, j} G + c_\pi K_{\pi,j} &= (\alpha_j - c_\pi k_{\pi, j}) G + c_\pi K_{\pi,j} = \alpha_j G\\
        \intertext{и}
        r_{\pi, j} \mathcal{H}_p(K_{\pi, j}) + c_\pi \tilde{K}_j &= (\alpha_j - c_\pi k_{\pi, j}) \mathcal{H}_p(K_{\pi, j}) + c_\pi \tilde{K}_j = \alpha_j \mathcal{H}_p(K_{\pi, j})\\
    \end{alignat*}
    Другими словами, также будет действительным \(c'_{\pi + 1} = c_{\pi+1}\).
\end{itemize}


\subsection*{Связываемость}

Если для создания какой-либо подписи повторно используется приватный ключ \(k_{\pi, j}\) , соответ\-ствующий образ ключа \(\tilde{K}_j\), передаваемый вместе с подписью, выявит его. Это соответствует нашему обобщённому определению связываемости.\footnote{Как и в случае с bLSAG, связанные подписи MLSAG не указывают, какой публичный ключ использовался для подписания. Тем не менее, если подписи в подциклах связующего образа ключа имеют только один общий ключ, виновник становится очевидным. Если виновник выявлен, все остальные участники обеих подписей раскрываются, так как они все используют общие индексы виновника.}


\subsection*{Требования к занимаемому месту}

В рамках этой схемы мы сохраняем (1+$m*n$) целых чисел, имеем $m$ образов ключей EC и используем $m*n$ публичных ключей.



\section{Компактные  связываемые подписи спонтанной анонимной группы (CLSAG)}
\label{sec:CLSAG}

Подписи CLSAG \cite{MRL-0011-CLSAG}\footnote{В основе данного раздела лежит работа, являющаяся препринтом, который будет вскоре окончательно оформлен для внешней редакции. Схема CLSAG в перспективе может заменить MLSAG с выходом новых версий протокола, но пока она не реализована, и, возможно, этого и не произойдёт в будущем.} находятся где-то между bLSAG и MLSAG. Предположим, у вас имеется «первичный» ключ, а с ним связано несколько «вспомогательных» ключей. Важно доказать знание всех приватных ключей, но свойство связываемости применимо только к первичному. Такое ограничение по связываемости позволяет создавать меньшие по размеру и более бы\-стрые подписи, чем в случае со схемой MLSAG.

Как и в случае с MLSAG у нас имеется набор из \(n \cdot m\) ключей (где $n$ является размером кольца, а $m$ — количеством подписывающих ключей), а первичные ключи находятся по индексу 1. Другими словами, существует $n$ первичных ключей, и такой ключ $\pi$\nth и его вспомогательные ключи будут использоваться для подписания.\vspace{.175cm}
\[\mathcal{R} = \{K_{i,j}\}  \quad \textrm{для} \quad  i \in \{1, 2, ..., n\} \quad \textrm{и} \quad j \in \{1, 2, ..., m\}\]

Нам известны приватные ключи \(\{k_{\pi, j}\}\), соответствующие поднабору \(\{K_{\pi, j}\}\), для некоторого индекса \(i = \pi\).


\subsection*{Подпись}

\begin{enumerate}
	\item Вычисляем образы ключей для \(\tilde{K_j} = k_{\pi, j} \mathcal{H}_p(K_{\pi, 1})\) всех \(j \in \{1, 2, ..., m\}\). Следует отме\-тить, что базовый ключ всегда будет одним и тем же, поэтому образы ключей, где $j>1$, будут «образами вспомогательных ключей». Для простоты обозначения мы называем их $\tilde{K}_j$.

	\item Генерируем случайные числа  \(\alpha \in_R \mathbb{Z}_l\) и \(r_{i} \in_R \mathbb{Z}_l\) для \(i \in \{1, 2, ..., n\}\) (за исключением \(i = \pi\)).

    \item Вычисляем совокупные публичные ключи $W_i$ для \(i \in \{1, 2, ..., n\}\) и образ совокупного ключа $\tilde{W}$\footnote{В документе CLSAG указано, что для разделения доменов следует использовать  разные хеш-функции, что моделируется нами путём добавления тега к каждому хешу в виде префикса \cite{MRL-0011-CLSAG}, например, $T_1 =$ ``CLSAG\_1", $T_c =$ ``CLSAG\_c" и так далее. Разделённые по доменам хеш-функции дают различные выходы даже при наличии одинаковых входов. Также в этом случае мы используем добавление префиксов ключей (включая $\mathcal{R}$, в который входят все ключи в хеше). Разделение доменов является новой политикой, используемой при разработке Monero, и, вероятнее всего, будет применяться в случае любого использования хеш-функций в будущем (в версиях после v13). Исторические способы использования хеш-функций, вероятно, останутся в прошлом.}%an actual implementation may use $j$ flags, instead of the index itself
    \begin{align*}
    W_i &= \sum^{m}_{j=1} \mathcal{H}_n(T_j, \mathcal{R}, \tilde{K}_1,...,\tilde{K}_{m})*K_{i,j}\\
    \tilde{W} &= \sum^{m}_{j=1} \mathcal{H}_n(T_j, \mathcal{R}, \tilde{K}_1,...,\tilde{K}_{m})*\tilde{K}_j
    \end{align*}{}
    где $w_{\pi} = \sum_j \mathcal{H}_n(T_j,...)*k_{\pi,j}$ является совокупным приватным ключом.

	\item Вычисляем
	\[c_{\pi+1} = \mathcal{H}_n(T_c, \mathcal{R}, \mathfrak{m}, [\alpha G], [\alpha \mathcal{H}_p(K_{\pi, 1})])\]

	\item Для \(i = \pi+1, \pi+2, ..., n, 1, 2, ..., \pi-1\), заменив \(n + 1 \rightarrow 1\), вычисляем\vspace{.175cm}
	\[c_{i+1} = \mathcal{H}_n(T_c, \mathcal{R}, \mathfrak{m}, [r_i G + c_i W_i], [r_{i} \mathcal{H}_p(K_{i,1}) + c_i \tilde{W}])\]

	\item Определяем \(r_{\pi} = \alpha - c_\pi w_\pi \pmod l\).
\end{enumerate}

Следовательно, \(\sigma(\mathfrak{m}) = (c_1, r_1, ..., r_n) \) с образом первичного ключа $\tilde{K}_1$ и вспомогательными образами $(\tilde{K}_2,...,\tilde{K}_{m})$.


\subsection*{Верификация}

Верификация подписи производится следующим образом.

\begin{enumerate}
    \item Для всех $j \in \{1,...,m\}$ проверяем $l \tilde{K}_j \stackrel{?}{=} 0$.\footnote{В случае с Monero мы проверяем только $l*\tilde{K}_1 \stackrel{?}{=} 0$ для образа первичного ключа. Вспомогательные ключи будут сохраняться как $(1/8)*\tilde{K}_j$, а во время верификации умножаться на 8 (см. подпункт \ref{elliptic_curves_section}), что более эффективно. Противоречивость метода заключается в выборе варианта реализации, поскольку связываемые образы ключей очень важны и нег должны беспорядочно смешиваться, а другой метод использовался с другими версиями протокола.}

    \item Вычисляем совокупные публичные ключи $W_i$ для \(i \in \{1, 2, ..., n\}\) и совокупный образ ключа $\tilde{W}$\vspace{.175cm}
    \begin{align*}
    W_i &= \sum^{m}_{j=1} \mathcal{H}_n(T_j, \mathcal{R}, \tilde{K}_1,...,\tilde{K}_{m})*K_{i,j}\\
    \tilde{W} &= \sum^{m}_{j=1} \mathcal{H}_n(T_j, \mathcal{R}, \tilde{K}_1,...,\tilde{K}_{m})*\tilde{K}_j
    \end{align*}{}

	\item Для \(i = 1, ..., n\), заменив \(n + 1 \rightarrow 1\), вычисляем\vspace{.175cm}
	\[c_{i+1} = \mathcal{H}_n(T_c, \mathcal{R}, \mathfrak{m}, [r_i G + c_i W_i], [r_{i} \mathcal{H}_p(K_{i,1}) + c_i \tilde{W}])\]

	\item Если \(c_1 = c'_1\), то подпись является действительной.
\end{enumerate}


\subsection*{Почему это работает}

Самой большой опасностью в случае с компактными подписями, подобными этой, является аннулирование ключей, если соответствующие образы ключей признаются нелегитимными, но по-прежнему суммируются до легитимного совокупного значения. Вот где в игру вступают агрегирующие коэффициенты $\mathcal{H}_n(T_j, \mathcal{R}, \tilde{K}_1,...,\tilde{K}_{m})$, блокирующие каждый ключ в соответ\-ствии с его ожидаемым значением. Мы оставляем отслеживание циклических результатов подделки образов ключей в качестве упражнения для наших читателей (для начала, например, представьте, что эти коэффициенты просто не существуют). Образы вспомогательных ключей являются артефактом доказательства того, что образ первичного ключа является легитим\-ным, так как совокупный приватный ключ $w_{\pi}$ содержащий все приватные ключи, применяется в отношении базовой точки $\mathcal{H}_p(K_{\pi,1})$.


\subsection*{Связываемость}

Если приватный ключ \(k_{\pi, 1}\) используется повторно для создания какой-либо подписи, соответ\-ствующий образ первичного ключа \(\tilde{K}_1\), получаемый с подписью, обнаружит это. Образы вспомогательных ключей игнорируются, поскольку они используются только для облегчения «компактной» части CLSAG.


\subsection*{Требования к занимаемому месту}

Мы сохраняем (1+$n$) целых чисел, имеем $m$ образов ключей и используем $m*n$ публичных ключей.

%NOTE: in Monero CLSAG key prefixing is a bit different.
%\[c_{i+1} = \mathcal{H}_n(T_c, one-time addresses, output commitments, pseudo output commitment, \mathfrak{m}, [r_i G + c_i W_i], [r_{i} \mathcal{H}_p(K_{i,1}) + c_i \tilde{W}])\]