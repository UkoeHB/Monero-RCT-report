Criptografia. Pode parecer que somente matemáticos e cientistas da computação tenham acesso a este obscuro, esotérico, poderoso e elegante tópico. De facto, muitos tipos de criptografia são simples o suficiente, tal que qualquer leitor possa aprender esses conceitos fundamentais.
\\ \newline
É conhecimento geral que a criptografia é usada para manter comunicações seguras, sejam essas interações cartas codificadas ou interacções digitais privadas. Uma das aplicações chama-se cripto-moeda. Essa moeda digital usa criptografia para transferir ou definir posse de fundos. Para garantir que a transferencia de fundos nao seja duplicada, ou que a criação de moedas não seja arbítrária, as cripto-moedas fazem uso de assim chamadas `blockchains'. Ou seja um livro público e distribuído que mantem a contabilidade das transacções efectuadas, e que pode ser verificado por terceiros \cite{Nakamoto_bitcoin}.
\\ \newline
Á primeira vista assume-se talvez que estas transacções tenham que ser enviadas e guardadas num formato de texto simples e nao codificado tal que a verificação pública seja possível. De facto é possível não só ocultar os participantes envolvidos numa transacção como também o montante transferido. E ao mesmo tempo garantir que entidades terceiras sejam aptas de verificar e concordar nessas transaccões efectuadas \cite{cryptoNoteWhitePaper}. Isto é exemplificado na cripto-moeda Monero.
\\ \newline
O empenho aqui feito é de ensinar a qualquer leitor que saiba algebra linear básica e simples ciencia da computação não só como   Monero funciona a um nível profundo e detalhado, e também mostrar como a criptografia pode ser práctica e elegante.
\\ \newline
Para os leitores experientes: Monero é uma cripto-moeda numa blockchain; grafo standard uni-dimensional distribuído e acíclico \cite{Nakamoto_bitcoin}. Em que as transacções são baseadas em criptografia de curva elíptica. A curva usada é a Ed25519 \cite{Bernstein2008}. Entradas de transacção são assinados com Assinaturas espontâneas anónimas ligadas de grupo com múltiplas camadas (MLSAG) \cite{MRL-0005-ringct}, e montantes nas saídas são ocultados com compromissos de pedersen \cite{maxwell-ct} e prova-se que estes estão num domínio legítimo com uma prova de domínio {\em Bulletproof} \cite{Bulletproofs_paper}. Grande parte da primeira metade deste relatório esplica estas ideias.    
%For our experienced readers: Monero is a standard one-dimensional distributed acyclic graph (DAG) cryptocurrency blockchain \cite{Nakamoto_bitcoin} where transactions are based on elliptic curve cryptography using curve Ed25519 \cite{Bernstein2008}, transaction inputs are signed with Schnorr-style multilayered linkable spontaneous anonymous group signatures (MLSAG) \cite{MRL-0005-ringct}, and output amounts (communicated to recipients via ECDH \cite{Diffie-Hellman}) are concealed with Pedersen commitments \cite{maxwell-ct} and proven in a legitimate range with Bulletproofs \cite{Bulletproofs_paper}. Much of the first part of this report is spent explaining these ideas.
