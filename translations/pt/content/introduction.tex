\chapter{Introdução}
\label{chapter:introduction}

Em termos digitais é trivial fazer inúmeras cópias de informação, e também com alterações arbitrárias. Para que uma moeda exista digitalmente e para que esta seja adoptada, os seus utilizadores têm de saber que a quantidade monetária total é limitada. Um recepiente de dinheiro têm de poder verificar que não está a receber moedas contrafeitas, ou moedas que já foram enviadas a outro destinatário. Para alcancar isto, sem requerer a colaboração de alguma entidade terceira central, a quantidade monetária e a completa história de transacções tem de ser públicamente verificável. 
    
%In the digital realm it is often trivial to make endless copies of information, with equally endless alterations. For a currency to exist digitally and be widely adopted, its users must believe its supply is strictly limited. A money recipient must trust they are not receiving counterfeit coins, or coins that have already been sent to someone else. To accomplish that without requiring the collaboration of any third party like a central authority, its supply and complete transaction history must be publicly verifiable.



%We can use cryptographic tools to allow data registered in an easily accessible database - the blockchain - to be virtually immutable and unforgeable, with legitimacy that cannot be disputed by any party.

Cripto-moedas guardam as transacções na lista de blocos. Esta lista de blocos é imutável e não falsificável com uma legítimidade que não pode ser disputada. A maioria das cripto-moedas guardam as transacções em texto claro e não encriptado, para facilitar a verificação das mesmas pela comunidade de utilizadores. Claramente uma lista de blocos desta natureza desafia qualquer conceito básico de privacidade ou de fungíbilidade. Visto que as transacções de qualquer pessoa estão literalmente abertas para que o público as veja \footnote{ \textbf{Fungível} significa capaz de mútua substituição no uso ou satisfação de um contrato. Exemplos : ... Dinheiro, etc \cite{mises-org-fungible}. Numa lista de blocos aberta como em Bitcoin, as moedas possuídas pela alice podem ser diferenciadas daquelas possuídas pelo bob, com base no histórico de transacções dessas moedas. Se o histórico de transacções incluí transacções relacionadas com actores supostamente nefastos, então essas moedas podem estar manchadas \cite{bitcoin-big-bang-taint}. Como tal, tais moedas tornam-se menos valiosas do que outras, para mais diz que Bitcoins acabados de minerar (vírgens), são mais caros pois não têm histórico \cite{new-bitcoin-premium}.} .
\\
Para evitar a falta de privacidade, utilizadores de cripto-moedas como Bitcoin podem obfuscar transacções ao utilizar endereços intermediáros \cite{DBLP:journals/corr/NarayananM17}. Mesmo assim com certas ferramentas é possivel analizar e muitas vezes ligar remetentes com destinatários \cite{DBLP:journals/corr/ShenTuY15b, DK-police-tracing-btc, Andrew-Cox-Sandia, chainalysis-2020-report}.

Em contraste, a cripto-moeda Monero (Moe-neh-row), utiliza endereços ocultos para receber montantes e assinaturas em anel para enviar montantes na lista de blocos.  
Com estes methodos não existem caminhos conhecidos para desvendar a ligação entre remetentes e destinatários. Adicionalmente os montantes em sí são escondidos por construções criptográficas, o que torno o fluxo de montantes opáco.\newline O resultado é uma cripto-moeda com um elevado grau de privacidade e fungibilidade.

\footnote{Dependendo do comportamento dos utilizadores, podem haver casos em que as transacções são analisáveis até um certo grau. Para um exemplo veja-se este artigo : \cite{monero-ring-heuristics-ryo}.}
%Depending on the behavior of users, there may be cases where transactions can be analyzed to some extent. For an example see this article: \cite{monero-ring-heuristics-ryo}.}

%Additionally, transaction amounts in the Monero blockchain are concealed behind cryptographic constructions, rendering currency flows opaque.
%The result is a cryptocurrency with a high level of privacy and fungibility.



\section{Objectivos}
\label{sec:goals}
Monero é uma cripto-moeda com mais de cinco anos de desenvolvimento
%Monero is an established cryptocurrency with over five years of development 
\cite{bitmonero-launched, monero-history}, e mantêm um nível crescente de adopção
%and maintains a steadily increasing level of adoption 
\cite{justin-defcon-2019-community-growth}.
\footnote{\label{marketcap_note} Em termos de capitalização de mercado, Monero têm-se mantido firme em comparação com outras cripto-moedas. Foi número 14 em Junho de 2018, e número 12 no quinto de Janeiro de 2020; veja-se \url{https://coinmarketcap.com/}.}

%In terms of market capitalization, Monero has been steady relative to other cryptocurrencies. It was 14\nth as of June 14\nth, 2018, and 12\nth on January 5\nth, 2020; see \url{https://coinmarketcap.com/}.} 
%Unfortunately, there is little comprehensive documentation describing the mechanisms it uses.
\footnote{Um esforço de documentação, em \url{https://monerodocs.org/}, tem umas entradas de auxílio relacionadas com a interface da linha de comandos. A {\em cli} é uma carteira de monero acessível através de um terminal. É a carteira de monero com o maior número de funcionalidades, ao custo de não ser tão fácil de usar como outras carteiras que benificiam de uma superfície gráfica.}
%One documentation effort, at \url{https://monerodocs.org/}, has some helpful entries, especially related to the Command Line Interface. The CLI is a Monero wallet accessible through a console/terminal. It has the most functionality out of all Monero wallets, at the expense of no user-friendly graphical interface.}
\footnote{Uma outra documentação mais geral chamada {\em Mastering Monero} pode ser encontrada aqui : \cite{mastering-monero}.}
Ainda pior, partes esênciais da estrutura teorética de Monero têm sido publicados em artigos não revisados por profissionais na matéria. Como tal só o código fonte núcleo, é fidedigno como fonte de informação.
%Another, more general, documentation effort called Mastering Monero can be found here: \cite{mastering-monero}.} 
%Even worse, essential parts of its theoretical framework have been published in non-peer-reviewed papers that are incomplete or contain errors. For significant parts of the theoretical framework of Monero, only the source code is reliable as a source of information.
\footnote{Sr. Seguias crio a série excelente {\em Monero Building Blocks} \cite{monero-building-blocks}, que contêm as provas de segurança criptográficas que justificam os esquemas de assinaturas em Monero. Bem como zero a Monero, primeira edição \cite{ztm-1}, a série de Seguias baseia-se na versão n° 7 do protocolo.}   
%Mr. Seguias has created the excellent Monero Building Blocks series \cite{monero-building-blocks}, which contains a thorough treatment of the cryptographic security proofs used to justify Monero's signature schemes. As with Zero to Monero: First Edition \cite{ztm-1}, Seguias's series is focused on v7 of the protocol.}

Para mais, para aqueles sem experiência em matemática, aprender os básicos de criptográfia de curva elíptica, que Monero utiliza de forma extensiva, pode ser difícil e frustrante.
%Moreover, for those without a background in mathematics, learning the basics of elliptic curve cryptography, which Monero uses extensively, can be a haphazard and frustrating endeavor.
\footnote{Uma tentativa prévia de esplicar como Monero funciona \cite{MRL-0003-about-monero} não elucidou criptografia de curva elíptica, estava incompleta, e está agora obsoleta}
Tenta-se introduzir os conceitos fundamentais necessários para compreender criptografia de curva elíptica, rever algoritmos e esquemas criptográficos e coleccionar informações detalhadas sobre o funcionamento interno de Monero.
%We intend to palliate this situation by introducing the fundamental concepts necessary to understand elliptic curve cryptography, reviewing algorithms and cryptographic schemes, and collecting in-depth information about Monero’s inner workings.
Para oferecer a melhor experiência aos nossos leitores, contruiu-se uma descrição passo-a-passo da cripto-moeda Monero. Na segunda edição deste relatório a atenção foi focada na versão 12 do protocolo de Monero
%To provide the best experience for our readers, we have taken care to build a constructive, step-by-step description of the Monero cryptocurrency.
%In the second edition of this report we have centered our attention on version 12 of the Monero protocol, 
o que corresponde á versão 0.15.x.x do pacote de software de Monero. Todos os mecanismos, relacionados ás transacções e a lista de blocos, descritos aqui pertencem a estas versões.
%!!!2
\footnote{O `protocolo' é o conjunto de regras que cada novo bloco tem de seguit antes de ser adicionado á lista de blocos. Este conjunto de regras incluí `o protocolo de transacção' (actualmente na versão 2, RingCT), que são regras gerais que definem como uma transacção é construída. Regras específicas para as transacções podem mudar e mudam de facto. Sem que a versão do protocolo de transacção mude. Só mudanças em grande escala da estrutura de transacção permitem mover o número da versão.}  
%The `protocol' is the set of rules that each new block is tested against before it can be added to the blockchain. This set of rules includes the `transaction protocol' (currently version 2, RingCT), which are general rules pertaining to how a transaction is constructed. Specific transaction rules can, and do, change, without the transaction protocol's version changing. Only large-scale changes to the transaction structure warrant moving its version number.}, corresponding to version 0.15.x.x of the Monero software suite. All transaction and blockchain-related mechanisms described here belong to those versions.
\footnote{A integridade do código fonte em Monero parte do princípio de que bastantes pessoas o leram e que todos os erros, ou pelo menos os mais relevantes foram removidos.
Espera-se que os nossos leitores não tomem as nossas esplicações como garantidas, e que verifiquem por eles próprios que o código fonte faz, o que deve fazer. Se não, espera-se que o leitor faça uma divulgação responsável. Em que se usa (\url{https://hackerone.com/monero}) para os maiores problemas, e um {\em pull request} no github em (\url{https://github.com/monero-project/monero}) para detalhes.}    
%The Monero codebase's integrity and reliability is predicated on assuming enough people have reviewed it to catch most or all significant errors. We hope that readers will not take our explanations for granted, and verify for themselves the code does what it's supposed to. If it doesn't, we hope you will make a responsible disclosure (\url{https://hackerone.com/monero}) for major problems, or Github pull request (\url{https://github.com/monero-project/monero}) for minor issues.}
\footnote{Diversos protocolos que valem a pena serem considerados para a próxima geração das transacções em Monero estão a ser investigadas e pesquisadas. Estes incluem Triptych \cite{triptych-preprint}, RingCT3.0 \cite{ringct3-preprint}, Omniring \cite{omniring-paper}, e Lelantus \cite{lelantus-preprint}.}  
%Several protocols worth considering for the next generation of Monero transactions are undergoing research and investigation, including Triptych \cite{triptych-preprint}, RingCT3.0 \cite{ringct3-preprint}, Omniring \cite{omniring-paper}, and Lelantus \cite{lelantus-preprint}.} 
Esquemas de transacções descontinuados não foram explorados, mesmo que estes ainda sejam parcialmente supportados por causa da compatibilidade com versões anteriores. O mesmo para características descontinuadas da lista de blocos. A primeira edição \cite{ztm-1} corresponde á versão 7 do protocolo, e á versão 0.12.x.x do pacote de software de Monero.
  
%Deprecated transaction schemes have not been explored to any extent, even if they may be partially supported for backward compatibility. Likewise with deprecated blockchain features. The first edition \cite{ztm-1} corresponded to version 7 of the protocol, and version 0.12.x.x of the software suite.
\section{Aos leitores}

Antecipa-se que muitos leitores irão encontrar este relatório com nenhum a pouco conhecimento de matemática discreta, estruturas algebraícas, criptografia e listas de blocos. Tentou-se ser o mais detalhado possível para que mesmo leigos de todas as perspectivas possam aprender Monero sem precisar de pesquisa externa.   
%We anticipate many readers will encounter this report with little to no understanding of discrete mathematics, algebraic structures, cryptography
\footnote{Um livro extensivo em criptografia aplicada pode ser encontrado aqui :
\cite{applied-cryptography-textbook}.}
%, and blockchains. We have tried to be thorough enough that laypeople from all perspectives may learn Monero without needing external research.

Omitiu-se intencionalmente, ou foi delegado para notas de rodapé, alguns detalhes matemáticos, que tornariam o texto menos claro. Foram também omitidos detalhes concretos de implementação quando estão não pareciam ser essenciais. O objectivo aqui é de apresentar a matéria a meio caminho entre a criptografia e a programação, de forma completa e clara.  
%We have purposefully omitted, or delegated to footnotes, some mathematical technicalities, when they would be in the way of clarity. We have also omitted concrete implementation details where we thought they were not essential. 
%Our objective has been to present the subject half-way between mathematical cryptography and computer programming, aiming at completeness and conceptual clarity.
\footnote{Algumas notas de rodapé, especialmente nos capítulos relacionados com o protocolo, estragam os capítulos ou secções futuras. A intenção destas, é de fazerem mais sentido numa segunda leitura, pois estas involvem detalhes de implementação que usualmente só fazem sentido para aqueles que já compreendem como Monero funciona.}   
%Some footnotes, especially in chapters related to the protocol, spoil future chapters or sections. These are intended to make more sense on a second read-through, since they usually involve specific implementation details that are only useful to those who have a grasp of how Monero works.}

\section{Origens da cripto-moeda Monero}

A cripto-moeda Monero, initialmente conhecida como BitMonero, foi criada em abril de 2014 e deriva da prova de conceito de {\em CryptoNote} \cite{bitmonero-launched}.
%The cryptocurrency Monero, initially known as BitMonero, was created in April 2014 as a derivative of the proof-of-concept currency CryptoNote \cite{bitmonero-launched}. 
Monero significa {\em moeda} na língua Esperanto, e o plural é Moneroj.
%Monero means `money' in the language Esperanto, and its plural form is Moneroj (Moe-neh-rowje, similar to Moneros but using the -ge from orange).
A cripto-moeda CryptoNote foi desenvolvida por vários indivíduos. O primeiro abstracto académico que a descreve foi públicado debaixo do pseudónimo de {\em Nicolas van Saberhagen} em Outubro de 2013 \cite{cryptoNoteWhitePaper}. Nessa altura a anônimidade do destinatário foi garantida com endereços ocultos, e a ambuiguidade do remetente com assinaturas em anel. 
%!!!2  
%CryptoNote is a cryptocurrency devised by various individuals. A landmark whitepaper describing it was published under the pseudonym of Nicolas van Saberhagen in October 2013 \cite{cryptoNoteWhitePaper}. 
%It offered receiver anonymity through the use of one-time addresses, and sender ambiguity by means of ring signatures.
Desde a sua incepção, Monero fortaleceu a sua privacidade ainda mais, ao implementar montantes ocultos, como descrito por Greg Maxwell (entre outros) em \cite{Signatures2015BorromeanRS} e integrado em assinaturas de anel baseado em recomendações por Shen Noether \cite{MRL-0005-ringct}. O que se tornou depois ainda mais eficiente com {\em Bulletproofs} \cite{Bulletproofs_paper}.

%Since its inception, Monero has further strengthened its privacy aspects by implementing amount hiding, as described by Greg Maxwell (among others) in \cite{Signatures2015BorromeanRS} and integrated into ring signatures based on Shen Noether's recommendations in \cite{MRL-0005-ringct}, then made more efficient with Bulletproofs \cite{Bulletproofs_paper}.

%\section{Outline}

%As mentioned, our aim is to deliver a self-contained and step-by-step description of the Monero cryptocurrency. Zero to Monero has been structured to fulfill this objective, leading the reader through all parts of the currency’s inner workings.
\subsection{Parte 1: `Essenciais'}

Nesta saga de conhecimento, são apresentados os elementos básicos de criptografia necessários para perceber as complexidades de Monero.\newline No capítulo \ref{chapter:basicConcepts} são desenvolvidos os aspectos essentiais de criptografia de curva elíptica.  
\\
No capítulo \ref{chapter:advanced-schnorr} expande-se o esquema de assinaturas tipo Schnorr do capítulo anterior. E esplica-se os algoritmos de assinaturas em anel que são aplicados para alcançar transacções confidenciais.

No capítulo \ref{chapter:addresses} esplora-se como Monero utiliza endereços para controlar a posse de montantes, e os diferentes tipos de endereços.

No capítulo \ref{chapter:pedersen-commitments} introduzimos os mecanismos criptográficos usados para ocultar montantes. 

Com todos os componentes dados, esplica-se o esquema de transacções usado em Monero no capítulo \ref{chapter:transactions}.

A lista de blocos de Monero é apresentada no capítulo \ref{chapter:blockchain}.

%The Monero blockchain is unfolded in Chapter \ref{chapter:blockchain}.

\subsection{Part 2: `Extenções'}

Uma cripto-moeda é mais do que o próprio protocolo, e em `extensões' fala-se sobre um número de ideias diferentes, muitas das quais ainda não foram implementadas. 
%A cryptocurrency is more than just its protocol, and in `Extensions' we talk about a number of different ideas, many of which have not been implemented.
\footnote{Note-se que as versões futuras do protocolo em Monero, especialmente aquelas que mudam a estrutura das transacções, podem fazer estas ideias impossíveis ou impraticáveis.}
%Please note that future protocol versions of Monero, especially those implementing new transaction protocols, may make any or all of these ideas impossible or impractical.}
Várias informações sobre uma transacção pode ser provadas a observadores, e esses methodos são o conteúdo do capítulo \ref{chapter:tx-knowledge-proofs}.
%Various information about a transaction can be proven to observers, and those methods are the content of Chapter \ref{chapter:tx-knowledge-proofs}.
Enquanto não essencial para a operação de Monero, existe muita utilidade em multi-assinaturas o que permite múltiplas pessoas enviar e receber fundos colaborativamente.
O capítulo \ref{chapter:multisignatures} descreve o esquema actual de multi-assinaturas de Monero, e também os seus futuros possíveis desenvolvimentos.
%While not essential to the operation of Monero, there is a lot of utility in multisignatures that allow multiple people to send and receive money collaboratively. 
%Chapter \ref{chapter:multisignatures} describes Monero's current multisignature approach and outlines possible future developments in that area.%This is formally called (N-1)-of-N and N-of-N threshold authentication.
O capítulo \ref{chapter:escrowed-market} constitui a architectura de mercados de garantia online com multi-assinaturas.  
%Of extreme importance is applying multisig to the interactions of vendors and shoppers in online marketplaces. Chapter \ref{chapter:escrowed-market} constitutes our original design of an escrowed marketplace using Monero multisig.
Primeiro apresentado aqui, TxTangle, descrito no capítulo \ref{chapter:txtangle}, é um protocolo decentralizado para reunir transacções de múltiplos individuos para uma só transacção.

%First presented here, TxTangle, outlined in Chapter \ref{chapter:txtangle}, is a decentralized protocol for joining the transactions of multiple individuals into one.


\subsection{Conteúdo adicional}

O apéndice \ref{appendix:RCTTypeBulletproof2} explica a estrutura de uma transacção exemplo da lista de blocos. O apéndice \ref{appendix:block-content} esplica a estrutura de um bloco (incluindo o cabeçalho e as transacções de mineiro). Finalmente o apéndice \ref{appendix:genesis-block}, fecha este relatório ao esplicar a estrutura do bloco génese de Monero. Estes apéndices oferecem uma ligação entre os elementos teóricos descritos nas secções anteriores com a sua implementação na vida real.
    
%explains the structure of a sample transaction from the blockchain. Appendix \ref{appendix:block-content} explains the structure of blocks (including block headers and miner transactions) in Monero's blockchain. Finally, Appendix \ref{appendix:genesis-block} brings our report to a close by explaining the structure of Monero's genesis block. These provide a connection between the theoretical elements described in earlier sections with their real-life implementation.

São usadas notas de margem \marginnote{isto é útil !}, para indicar onde é que detalhes de implementação existem dentro do código fonte. 
%We\marginnote{Isn't this useful?} use margin notes to indicate where Monero implementation details can be found in the source code.
\footnote{As notas de margem são precisas para a versão 0.15.x.x do pacote de software de Monero. mas podem se tornar gradualmente imprecisos á medida que o código base muda. O código fonte está guardado num repositório de {\em git} (\url{https://github.com/monero-project/monero}), portanto uma história completa de alterações está disponível}
%Our margin notes are accurate for version 0.15.x.x of the Monero software suite, but may gradually become inaccurate as the codebase is constantly changing. However, the code is stored in a git repository (\url{https://github.com/monero-project/monero}), so a complete history of changes is available.} 
Usualmente existe um directório, como por exemplo : src/ringct/rctOps.cpp . E também uma função como por exemplo : \(\textrm{{\tt ecdhEncode()}}\). Note-se : `-' indica texto dividido, como por exemplo : crypto- note $\rightarrow$ cryptonote.\newline Normalmente os qualificadores de {\em namespace} não são incluídos (e.g. {\tt Blockchain::}).

%There is usually a file path, such as src/ringct/rctOps.cpp, and a function, such as \(\textrm{{\tt ecdhEncode()}}\). Note: `-' indicates split text, such as crypto- note $\rightarrow$ cryptonote, and we neglect namespace qualifiers (e.g. {\tt Blockchain::}) in most cases.



\section{Aviso}
Todos os esquemas de assinatura, aplicações de curva elíptica, e detalhes de  implementação devem ser considerados somente descriptivos. Leitores que consideram aplicações sérias, ao invés de explorações para extender o conhecimento, devem consultar fontes primárias e especificações técnicas (que são citadas sempre que possível).
%All signature schemes, applications of elliptic curves, and Monero implementation details should be considered descriptive only. Readers considering serious practical applications (as opposed to a hobbyist's explorations) should consult primary sources and technical specifications (which we have cited where possible). 
Esquemas de assinatura precisam provas com uma segurança bem definida, e detalhes de implementação podem ser encontrados no código fonte de Monero.\newline Em particular, como se diz em inglés : `don't roll your own crypto' . O que significa : `não divulgues a tua própria cripto-moeda.' Código que implemente primitivas criptográficas deve ser revisto por especialistas, e deve ter um longo histórico de desempenho confiável.\newline Finalmente contribuições originais para este documento podem não ser bem revistas e provávelmente não são testadas, tanto que, leitores devem exercitar cautela.    
%Signature schemes need well-vetted security proofs, and Monero implementation details can be found in Monero's source code. In particular, as a common saying goes, `don't roll your own crypto'. Code implementing cryptographic primitives should be well-reviewed by experts and have a long history of dependable performance. Moreover, original contributions in this document may not be well-reviewed and are likely untested, so readers should exercise their judgement when reading them.
\section{A História de Zero a Monero}

De zero a Monero é uma expansão da tese de mestrado de Kurt Alonso `Monero - Privacy in the Blockchain' \cite{kurt-original}, publicado em maio de 2018. A primeira edição foi publicada em junho de 2018 \cite{ztm-1}.
%Zero to Monero is an expansion of Kurt Alonso's master's thesis, `Monero - Privacy in the Blockchain' \cite{kurt-original}, published in May 2018. The first edition was published in June 2018 \cite{ztm-1}.
Na segunda edição, melhorou-se a forma de apresentar assinaturas em anel (Chapter \ref{chapter:advanced-schnorr}), reorganizou-se como as transacções são esplicadas
(adicionou-se o capítulo \ref{chapter:addresses} em endereços Monero). Modernizou-se o méthodo usado para comunicar montantes de saídas de transacção (secção  \ref{sec:pedersen_monero}). Assinaturas em anel {\em Borromean} são substituídas com {\em Bulletproofs} . Descontinou-se {\tt RCTTypeFull} (capítulo \ref{chapter:transactions}). Renovou-se o peso dinámico de bloco e o sistema de taxa dos mineiros (capítulo \ref{chapter:blockchain}).
%For the second edition we have improved how ring signatures are introduced (Chapter \ref{chapter:advanced-schnorr}), reorganized how transactions are explained (added Chapter \ref{chapter:addresses} on Monero Addresses), 
%modernized the method used to communicate output amounts (Section \ref{sec:pedersen_monero}), replaced Borromean ring signatures with Bulletproofs (Section \ref{sec:range_proofs}), deprecated {\tt RCTTypeFull} (Chapter \ref{chapter:transactions}), updated and elaborated Monero's dynamic block weight and fee system (Chapter \ref{chapter:blockchain}), 
Investigaram-se provas relacionadas com transacções (capítulo \ref{chapter:tx-knowledge-proofs}). Descreveram-se multi-assinaturas de Monero (capítulo \ref{chapter:multisignatures}). Desenvolveram-se soluções para mercados com garantia (capítulo \ref{chapter:escrowed-market}). Foi proposto um novo protocolo de transacções chamado TxTangle (capítulo \ref{chapter:txtangle}). Adicionaram-se vários detalhes para alignar com o protocolo actual (v12) e o pacote de software Monero (v0.15.x.x). E poliu-se o documento inteiro para aumentar a qualidade de leitura.  
%investigated transaction-related proofs (Chapter \ref{chapter:tx-knowledge-proofs}), described Monero multisignatures (Chapter \ref{chapter:multisignatures}), 
%designed solutions for escrowed marketplaces (Chapter \ref{chapter:escrowed-market}), proposed a new decentralized joint transaction protocol named TxTangle (Chapter \ref{chapter:txtangle}), updated or added various minor details to align with the most current protocol (v12) and Monero software suite (v0.15.x.x), and scoured the document for quality-of-reading edits.
\footnote{O código fonte \LaTeX{} de ambas as edições de zero a Monero pode ser encontrado aqui (a primeira edição está no ramo `ztm1'):
\url{https://github.com/UkoeHB/Monero-RCT-report}.}



\section{Reconhecimentos}
\label{sec:acknowledgements}

Escrito pelo autor `koe'.
Este relatório não existiria sem a these de mestrado de Kurt Alonso \cite{kurt-original}, portanto a ele tenho uma enorme dívida de gratidão. O pesquisador Brandon ``Surae Noether" Goodell e o pseudónimo `Sarang Noether' do laboratório de pesquisas de Monero (MRL), têm sido fontes de conhecimento confiáveis ao longo do  desenvolvimento de ambas as edições de zero a Monero.\newline O pseudónimo `moneromooo', o programador mais prolífico do código fonte do projecto de Monero, tem provavelmente o conhecimento mais extensivo do código neste planeta, e deu-me númeras vezes as direcções indicadas. E claro outros contribuidores magníficos de Monero que gastaram tempo a esplicarem-me as minhas inúmeras questões. E finalmente os vários leitores que nos contactaram com comentários encoragedores, obrigado !         

