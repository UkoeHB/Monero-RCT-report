\chapter{Transacções juntas ({\tt TxTangle})}
\label{chapter:txtangle}

Existem algumas heurísticas para construir grafos de transacções, que são inevitáveis e que resultam da natureza de certos scenários e entidades. Em particular, o comportamento de mineiros, piscinas (do inglés : pools; secção 5.1 de \cite{AnalysisOfLinkability}),
mercados online e bolsas de câmbio apresentam padrões abertos a análise apesar do protocolo privado de Monero.

%There are a number of unavoidable transaction graph heuristics created by the nature of different entities and scenarios. In particular, the behavior of miners, pools (Section 5.1 of \cite{AnalysisOfLinkability}), escrowed marketplaces, and exchanges have clear patterns open to analysis even within Monero's ring signature-based protocol.

Descreve-se aqui {\em TxTangle}, análogo ao {\em CoinJoin} de Bitcoin um méthodo para confundir essas heuristicas. 

%We describe here TxTangle, analogous to Bitcoin's CoinJoin \cite{coinjoin-wiki}, one method to confuse those heuristics.
\footnote{Este capítulo constitui a proposta para um protocolo de transacções juntas. Isto até á data ainda não foi implementado. Uma proposta prévia, chamada {\em MoJoin} e criada pelo co-autor, investigador do laboratório de pesquisas de Monero (MRL), de pseudónimo Sarang Noether, requeria uma entidade terçeira de confiança para funcionar. Como tal {\em MoJoin}, não foi mais desenvolvido, e não está implementado.}     
%This chapter constitutes the proposal for a joint transaction protocol. No such protocol has been implemented as of this writing. A previous proposal, named MoJoin and created by pseudonymous co-author Monero Research Lab (MRL) researcher Sarang Noether, required a trusted dealer to function. Such a dealer seems to conflict with the Monero project's basic commitment to privacy and fungibility, and hence MoJoin was not pursued further.} 

Em essência, várias transacções são juntas para formar uma transacção, tal que os padrões de cada participante, derretam juntamente.

%In essence, several transactions are squashed into one transaction, making the behavior patterns of each participant blend together. 

Para alcançar tal obfuscação, tem de ser praticamente impossivel para que observadores usem essa informação contida numa transacção junta para associar entradas e saídas a participantes individuais ou até de saber quantos participantes estão envolvidos. 

%To accomplish that obfuscation, it must be unreasonably difficult for observers to use the information contained in a joint transaction to group inputs and outputs, and associate them with individual participants, nor know how many participants there actually were.
\footnote{Como em Bitcoin os montantes são claramente visíveis, é muitas vezes possível agrupar saídas e entradas de transacções com base na soma dos montantes \cite{coinjoin-sudoku}.}
%Since in Bitcoin amounts are clearly visible, it is often possible to group CoinJoin inputs and outputs based on amount sums. \cite{coinjoin-sudoku}} 

Para além disso, mesmo os próprios participantes não deviam estar a par desse número. Nem de associar entradas e saídas a outros participantes ( excepto em certos cenários ).
%Moreover, even the participants themselves should not be aware of the number of participants, or be able to group the inputs and outputs of other participants unless they control all but one participant's grouping.
\footnote{Poluír transacções juntas de forma maliciosa é um potencial ataque a este método, e que foi primeiro identificado para CoinJoin. \cite{coinjoin-pollution}} 
%Maliciously polluting joint transactions is a potential attack on this method, first identified for CoinJoin. \cite{coinjoin-pollution}} 
Finalmente devia ser possível de construír transacções juntas sem ter de depender numa autoridade central \cite{exa-blockchain-analysis}. Felizmente todos esses requisitos são alcancados em Monero.
%Finally, it should be possible to construct joint transactions without relying on a central authority \cite{exa-blockchain-analysis}. Fortunately, all these requirements can be achieved in Monero.

\section{Construir transacções juntas}
\label{sec:building-txtangle}

Numa transacção normal, entradas e saídas são construídas usando a prova de que os montantes se igualam. Da secção \ref{sec:commitments-and-fees}, a soma de pseudo compromissos de saída é igual á soma de compromissos de saída ( mais o compromisso da taxa).
%In a normal transaction, inputs and outputs are tied together using the proof that amounts balance. From Section \ref{sec:commitments-and-fees}, the sum of pseudo output commitments equals the sum of output commitments (plus the fee commitment).
\begin{align*}
\sum_j C'^a_{j} - (\sum_t C^b_{t} + f H) = 0
\end{align*}

Uma transacção junta trivial podia pegar todo o conteúdo de múltiplas transacções, e juntá-las. As mensagens MLSAG assinariam todos os dados das sub-transacções, e os montantes iriam-se igualar (0 + 0 + 0 = 0).    

%A trivial joint transaction could take all the content of multiple transactions, and stick them in one. MLSAG messages would sign all the sub-transactions' data, and amount balancing would quite obviously work (0 + 0 + 0 = 0).
\footnote{Desde que provas de domínio do tipo {\em Bulletproof} são agregadas para uma só (secção \ref{sec:range_proofs}), os participantes teriam de colaborar entre sí até um certo grau, mesmo no caso trivial.}
%Since Bulletproofs-style range proofs are actually aggregated into one (Section \ref{sec:range_proofs}), participants would have to collaborate to some extent even in the trivial case.} 
Porém, grupos de entradas e saídas seriam identificados trivialmente, porque seria possível testar se conjuntos de entradas e saídas têm montantes que se igualam. 
%However, input and output groupings could just as trivially be identified based on testing if input/output subsets have balancing amounts.
\footnote{Os participantes poderiam tentar dividir a taxa de uma forma esperta para confundir os observadores, mas isto iria falhar á face da {\em força bruta}, pois as taxas não são assim tão grandes (por volta de 32 bits ou menos).} 
%Participants could try to divide up the fee in a fancy manner to confuse observers, but it would fail in the face of brute force since fees are not that large (around 32 bits or less).}

É possivel dar a volta a isto ao calcular segredos partilhados entre pares de participantes, ao adicionar estes offsets ás máscaras de pseudo compromissos de saída
(Section \ref{sec:ringct-introduction}). Em cada par, um participante adiciona o segredo partilhado a um dos seus pseudo compromissos de saída, e um outro participante subtrai isso de um dos {\em seus} pseudo compromissos. Na sua soma, os segredos partilhados cancelam-se mutuamente.
\newline E desde que cada par de participantes tem um segredo partilhado, o montante só aparece depois de todos os compromissos serem combinados.    
\footnote{O {\em offset} é feito dos pseudo compromissos de saída, em vez dos compromissos de saída, pois as máscaras dos compromissos de saída são construídas a partir do endereço do destinatário (secção \ref{sec:pedersen_monero}).}
%We offset the pseudo output commitments instead of output commitments since output commitment masks are constructed from the recipient's address (Section \ref{sec:pedersen_monero}).}

%We can easily get around this by computing shared secrets between each pair of participants, then adding these offsets to their pseudo output commitments' masks (Section \ref{sec:ringct-introduction}). In each pair, one participant adds the shared secret to one of his pseudo output commitments, and the other participant subtracts it from one of {\em his} pseudo commitments. When summed together, the secrets cancel out, and since each pair of participants has a shared secret the amount balance only appears after all commitments are combined.

%Shared secrets may hide input/output groupings in the immediate sense, but participants must learn about all the inputs and outputs somehow, and the easiest way is if they each communicate their individual input/output groupings. Clearly this violates the initial premise, and in any case implies participants know the number of participants.
%??? so?

\subsection{Canal de comunicação em grupo}
\label{subsec:n-way-channel}

O número máximo de participantes para um {\em TxTangle} é o número de entradas ou de saídas, dependendo de qual é menor. Isto é modelado através de cada participante real pretender ser uma pessoa differente para cada saída que ele envia. Isto é feito para estabelecer um canal de comunicação em grupo com outros potenciais participantes, sem que seja revelado quantos participantes existem. 

%The maximum number of participants to a TxTangle is either the number of outputs or inputs (whichever is lower). We model that by each real participant pretending to be a different person for each output he is sending. This is primarily for the purpose of setting up a group communication channel with other would-be participants, without revealing how many participants there are.

Imagine-se $n$ ($2 \leq n \leq 16$, apesar de que pelo menos 3 é recomendado)
pessoas supostamente não relacionadas juntam-se a um {\em chatroom}. Agendado para abrir no tempo $t_0$ e fechar no tempo $t_1$. Só 16 pessoas podem estar no {\em chatroom} ao mesmo tempo. E existem condições como prioridades de taxa, taxa base por cada byte e tipos de transacção aceites. Ao tempo $t_1$ todos os membros signalam que querem proceder ao publicarem uma chave pública, e o {\em chatroom} é convertido para um canal de comunicação de grupo ao construir um segredo partilhado entre todos os membros desvio.   
%Imagine $n$ ($2 \leq n \leq 16$, although at least 3 is recommended)
\footnote{Actualmente uma transacção pode ter no máximo 16 saídas.}
%Currently, a transaction may have at most 16 outputs.} 
%supposedly unrelated people gather at apparently random intervals in a chatroom, scheduled to open at time $t_0$ and close at $t_1$ (only 16 people can be in a chatroom at once, and the chatroom has conditions like fee priority, base fee per byte [to simplify consensus around the current median and block reward], and range of acceptable transaction types since e.g. currently transactions from the beginning of Monero can't be directly spent in a RingCT transaction \cite{pre-ringct-outputs-like-coinbase-research-issue-59}). At $t_1$ all mock-members signal desire to proceed by publishing a public key, and the room is converted into an $n$-way communication channel by constructing a shared secret amongst all the mock-members.
\footnote{O método de multi-assinatura da secção \ref{sec:m-of-n} é uma forma, que extende M-de-N até 1-de-N.}
%The multisig method from Section \ref{sec:m-of-n} is one way, extending M-of-N all the way to 1-of-N.} 
Este segredo partilhado é usado para encriptar conteúdos de mensagens, enquanto que membros desvio assinam mensagens relacionadas a entradas com assinaturas SAG (secção \ref{SAG_section}). Portanto não é claro quem é que enviou uma dada mensagem.
\newline As mensagens relacionadas com as saídas usam uma assinatura tipo bLSAG (secção \ref{blsag_note}) no conjunto das chaves públicas dos membros desvio tal que as saídas não estão associadas aos membros desvio.    
%This shared secret is used to encrypt message contents, while mock-members sign input-related messages using SAG signatures (Section \ref{SAG_section}) so it's never clear who sent a given message, and output-related messages with a bLSAG (Section \ref{blsag_note}) on the set of mock-member public keys so actual outputs are dissociated from mock-members.
\footnote{Cada conjunto separado de assinaturas TxTangle bLSAG, devia usar as mesmas imagens de chave, desde que tudo relacionado com uma dada saída está ligado.}
%Each separate set of TxTangle bLSAGs should use the same key images, since everything related to a given output is linked together.}% can create their key images using a different hash-to-point algorithm, or more simply by tagging the hash with a string, e.g. $\tilde{K}_t = \mathcal{H}_p(``TxTangle\_bLSAG\_2",K_t)$.}


\subsection{Rondas de mensagens para construir uma transacção junta}
\label{subsec:message-rounds-txtangle}

Depois do canal estar pronto, transacções de {\em TxTangle} podem ser construídas em cinco rondas de comunicação, em que a próxima ronda só pode começar se a ronda prévia foi finalizada. Em cada ronda as mensagens são publicadas dentro de um certo intervalo de tempo limite, e o tempo de publicação de cada uma é aleatório. Isto serve para prevenir que grupos de mensagens revelem conjuntos de saída/entrada. 

%After the channel is set up, TxTangle transactions can be constructed in five rounds of communication, where the next round may only begin after the previous is finished, and each round is given a timed communication interval within which messages should be randomly published. These intervals are intended to prevent message clusters that would reveal input/output groupings.
\begin{enumerate}
    \item Cada membro desvio gera privadamente um escalar aleatório para cada saída, e assina esta com uma assinatura tipo bLSAG. Uma lista ordenada destes escalares serve para determinar indexes de saída. O menor escalar recebe o index $t = 0$ ( secção \ref{sec:multi_out_transactions}).   

%Each mock-member privately generates a random scalar for each intended output, and signs them with bLSAGs. A sorted list of these scalars is used to determine output indices 
%(recall Section \ref{sec:multi_out_transactions}; the smallest scalar gets index $t = 0$).
\footnote{A selecção do index de saída devia ser igual a outras implementações que constroiem transacções, para evitar que software diferente seja identificado. Utiliza-se efectivamente uma abordagem aleatória para alinhar com a implementação núcleo, que também mistura as saídas de forma aleatória.}  
%Output index selection should match other implementations of transaction construction to avoid fingerprinting different software. We use this effectively random approach to align with the core implementation, which also randomizes outputs.} 
Essas assinaturas são publicadas, e também assinaturas SAG que assinam o número da versão de entradas de transacção. Depois desta ronda os participantes podem calcular o peso de transacção baseado no número de entradas e saídas, e estimar precisamente a taxa necessária.
%They publish those bLSAGs, and also SAGs which sign the transaction version numbers of intended inputs. After this round participants can calculate the transaction weight based on number of inputs and outputs, and accurately estimate the fee required.
\footnote{Se acontece que só podem haver dois participantes reais, baseado na comparação do número de entradas e saídas total, com o número de entradas/saídas do próprio participante, o TxTangle pode ser abandonado. Recomenda-se que cada participante tenha pelo menos duas entradas e duas saídas, no caso de actores maliciosos que não abandonam TxTangles mesmo quando reconhecem que só existem dois participantes. Esta recomendação está aberta ao debate, pois usar mais entradas e saídas não é heuristicamente neutral.}
%If it turns out there must be only two real participants, based on comparing the number of inputs and outputs to one's own input/output count, the TxTangle can be abandoned. It's recommended for each participant to have at least two inputs and two outputs, in case of malicious actors who don't abandon TxTangles even when they realize it's just two participants. This recommendation is open to debate, since using more inputs and outputs is not heuristically neutral.}
\footnote{Transacções de TxTangle não deviam ter informação adicional no campo {\em extra} (e.g. nenhum ID de pagamento, a não ser que seja um TxTangle com só duas saídas, que deve ter pelo menos um ID de pagamento desvio encriptado) }
%TxTangle transactions should not have extraneous information stored in the extra field (e.g. no encrypted payment ID unless it's only a 2-output TxTangle which should have at least a dummy encrypted payment ID).}
\footnote{A estimação da taxa devia ser baseada numa abordagem standard, assim cada participante calcula a mesma coisa. De outra forma as saídas podem ser agrupadas com base no método do cálculo da taxa. Este standard da taxa devia ser implementado fora de TxTangle, para promover que transacções de TxTangle se parecam iguais a transacções normais.} 
%Fee estimation should be based on a standardized approach, so each participant calculates the same thing. Otherwise outputs may be clustered based on fee calculation method. This same fee standard should be implemented outside of TxTangle, to promote TxTangle transactions looking the same as normal transactions.}
    \item Cada membro desvio utiliza a lista de chaves públicas para construir o segredo partilhado com cada outro membro. Isto é feito para fazer um offset dos pseudo compromissos de saída. É decidido quem adiciona ou subtrai baseado na chave pública menor entre cada par.
%Each mock-member uses the list of public keys to construct a shared secret with each other member for offsetting their pseudo output commitments, and decides who will add or subtract based on each pair's smaller public key.
\newline Cada membro desvio tem de pagar 1/$n$ da taxa estimada ( usando divisão inteira ). O membro desvio com o index de saída menor, tem a responsabilidade de pagar o resto da divisão inteira ( o que deve ser um montante infinitésimal, mas tem de ser posto em conta para evitar que a transacção seja reconhecida como junta ).
%Each mock-member must pay for 1/$n$\nth of the fee estimate (using integer division). The mock-member with lowest output index is given the responsibility of paying for the remainder after dividing (it will be a truly infinitesimal amount, but must be taken into account to prevent fingerprinting TxTangle transactions). 
\newline Eles geram privadamente chaves públicas de transacção para cada uma das suas saídas (por enquanto não se envia isto a outros membros), e constroíem os compromissos de saída, montantes codificados, a parte A de provas parciais para serem utilizadas para a prova agregada de domínio {\em Bulletproof}, e assinar isto tudo com bLSAGs (um compromisso, um montante encriptado, uma prova parcial por mensagem bLSAG, e a imagem de chave liga estes dados á lista original de escalares aleatórios que foi utilizado para especificar indíces de saída). Pseudo compromissos de saída são gerados normalmente (secção \ref{sec:ringct-introduction}), depois o offset com o segredo partilhado, e assinado com SAGs. Depois de bLSAGs e SAGs serem publicados, e assumindo que os participantes estimaram o total em taxas da mesma forma, eles podem agora verificar que os montantes em geral se igualam.           
%They privately generate transaction public keys for each of their outputs (not to be sent to other members just yet), and construct their output commitments, encoded amounts, and Part A partial proofs to be used for the aggregate Bulletproof range proof, signing all of it with bLSAGs (one commitment, one encoded amount, and one partial proof per bLSAG message, and the key image links these to the original list of random scalars that was used to specify output indices). 
%Pseudo output commitments are generated as normal (Section \ref{sec:ringct-introduction}), then offset with the shared secrets, and signed with SAGs. After the bLSAGs and SAGs are published, and assuming participants estimated the total fee in the same way, they may now verify that amounts balance overall.

\footnote{Visto que os pontos de CE são comprimidos (secção \ref{point_compression_section}), as chaves são interpretadas como inteiros de 32 bytes. É convenção que o dono da chave menor adiciona, e o dono da chave maior subtrai.}
%Since points are compressed (Section \ref{point_compression_section}), just interpret the keys as 32-byte integers. The smaller key's owner adds, and larger key's owner subtracts, by convention.} 
\footnote{Não se publicam os montantes das taxas separadas pagas, no caso em que um participante se enganou no cálculo da taxa. O que pode revelar um grupo de saídas, devido aos montantes estranhos que sobressaem. Se os montantes não...?}
%??? Se os montantes não...? quais o que?  
%We don't publish the separate fee amounts paid for in case a participant calculated it wrong, which may reveal an output cluster due to a collection of non-standard fee amounts. If amounts don't balance properly, the TxTangle transaction may be abandoned.}
    \item Se os montantes se igualam própriamente, começa uma ronda para construir as provas de domínio agregadas, o que prova que todas os montantes de saída estão no domínio.\newline Cada membro desvio usa as provas parciais e os compromissos de saída da ronda parte A, e calcula privadamente o desafio agregado A. Depois constrõem a prova parcial parte B que depois é enviada para o canal usando uma assinatura tipo bLSAG.      

%If amounts balance properly we begin a small additional round for building the aggregate Bulletproof that proves all output amounts are in range. Each mock-member uses the previous round's Part A partial proofs and output commitments, and privately computes the aggregate challenge A. They use it to construct their Part B partial proof, which they send to the channel with a bLSAG.
    \item Os participantes começam a preencher a mensagem para ser assinada com assinaturas do tipo MLSAG. Dois tipos de mensagens são publicados em ordem aleatória dentro do intervalo de comunicação.\newline Cada offset de entrada que pertence a um anel, e as imagens de chave são assinadas com uma assinatura tipo SAG, e associadas com o pseudo compromisso de saída correcto.\newline Cada endereço oculto de uma saída, chave pública de transacção, e a prova parcial parte C são assinados com uma assinatura do tipo bLSAG. A prova parcial parte C é calculada com base nas provas parciais da parte B, e com o desafio agregado B. Pode também existir uma componente aleatória que é uma chave pública de transacção, esta serve para mitigar o "Janus falso".        

%Participants begin filling in the message to be signed by MLSAGs (recall the footnote in Section \ref{full-signature}). Published in random order over the communication interval are two kinds of messages. Each input's ring member offsets and key images are signed with a SAG and associated with the correct pseudo output commitment. Each output's one-time address, transaction public key, and Part C partial proof (computed based on the Part B partial proofs and an aggregate challenge B) are signed with a bLSAG (these can also include a random base transaction public key component, which as we will see can be used for fake Janus mitigation).
    \item Os participantes usam as provas parciais para completar a prova de domínio agregada tipo {\em Bulletproof}. E cada um aplica uma técnica de produto interno logarítmico para que a compressão e a prova final seja incluída nos dados de transacção. Uma vez que todas as informações a serem assinadas por assinaturas do tipo MLSAG estão presentes. Cada participante completa as assinaturas MSLAG ás entradas e envia isso (com uma assinatura SAG para cada) para o canal de comunicação.\newline Cada participante pode submeter a transacção quando quiser.
%??? dentro do intervalo de comunicação ou depois?    

%Participants use all the partial proofs to complete the aggregate Bulletproof, and privately apply a logarithmic inner product technique to compress it for the final proof to be included in transaction data. Once all the information to be signed by MLSAGs is collected, each participant completes their inputs' MSLAGs and randomly sends them (with a SAG for each) to the channel over the communication interval. Any participant may submit the transaction as soon as they have all the pieces.
\end{enumerate}{}

\subsubsection*{Chaves públicas de transacção e a mitigação de Janus}

Se cada participante de um {\em TxTangle} sabe a chave privada de transacção $r$ (Section \ref{sec:one-time-addresses}), então qualquer um deles pode testar os endereços ocultos de saída, contra uma lista de endereços conhecidos. Por causa disto é necessário construír transacções de {\em TxTangle} como se existisse um recipiente com um sub-endereço (secção \ref{sec:subaddresses}), e também diferentes chaves públicas de transacção para cada saída.

%If every participant to a TxTangle knows the transaction private key $r$ (Section \ref{sec:one-time-addresses}), then any of them may test the others' one-time output addresses against a list of known addresses. For this reason it is necessary to construct TxTangle transactions as if there were a subaddress recipient (Section \ref{sec:subaddresses}), including different transaction public keys for each output.
Para complementar uma implementação possível de uma mitigação do ataque de Janus a sub-endereços, em que uma adicional chave pública de transacção `base' é incluída no campo extra \cite{janus-mitigation-issue-62}, transacções do tipo {\em TxTangle} também devem ter uma falsa chave pública composta por uma soma de chaves aleatórias geradas por cada membro desvio. Muitos participantes de uma transacção {\em TxTangle} que enviam dinheiro para um sub-endereço irão ter provavelmente duas saídas. Uma das quais diverte o troco de volta para o participante. Isto significa que cada participante pode activar uma mitigação de Janus ao tornarem a chave pública de transacção do troco também a chave `base' para o recipiente do sub-endereço.    

%To match with possible implementation of a mitigation for the Janus attack related to subaddresses, in which one additional `base' transaction public key is included in the extra field \cite{janus-mitigation-issue-62}, TxTangles should also have a fake `base' key composed of a sum of random keys generated by each mock-member.
\footnote{O cancelamento de chave (secção \ref{subsec:drawbacks-naive-aggregation-cancellation}) não devia ser um problema, desde que é só uma chave falsa e devia idealmente ser indexada de forma aleatória, entre a lista de chaves públicas de transacção.}
%Key cancellation (Section \ref{subsec:drawbacks-naive-aggregation-cancellation}) should not be a problem, since it's just a fake key and should ideally be randomly indexed within the list of transaction public keys.}
%Many TxTangle participants sending money to a subaddress will likely have at least two outputs, one of which diverts change back to the participant. This means any TxTangle participant can still enable Janus mitigation by making their change's transaction public key also the `base' key for the subaddress recipient.
\footnote{Se alguêm envia para o seu próprio sub-endereço, não existe a necessidade para uma mitigação de Janus. As carteiras que estão activadas para a mitigação de Janus deviam reconhecer que o montante gasto numa transacção TxTangle, é igual ao montante recebido pelo próprio sub-endereço. Para que o utilizador não seja notificado, de forma errónea, de um problema.}
%If sending to your own subaddress, there is no need for Janus mitigation. Wallets enabled for Janus mitigation should recognize the amount spent in a TxTangle equals the amount received to your subaddress, so they don't erroneously notify the user of a problem.} 
O recipiente do sub-endereço poderá realizar que a transacção é do tipo {\em TxTangle}, e que a chave `base' provavelmente corresponde á saída de transacção do troco do participante.     
%The subaddress recipient might realize the transaction is a TxTangle, and that the `base' key probably corresponds with the sender's change output.
\footnote{Isto assume que as chaves públicas de transacção são 1:1 com as saídas, como aparenta ser o caso actualmente. Se fosse standard para as chaves públicas de transacção estarem numa ordem aleatória ou estrita dentro do campo {\em extra}, então as transacções TxTangle e não TxTangle seriam largamente indistinguíveis para os destinatários com sub-endereços. Existem casos específicos em que os participantes de TxTangle são incapazes de incluir uma chave `base' (e.g. quando todas as saídas são para sub-endereços), ou quando se trata claramente de uma transacção não TxTangle, pois o destinatário do sub-endereço recebe a maioria ou todas as saídas. Note-se que transacções TxTangle geralmente têm muito mais saídas do que uma transacção típica, esta observação pode ser usada para diferenciar transacções normais de TxTangles.}
%??? para diferenciar transacções normais (com sub-endereços) de TxTangles.
%This is assuming transaction public keys are 1:1 with the outputs, as is apparently the case today. If it was standard for transaction public keys to be in random or sorted order within the extra field, then TxTangle and non-TxTangle transactions would be largely indistinguishable for subaddress recipients. There are niche cases where TxTangle participants are unable to include a `base' key (e.g. when all their outputs are to subaddresses), or where it is clearly non-TxTangle since the subaddress recipient receives most or all of the outputs. Note that since TxTangle transactions would generally have a lot more outputs than a typical transaction, this heuristic can be used to differentiate TxTangles from normal subaddress'd transactions.}


\subsection{Fraquezas}
\label{subsec:weaknesses-txtangle}

Actores maliciosos têm duas opções para derrotar o propósito de transacções do tipo {\em TxTangle}, que é de esconder conjuntos de saídas/entradas de transacção de analistas ou adversários. As transacções podem ser poluídas, tal que o conjunto de participantes é menor ou até não existente \cite{coinjoin-pollution}. Eles podem também tentar fazer com que tentativas de transacção {\em TxTangle} falhem. E usar as tentativas subsequentes pelos mesmos participantes para estimar conjuntos de saídas/entradas.
  
%Malicious actors have two primary ways to defeat the purpose of TxTangle, which is to hide input/output groupings from potential adversaries/analysts. They may pollute the transactions, such that the subset of honest participants is smaller (or even non-existent) \cite{coinjoin-pollution}. They may also cause TxTangle attempts to fail, and use subsequent attempts by the same participants to estimate input/output groupings.

O primeiro caso não é fácil de mitigar, especialmente no caso decentralizado em que nenhum participante tem uma reputação. Uma possível aplicação de {\em TxTangle} é a de piscinas colaborativas, que podem esconder a qual piscina pertencem os mineiros. Tais piscinas saberiam os conjuntos de entradas/saídas, mas desde que o propósito é ajudar os mineiros a elas ligados, existe uma motivação de os ajudar e de manter esta informação secreta. Para mais, tais transacções {\em TxTangle} não permitiriam maus actores, assumindo que as piscinas são honestas.
  
%The former case is not easy to mitigate, especially in the very decentralized case where no participant has a reputation. One possible application of TxTangle is with collaborating pools, who may hide which pool their miners belong to among a collection of pools. Such pools would know the input/output groupings, but since the purpose is helping their connected miners it would behoove them to keep the information secret. Moreover, such TxTangles would not permit bad actors, assuming the pools are somewhat honest.

O segundo caso pode ser evitado ao só tentar participar em transacções {\em TxTangle} poucas vezes antes de fazer uma pausa. E sempre que novas tentativas são feitas utilizar os elementos constituintes da transacção, regenerados de forma aleatória. Estes elementos são : as chaves públicas de transacção, máscaras de pseudo compromissos, escalares da prova de domínio, e escalares MLSAG. Em particular, o conjunto de membros desvio de anel para cada entrada devia permanecer o mesmo para prevenir comparações cruzadas que revelem a verdadeira entrada. Se possível, diferentes entradas verdadeiras deveriam ser utilizadas para tentativas de {\em TxTangle} diferentes. Desde que esta fraqueza é inevitável, torna a próxima secção mais atrativa.    

%The latter case can be defended against by only attempting to TxTangle a few times before taking a break, and by always regenerating most random elements of a transaction for new attempts. These elements include the transaction public keys, pseudo commitment masks, range proof scalars, and MLSAG scalars. In particular, the set of ring decoys for each input should remain the same to prevent cross-comparisons revealing the true input. If possible, different true inputs should be used for different TxTangle attempts. Since this weakness is inevitable, it makes the next section's concept more palatable.



\section{Servidor TxTangle}
\label{sec:hosted-txtangle}

{\em TxTangle} decentralizado tem algumas questões abertas. Como é que as rondas são iniciadas e enforçadas? Como é que são criados {\em chatrooms}, para que os participantes se encontrem mútuamente? A maneira mais simples é através de um servidor de {\em TxTangle}, que gera e mantêm esses {\em chatrooms}. 

%Truly decentralized TxTangle has some open questions. How are the timed rounds initiated and enforced? How are chatrooms created in the first place, for participants to find each other? The most straightforward way is with a TxTangle host, who generates and manages those chatrooms.

Um servidor desses parece desafiar o objectivo da participação obfuscada, desde que cada individuo tem de se ligar, e enviar mensagens que podem ser usadas para correlar entradas e saídas. Pode-se usar uma rede do tipo I2P para fazer com que cada mensagem recebida pelo servidor aparente provenir de um individuo único. 

%Such a host would seem to defy the goal of obfuscated participation, since each individual must connect, and send it messages that could be used to correlate input/output groupings (especially if the host participates and knows the message contents). We can use a network like I2P
\footnote{O projecto invisível da internet (\url{https://geti2p.net/en/}).}
%The Invisible Internet Project (\url{https://geti2p.net/en/}).} 
%to make each message received by the host appear as if from a unique individual.

\subsection{Comunicação básica com um servidor sobre I2P, e outras características}
\label{subsec:txtangle-host-communication}

Com I2P, utilisadores fazem assim chamados `túneis' que passam mensagens encriptadas por outros clientes antes de chegarem á destinação. Daquilo que se percebe, estes túneis podem transportar múltiplas mensagens antes de serem destruídos e recriados  
%With I2P, users make so-called `tunnels' that pass heavily encrypted messages through other users' clients before reaching their destination. From what we understand, these tunnels can transport multiple messages before being destroyed and recreated (aparenta haver um tempo limite de 10 minutos para estes túneis). 
É essencial para este uso, que haja um controlo cuidado sobre a abertura e o fecho dos túneis, bem como quais mensagens saem do mesmo túnel.
%It is essential for our use case to carefully control when new tunnels are created, and which messages may come out of the same tunnel.
\footnote{Em I2P existem túneis de `entrada' e de `saída' (see \url{https://geti2p.net/en/docs/how/tunnel-routing}). Tudo o que é recebido através de um túnel de entrada aparenta ser da mesma fonte mesmo que existam múltiplas fontes. Portanto á superficie parece que utilizadores de TxTangle não precisam de criar túneis distintos para cada um dos seus usos. Contudo se o anfitrião TxTangle faz com que ele próprio seja o ponto de entrada para o seu próprio tunel de entrada, então ele ganha o acesso directo aos túneis de saída dos participantes de TxTangle.}
%In I2P there are `outbound' and `inbound' tunnels (see \url{https://geti2p.net/en/docs/how/tunnel-routing}). Everything received through an inbound tunnel looks like it's from the same source even if from multiple sources, so on the surface it would appear TxTangle users don't need to create different tunnels for all their usecases. However, if the TxTangle host makes himself the entry point for his own inbound tunnel, then he gains direct access to the outbound tunnels of TxTangle participants.}%\footnote{For the sake of absolutely minimal information leaks, what we describe here is probably incredibly inefficient, especially since I2P is already very inefficient compared to the `clear' web.} is it?
\begin{enumerate}
    \item {\em Inscrever-se para TxTangles}: Na nossa proposta original com $n$ direcções (secção \ref{subsec:n-way-channel}) os participantes gradualmente adicionam os seus membros desvio a quartos de TxTangle disponíveis antes destes estarem agendados para fechar. Contudo se um volume suficientemente grande de utilisadores tenta fazer um TxTangle ao mesmo tempo, é provável de haver muitas falhas. Os utilizadores tentam de forma aleatória pór todas as suas saídas no mesmo `quarto' de TxTangle, mas assim os quartos enchem-se demasiadamente rápido, e como tal os utilizadores teriam de abortar o TxTangle.        
%In our original $n$-way proposal (Section \ref{subsec:n-way-channel}) participants gradually add their mock-members to available TxTangle rooms before they are scheduled to close. However, if a large enough volume of users try to TxTangle concurrently, there is likely to be a high rate of failure as users try to randomly put all their intended outputs in the same TxTangle `room', but then the rooms get full too soon so they have to back out. It would be quite a mess.

Pode-se fazer uma optimização ao dizer ao anfitrião quantas saídas estão em causa (e.g. dando-lhe uma lista das chaves públicas dos membros desvio), e deixar que ele junte os participantes para o TxTangle. Desde que os participantes retêm o protocolo de mensagens SAG e bLSAG, o anfitrião não será capaz de identificar conjuntos de saídas na transacção final. Tudo o que ele sabe é o número de participantes, e quantas saídas cada um tinha. Para mais, neste cenário os observadores não conseguem monitorizar quartos de TxTangle abertos para inferir qualquer informação sobre os participantes, uma melhoria importante na privacidade. Note-se que o poder do anfitrião de polluir os TxTangles não é significamente diferente do modelo sem anfitrião, portanto esta mudança é neutral para esse vector de ataque.   
%giving him a list of our mock-member public keys), 
%    We can make an impactful optimization by telling the host how many outputs we have (e.g. giving him a list of our mock-member public keys), and letting him assemble each TxTangle's participants. Since we still retain the bLSAG and SAG messaging protocol, the host won't be able to identify output groupings in the final transaction. All he knows is the number of participants, and how many outputs each had. Moreover, in this scenario observers can't monitor open TxTangle rooms to deduce information about the participants, an important privacy improvement. Note that the host's power to pollute TxTangles isn't significantly different from the non-host design, so this change is neutral to that attack vector.
    \item {\em Método de comunicação}: Como o anfitrião já actua como o lugar de transporte das mensagens, é simples que seja ele a gerir a comunicação de TxTangle. Durante cada ronda o anfitrião colecciona mensagens dos membros desvio (estas comunicações ocorrem sobre intervalos de tempo aleatórios). E no fim de uma ronda existe uma curta phase de distribuição em que ele envia os dados coleccionados a cada participante, a isto dá-se um certo tempo de espera antes da próxima ronda para que se garanta que as mensagens foram recebidas e que houve tempo suficiente para as processar.        
%Since the host is already acting as the locus of message transport, it is simplest for him to manage TxTangle communication. During each round the host collects messages from mock-members (still at random over a communication interval), and at the end of a round there is a short data distribution phase where he sends all the collected data to each participant, with a buffer period before the next round to ensure the messages are received and given time to process.% If someone controls multiple mock-members, it's possible that some of those messages just fizzle out, or perhaps that person receives duplicate messages (e.g. the host is given a different destination for each mock-member). The host should not realize how many real people he is distributing to, nor which mock-member has a given output.
    \item {\em Túneis e grupos de saídas/entradas}: Quando um TxTangle é iniciado, os participantes devem desasociar as identidades dos membros desvio das saídas verdadeiras. Isto significa criar novos túneis para mensagens assinadas tipo bLSAG. Cada túnel destes so pode transmitir mensagens relativamente a uma certa saída (a informação sobre uma saída vem sempre da mesma fonte). Os participantes devem também criar novos túneis para mensagens assinadas tipo SAG, respectivas a certas entradas.  
%Tunnels and input/output groupings}: Once a TxTangle has been initiated, users should dissociate their mock-member identities from the actual outputs that get created. This means creating new tunnels for bLSAG-signed messages, and each such tunnel may only transmit messages related to a specific output (it is fine to transmit multiple such messages through the same tunnel, since obviously information about the same output comes from the same source). They should also create new tunnels for SAG-signed messages related to specific inputs.
    \item {\em Ameaça de um ataque MITM pelo anfitrião}: 
O anfitrião pode enganar um participante ao pretender ser um outro participante, visto que ele controla o envio da lista de membros desvio para construir bLSAGs e SAGs. Por outras palavras a lista que ele envia ao participante A, pode conter os membros desvio do participante A, e todo o resto são os seus próprios. As mensagens recebidas pelo participante B são assinadas outra vez com a lista de A antes de ser re-transmitida a A. Como todas as mensagens assinadas por A pertencem a A, o anfitrião teria visibilidade directa para os agrupamentos de entradas/saídas de A.
%Threat of host MITM attack}: The host might fool a participant by pretending to be other participants, since he controls sending out the mock-member list for constructing bLSAGs and SAGs. In other words, the list he sends participant A might contain participant A's mock-members, and all the rest are his own. Messages received by participant B are re-signed using A's list before being retransmitted to A. Since all messages signed by A's list belong to A, the host would have direct insight into A's input/output groupings!

É possível prevenir que o anfitrião aja como MITM a partir de interações honestas entre os participantes, ao modificar como as chaves públicas de transacção são feitas. 
Os participantes enviam uns aos outros as chaves públicas de transacção (com uma assinatura bLSAG), depois, bem como na agregação robusta de chave da secção \ref{sec:robust-key-aggregation}, as chaves que de facto são incluídas nos dados de transacção (e que são utilizados para fazer máscaras de compromissos de saída, etc. ) são prefixos com uma hash da lista dos membros desvio. Por outras palavras, a {\em t-ésima} chave pública de transacção é :
\begin{align*}
\mathcal{H}_n(T_{agg},\mathbb{S}_{mock},r_t G)*r_t G
\end{align*}
Pôr a lista de membros desvio dentro da transacção faz com que seja muito difícil completar TxTangles sem que haja uma comunicação directa entre todos os actuais participantes. 
%We can prevent the host from acting as MITM of honest participant interactions by modifying how transaction public keys are made. Participants send each other their intended transaction public keys as normal (with a bLSAG), then, much like robust key aggregation from Section \ref{sec:robust-key-aggregation}, the actual keys that get included in transaction data (and used to make output commitment masks, etc.) are prefixed with a hash of the mock-member list. In other words, $\mathcal{H}_n(T_{agg},\mathbb{S}_{mock},r_t G)*r_t G$ is the $t$\nth transaction public key. Baking the mock-member list into the transaction itself makes it very difficult to complete TxTangles without direct communication between all actual participants.
\footnote{Se a mitigação de Janus for implementada, esta defesa ao ataque MITM devia em vez disso ser feita com a chave base falsa de Janus. Cada membro desvio oferece uma chave aleatória $r_{desvio} G$, depois a actual chave base é :
\begin{align*}
\sum_{desvio} \mathcal{H}_n(T_{agg},\mathbb{S}_{desvio},r_{desvio} G)*r_{desvio} G
\end{align*}
}
%If Janus mitigation is implemented, this MITM defense should instead be done with the fake Janus base key. Each mock-member provides a random key $r_{mock} G$, then the actual base key is $\sum_{mock} \mathcal{H}_n(T_{agg},\mathbb{S}_{mock},r_{mock} G)*r_{mock} G$.}
\end{enumerate}{}


\subsection{Um anfitrião como um serviço}

\label{subsec:txtangle-host-service}

É importante para a sustentabilidade e para a melhoria continua que um serviço de TxTangle opere com lucro. 
%It's important for sustainability and continuous improvement that a TxTangle service operate for profit.
\footnote{Enquanto que um serviço de TxTangle opere com lucro, o próprio código pode ser de código aberto. Isto é importante para auditar o software de carteira que interage com um tal serviço.} 
%While a deployed TxTangle service may be for profit, the code itself could be open source. This would be important for auditing wallet software that interacts with a TxTangle service.} 
Em vez de comprometer as identidades dos utilizadores com um modelo de conta, o anfitrião pode participar em cada TxTangle com uma só saída, e requerer que os participantes financiem essa saída. Quando os participantes acedem ao serviço/`eepsite' do anfitrião, são informados dessa taxa actual. Esta taxa do anfitrião deve ser paga por cada saída.   

%Rather than compromise user identities with an account-based model, the host may participate in each TxTangle as a lone output, and require the participants to fund it. When accessing the host's `eepsite'/service to apply for TxTangles, users are notified of the current hosting charge, which should be paid on a per-output basis.

Participantes seriam responsáveis por pagar fracções da taxa de mineiro {\em e} a taxa do anfitrião. Desta vez, a chave pública mais pequena do membro desvio (excluindo a chave do anfitrião) pagaria o resto de ambas as taxas. 
%Participants would be responsible for paying fractions of the fee {\em and} the host charge. This time, the smallest mock-member's public key (excluding the host's key) would take the remainder of both the fee and host charge.
\footnote{É necessário utilizar aqui as chaves do membros desvio, pois o anfitrião não paga taxa, e o seu index de saída é desconhecido.}
%We must use mock-member keys here since the host doesn't pay a fee, and his output index is unknown.} 
Desde que o anfitrião não tem nenhuma entrada, ele também não tem nenhum pseudo compromisso de saída para cancelar a sua máscara do compromisso de saída. Em vez disso ele cria segredos partilhados com os outros membros desvio, depois separa a sua máscara de compromisso em partes de tamanho aleatório e divide estas partes pelos segredos partilhados. Ele pública uma lista desses escalares, assina com a sua chave tal que os participantes saibam que se trata do anfitrião. Esta lista, ao aparecer assinála o começo da ronda n° 1 da secção \ref{subsec:message-rounds-txtangle} (e.g. o fim da ronda `0'). Membros desvio irão multiplicar o seu escalar do anfitrião pelo segredo partilhado apropriado, e adicionam isso á sua máscara de pseudo compromisso. Desta forma, mesmo a saída do anfitrião não pode ser identificada por qualquer participante na transacção final sem que haja uma coalição total contra ele.       
%Since the host has no inputs, he has no pseudo output commitment to cancel his output's commitment mask. Instead, he creates shared secrets with the other mock-members as usual, then separates his real commitment mask into randomly sized chunks for each other mock-member and divides them by the shared secrets. He publishes a list of those scalars (corresponding them with each other mock-member based on their public key), signing with his mock-member key so participants know it's from the host. The appearance of this list signals the beginning of round 1 from Section \ref{subsec:message-rounds-txtangle} (e.g. the end of setup round `0'). Mock-members will multiply their host-scalar by the appropriate shared secret, and add that to their pseudo commitment mask. In this way, even the host's output cannot be identified by any participant in the final transaction without a full coalition against him.

Para simplificar as calculações da taxa, o anfitrião pode distribuir a taxa total a ser usada na transacção no fim da ronda n° 1, pois ele irá saber qual é o peso da transacção relativamente cedo. Os participantes podem verificar que esse montante é semelhante ao montante esperado, e pagar uma fracção do mesmo.  
%To simplify calculations of the fee, the host may distribute the total fee to be used in the transaction at the end of round 1, since he will learn the transaction weight early. Participants can verify that amount is similar to the expected amount, and pay their fraction of it.

Se os participantes colaboram para enganar, e não querem pagar o preço do anfitrião, então o anfitrião pode terminar o TxTangle na ronda n° 3. Ou então se as mensagens que aparecem no canal não deviam estar ali ou são inválidas. 

%If participants collaborate to cheat, and not provide the hosting charge, then the host may terminate the TxTangle at round 3. He may also terminate if messages appear in the channel that should not be there or that aren't valid.

No fim da ronda 5 o anfitrião completa a transacção e submete isso á rede para verificação, como parte do seu serviço. Ele incluí a hash da transacção na mensagem que é distribuida no fim.

%At the end of round 5 the host completes the transaction and submits it to the network for verification, as part of his service. He includes the transaction hash in the final distribution message.

\section{Usar um administrador á confiança}

\label{sec:dealer-txtangle}

Existem aspectos negativos para o TxTangle decentralizado. Requer que os participantes se encontrem uns aos outros, e que comuniquem activamente dentro de um tempo planeado. Isto é também difícil de implementar em código. 
%There are some drawbacks to decentralized TxTangle. It requires all participants actively communicate within a strict timing schedule (and find each other to begin with), and is challenging to implement.

Usar um administrador central, que é responsável para coleccionar informação de transacção de cada participante e obfuscar os grupos de entrada/saída, simplifica o processo. O custo é ter uma entidade terçeira á confiança, que sabe (no mínimo) esses grupos de entrada/saída.
%Adding a central dealer, who is responsible for collecting transaction information from each participant and obfuscating the input/output groupings, can simplify the proceedings. The cost is a higher bar of trust, since the dealer must (at minimum) learn those groupings.
\footnote{Esta secção é inspirada pelo protocolo {\em MoJoin}.}

\subsection{Processo baseado num administrador}
\label{subsec:dealer-procedure-txtangle}

O administrador revela que ele está disponível para gerir TxTangles, e colecciona inscrições de participantes potenciais (que consistem do número de entradas [com os seus tipos] e saídas). Ele pode se quiser participar com o seu próprio conjunto de saídas/entradas.
%consisting of the number of intended inputs [with their types] and outputs).The dealer will advertise that he is available to manage TxTangles, and collects applications from potential participants (consisting of the number of intended inputs [with their types] and outputs). He may participate with his own input/output set if he wishes.

Depois de quase 16 saídas terem sido juntadas num grupo (têm de existir pelo menos dois ou mais participantes, e nenhum participante deve ter todas as entradas ou saídas), o administrador começa a primeira de 5 rondas. Em cada ronda ele acumula informação de cada participante, faz algumas decisões, e envia mensagens que significam o começo de uma nova ronda.
%After a grouping of almost 16 outputs is assembled (there should be two or more participants, and no participant may have all but one output or input), the dealer will start the first of five rounds. In each round he accumulates information from each participant, makes some decisions, and sends out messages that signify the beginning of a new round.
\begin{enumerate}
    \item Para começar, o administrador gera, para cada par de participantes, um escalar aleatório, e decide quem nesse par recebe a versão negativa desse escalar. Ele usa o número e tipo de entradas e saídas, para estimar a taxa total necessária. Ele faz a soma dos escalares dos participantes, e de forma privada envia para cada um essa soma, juntamente com a taxa que eles devem pagar e os indíces das suas saídas (escolhidas de forma aleatória). Estas mensagens constituem um sinal para os participantes que um TxTangle está a começar.   
%To begin, the dealer generates, for each pair of participants, a random scalar, and decides which in each pair should have the positive or negative version. He uses the number and type of inputs and outputs to estimate the total fee required. He sums each participant's scalars together, and privately sends to each their sum, along with the fee fraction they are expected to pay for, and the (randomly chosen) indices of their outputs. These messages constitute a signal to participants that a TxTangle is beginning.
    \item Cada participante constroí a sua sub-transacção como se fosse uma transacção normal, com chaves públicas de transacção para as suas saídas (com uma mitigação de Janus se necessário), calcular endereços ocultos de saída, codificar montantes de saída, fazer pseudo compromissos de saída que igualam os compromissos de saída mais a taxa para os mineiros, fazer uma lista de offsets de membros de anel para o uso em assinaturas MLSAG juntamente com as imagens de chave respectivas, e adicionar a um dos seus pseudo compromissos de saída, o escalar enviado pelo administrador (multiplicado por $G$). Eles criam a parte A de provas parciais para as suas saídas, e enviam toda esta informação ao administrador. O administrador verifica que os montantes de entrada e saída se igualam, e depois envia a lista completa da parte A de provas parciais a cada participante.      

%Each participant constructs their sub-transaction as they normally would, generating separate transaction public keys for their outputs (with Janus mitigation as needed), calculating one-time output addresses and encoding output amounts, making pseudo output commitments that balance output commitments and the fee fraction, assembling a list of ring member offsets for use in MLSAG signatures along with the appropriate key images, and add to one of their pseudo output commitments the dealer-sent scalar (multiplied by $G$). They create Part A partial proofs for their outputs, and send all this information to the dealer. The dealer verifies that input and output amounts balance, and sends the complete list of Part A partial proofs to each participant.
    \item Cada participante calcula o desafio agregado A, e gera a parte B de provas parciais que depois são enviadas ao administrador. Este por sua vez colecciona as provas parciais e distribui estas por todos os participantes.
%Each participant computes the aggregate challenge A, and generates Part B partial proofs which they send to the dealer. The dealer collects the partial proofs and distributes them to all the other participants.
    \item Cada participante calcula o desafio B, e gera a parte C de provas parciais que eles enviam ao administrador. O administrador por sua vez colecciona estas provas e aplica-lhes a técnica do produto interno logarítmico para as comprimir para a prova final. Assumindo que a prova se verifica como devia, ele gera uma chave pública `base' de transacção falsa tipo Janus, e envia a mensagem para ser assinada em MLSAGs para cada participante.
%???
%Each participant computes the aggregate challenge B, and generates Part C partial proofs which they send to the dealer. The dealer collects these, and applies the logarithmic inner product technique to compress them into the final proof. Assuming the proof verifies as it should, he generates a random fake Janus `base' transaction public key, and sends the message to be signed in MLSAGs to each participant.
    \item Cada participante completa a sua assinatura MLSAG e envia isso ao administrador. Uma vez que este tem todas as peças, ele pode finalizar a construção da transacção, e submeter isso á rede. Ele pode também enviar a ID da transacção a cada participante para que estes confirmem que ela foi publicada.
%Each participant completes their MLSAGs and sends them to the dealer. Once he has all the pieces, he may finish constructing the transaction, and submit it to be included in the blockchain. He may also send the transaction ID to each participant so they can confirm it was published.
\end{enumerate}{}
