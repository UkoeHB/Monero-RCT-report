% this file is called up by main.tex
% content in this file will be fed into the main document

% ---------------------------------------------------------------------------

Crittografia. Potrebbe sembrare che solo matematici e informatici abbiano accesso a questo argomento oscuro, esoterico, potente ed elegante. In realtà, diverse tipologie di crittografia sono abbastanza semplici da poter essere comprese da chiunque ne apprenda i concetti fondamentali.
\\ \newline
È risaputo che la crittografia viene utilizzata per proteggere le comunicazioni, che si tratti di messaggi cifrati o interazioni digitali private. Un'altra applicazione si trova nelle cosiddette criptovalute. Queste valute digitali utilizzano la crittografia per assegnare e trasferire la proprietà dei fondi. Per garantire che nessuna unità di valuta possa essere duplicata o creata arbitrariamente, le criptovalute si basano generalmente su blockchain, ovvero registri pubblici e distribuiti contenenti tutte le transazioni di valuta, verificabili da terze parti \cite{Nakamoto_bitcoin}.
\\ \newline
A prima vista potrebbe sembrare che le transazioni debbano essere inviate e archiviate in modo trasparente per poter essere pubblicamente verificabili. In realtà, è possibile nascondere sia i partecipanti di una transazione sia gli importi coinvolti, utilizzando strumenti crittografici che permettono comunque la verifica e il raggiungimento del consenso da parte di osservatori esterni \cite{cryptoNoteWhitePaper}. Questo approccio è esemplificato nella criptovaluta Monero.
\\ \newline
L'obiettivo di questo documento è spiegare a chiunque conosca l'algebra di base e alcuni semplici concetti informatici, come la 'rappresentazione binaria di un numero', non solo come funziona Monero in modo approfondito e completo, ma anche quanto possa essere utile e affascinante la crittografia.
\\ \newline
Per i lettori più esperti: Monero è una criptovaluta basata su una blockchain standard, a grafo aciclico diretto (DAG) unidimensionale \cite{Nakamoto_bitcoin} in cui le transazioni utilizzano la crittografia a curve ellittiche (in particolare la curva Ed25519 \cite{Bernstein2008}). Gli input delle transazioni sono firmati con firme di gruppo anonime multilivello, spontanee, linkabili, in stile Schnorr (MLSAG) \cite{MRL-0005-ringct}, mentre gli importi in uscita (comunicati ai destinatari tramite ECDH \cite{Diffie-Hellman}) sono nascosti mediante impegni di Pedersen \cite{maxwell-ct} e dimostrati appartenere ad un intervallo valido grazie ai Bulletproof \cite{Bulletproofs_paper}. Gran parte della prima sezione di questo documento è dedicata a spiegare questi concetti.