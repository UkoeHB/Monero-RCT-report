\begin{appendices}

\renewcommand{\theFancyVerbLine}{%
	\textcolor{red}{\small
		\arabic{FancyVerbLine}}}

\chapter{Struttura delle Transazioni {\tt RCTTypeBulletproof2}}
\label{appendix:RCTTypeBulletproof2}

In questa appendice presentiamo la struttura, anche detto \emph{dump}, di una transazione Monero reale di tipo {\tt RCTTypeBulletproof2}, insieme a note esplicative per i campi rilevanti.

Questo dump è stato ottenuto dal block explorer \url{https://xmrchain.net}, ma può essere reperito anche eseguendo il comando {\tt print\_tx <TransactionID> +hex +json} attraverso l'eseguibile {\tt monerod} avviato in modalità non-detached. {\tt <TransactionID>} è l'hash della transazione (Sezione \ref{subsec:transaction-id}). La prima linea del dump indica il comando eseguito.% see chapter 7 blockchain on TransactionID

Per replicare gli stessi risultati, i lettori possono seguire i seguenti passaggi:%\footnote{Blockchain data can also be found with a web block explorer, such as \url{https://moneroblocks.info/} or \url{https://xmrchain.net}.}
\begin{enumerate}
    \item È necessario predisporre un ambiente dotato dello strumento Linea di Comando di Monero (command line tool CLI), che può essere scaricato dal sito ufficiale \url{https://web.getmonero.org/downloads/} (oltre che da altri posti). È sufficiente scaricare la CLI per il proprio sistema operativo, spostare il file in una posizione opportuna nel file system ed infine estrarre l'archivio zip.
    \item Aprire dunque un terminale/prompt dei comandi e navigare nella cartella creata in seguito all'estrazione.
    \item Eseguire {\tt monerod} tramite il comando {\tt ./monerod}. In seguito, si avvierà il processo di sincronizzazione che scaricherà una copia della blockchain in locale. Purtroppo, non c'è un altro modo semplice per visualizzare le transazioni nel proprio ambiente (ad esempio senza usare un servizio blockchain explorer esterno) senza scaricare la blockchain.
    \item Una volta che il processo di sincronizzazione si completa, sarà possibile invocare comandi come {\tt print\_tx} dal terminale. È disponibile il comando {\tt help} per una panoramica sui comandi disponibili.
\end{enumerate}

%For editorial reasons we have shortened long hexadecimal chains, presenting only the beginning and end as in {\tt 0200010c7f[...]409}.

Il campo {\tt rctsig\_prunable}, come indica il nome, si può {\sl potare} (rimuovere) dalla propria copia della blockchain. Ciò significa che una volta che è stato raggiunto il consenso per il blocco e l'attuale lunghezza della catena esclude tutte le possibilità di attacchi a doppia spesa, l'intero campo può essere potato e sostituito con il suo hash, in modo da poter essere utilizzato all'interno dell'albero di Merkle.

Le immagini chiave (key image) vengono archiviate separatamente, nell'area non potabile delle transazioni. Sono essenziali per rilevare gli attacchi di doppia spesa e non possono essere rimosse.
\\

La nostra transazione di esempio ha 2 input e 2 output ed è stata aggiunta alla blockchain al timestamp 2020-03-02 19:01:10 UTC (come riportato dal miner del blocco).

\begin{Verbatim}[commandchars=\\\{\}, numbers=left]
print_tx 84799c2fc4c18188102041a74cef79486181df96478b717e8703512c7f7f3349
Found in blockchain at height 2045821
\{
  "version": 2, 
  "unlock_time": 0, 
  "vin": [ \{
      "key": \{
        "amount": 0, 
        "key_offsets": [ 14401866, 142824, 615514, 18703, 5949, 22840, 5572, 16439,
        983, 4050, 320
        ], 
        "k_image": "c439b9f0da76ca0bb17920ca1f1f3f1d216090751752b091bef9006918cb3db4"
      \}
    \}, \{
      "key": \{
        "amount": 0, 
        "key_offsets": [ 14515357, 640505, 8794, 1246, 20300, 18577, 17108, 9824, 581,
        637, 1023
        ], 
        "k_image": "03750c4b23e5be486e62608443151fa63992236910c41fa0c4a0a938bc6f5a37"
      \}
    \}
  ], 
  "vout": [ \{
      "amount": 0, 
      "target": \{
        "key": "d890ba9ebfa1b44d0bd945126ad29a29d8975e7247189e5076c19fa7e3a8cb00"
      \}
    \}, \{
      "amount": 0, 
      "target": \{
        "key": "dbec330f8a67124860a9bfb86b66db18854986bd540e710365ad6079c8a1c7b0"
      \}
    \}
  ], 
  "extra": [ 1, 3, 39, 58, 185, 169, 82, 229, 226, 22, 101, 230, 254, 20, 143,
  37, 139, 28, 114, 77, 160, 229, 250, 107, 73, 105, 64, 208, 154, 182, 158, 200,
  73, 2, 9, 1, 12, 76, 161, 40, 250, 50, 135, 231
  ], 
  "rct_signatures": \{
    "type": 4, 
    "txnFee": 32460000, 
    "ecdhInfo": [ \{
        "amount": "171f967524e29632"
      \}, \{
        "amount": "5c2a1a9f54ccf40b"
      \}], 
    "outPk": [ "fed8aded6914f789b63c37f9d2eb5ee77149e1aa4700a482aea53f82177b3b41",
    "670e086e40511a279e0e4be89c9417b4767251c5a68b4fc3deb80fdae7269c17"]
  \}, 
  "rctsig_prunable": \{
    "nbp": 1, 
    "bp": [ \{
        "A": "98e5f23484e97bb5b2d453505db79caadf20dc2b69dd3f2b3dbf2a53ca280216", 
        "S": "b791d4bc6a4d71de5a79673ed4a5487a184122321ede0b7341bc3fdc0915a796", 
        "T1": "5d58cfa9b69ecdb2375647729e34e24ce5eb996b5275aa93f9871259f3a1aecd", 
        "T2": "1101994fea209b71a2aa25586e429c4c0f440067e2b197469aa1a9a1512f84b7", 
        "taux": "b0ad39da006404ccacee7f6d4658cf17e0f42419c284bdca03c0250303706c03", 
        "mu": "cacd7ca5404afa28e7c39918d9f80b7fe5e572a92a10696186d029b4923fa200", 
        "L": [ "d06404fc35a60c6c47a04e2e43435cb030267134847f7a49831a61f82307fc32",
        "c9a5932468839ee0cda1aa2815f156746d4dce79dab3013f4c9946fce6b69eff",
        "efdae043dcedb79512581480d80871c51e063fe9b7a5451829f7a7824bcc5a0b",
        "56fd2e74ac6e1766cfd56c8303a90c68165a6b0855fae1d5b51a2e035f333a1d",
        "81736ed768f57e7f8d440b4b18228d348dce1eca68f969e75fab458f44174c99",
        "695711950e076f54cf24ad4408d309c1873d0f4bf40c449ef28d577ba74dd86d",
        "4dc3147619a6c9401fec004652df290800069b776fe31b3c5cf98f64eb13ef2c"
        ], 
        "R": [ "7650b8da45c705496c26136b4c1104a8da601ea761df8bba07f1249495d8f1ce",
        "e87789fbe99a44554871fcf811723ee350cba40276ad5f1696a62d91a4363fa6",
        "ebf663fe9bb580f0154d52ce2a6dae544e7f6fb2d3808531b0b0749f5152ddbf",
        "5a4152682a1e812b196a265a6ba02e3647a6bd456b7987adff288c5b0b556ec6",
        "dc0dcb2e696e11e4b62c20b6bfcb6182290748c5de254d64bf7f9e3c38fb46c9",
        "101e2271ced03b229b88228d74b36088b40c88f26db8b1f9935b85fb3ab96043",
        "b14aae1d35c9b176ac526c23f31b044559da75cf95bc640d1005bfcc6367040b"
        ], 
        "a": "4809857de0bd6becdb64b85e9dfbf6085743a8496006b72ceb81e01080965003", 
        "b": "791d8dc3a4ddde5ba2416546127eb194918839ced3dea7399f9c36a17f36150e", 
        "t": "aace86a7a1cbdec3691859fa07fdc83eed9ca84b8a064ca3f0149e7d6b721c03"
      \}
    ], 
    "MGs": [ \{
        "ss": [[ "d7a9b87cfa74ad5322eedd1bff4c4dca08bcff6f8578a29a8bc4ad6789dee106",
        "f08e5dfade29d2e60e981cb561d749ea96ddc7e6855f76f9b807842d1a17fe00"],
        ["de0a86d12be2426f605a5183446e3323275fe744f52fb439041ad2d59136ea0b",
        "0028f97976630406e12c54094cbbe23a23fe5098f43bcae37339bfc0c4c74c07"],
        ["f6eef1f99e605372cb1ec2b3dd4c6e56a550fec071c8b1d830b9fda34de5cc05",
        "cd98fc987374a0ac993edf4c9af0a6f2d5b054f2af601b612ea118f405303306"],
        ["5a8437575dae7e2183a1c620efbce655f3d6dc31e64c96276f04976243461e08",
        "5090103f7f73a33024fbda999cd841b99b87c45fa32c4097cdc222fa3d7e9502"],
        ["88d34246afbccbd24d2af2ba29d835813634e619912ea4ca194a32281ac14e0e",
        "eacdf59478f132dd8dbb9580546f96de194092558ffceeff410ee9eb30ce570e"],
        ["571dab8557921bbae30bda9b7e613c8a0cff378d1ec6413f59e4972f30f2470d",
        "5ca78da9a129619299304d9b03186233370023debfdaddcd49c1a338c1f0c50d"],
        ["ac8dbe6bb28839cf98f02908bd1451742a10c713fdd51319f2d42a58bf1d7507",
        "7347bf16cba5ee6a6f2d4f6a59d1ed0c1a43060c3a235531e7f1a75cd8c8530d"],
        ["b8876bd3a5766150f0fbc675ba9c774c2851c04afc4de0b17d3ac4b6de617402",
        "e39f1d2452d76521cbf02b85a6b626eeb5994f6f28ce5cf81adc0ff2b8adb907"],
        ["1309f8ead30b7be8d0c5932743b343ef6c0001cef3a4101eae98ffde53f46300",
        "370693fa86838984e9a7232bca42fd3d6c0c2119d44471d61eee5233ba53c20f"],
        ["80bc2da5fc5951f2c7406fce37a7aa72ffef9cfa21595b1b68dfab4b7b9f9f0c",
        "c37137898234f00bce00746b131790f3223f97960eefe67231eb001092f5510c"],
        ["01c89e07571fd365cac6744b34f1b44e06c1c31cbf3ee4156d08309345fdb20e",
        "a35c8786695a86c0a4e677b102197a11dadc7171dd8c2e1de90d828f050ec00f"]], 
        "cc": "0d8b70c600c67714f3e9a0480f1ffc7477023c793752c1152d5df0813f75ff0f"
      \}, \{
        "ss": [[ "4536e585af58688b69d932ef3436947a13d2908755d1c644ca9d6a978f0f0206",
        "9aab6509f4650482529219a805ee09cd96bb439ee1766ced5d3877bf1518370b"],
        ["5849d6bf0f850fcee7acbef74bd7f02f77ecfaaa16a872f52479ebd27339760f",
        "96a9ec61486b04201313ac8687eaf281af59af9fd10cf450cb26e9dc8f1ce804"],
        ["7fe5dcc4d3eff02fca4fb4fa0a7299d212cd8cd43ec922d536f21f92c8f93f00",
        "d306a62831b49700ae9daad44fcd00c541b959da32c4049d5bdd49be28d96701"],
        ["2edb125a5670d30f6820c01b04b93dd8ff11f4d82d78e2379fe29d7a68d9c103",
        "753ac25628c0dada7779c6f3f13980dfc5b7518fb5855fd0e7274e3075a3410c"],
        ["264de632d9cb867e052f95007dfdf5a199975136c907f1d6ad73061938f49c01",
        "dd7eb6028d0695411f647058f75c42c67660f10e265c83d024c4199bed073d01"],
        ["b2ac07539336954f2e9b9cba298d4e1faa98e13e7039f7ae4234ac801641340f",
        "69e130422516b82b456927b64fe02732a3f12b5ee00c7786fe2a381325bf3004"],
        ["49ea699ca8cf2656d69020492cdfa69815fb69145e8f922bb32e358c23cebb0f",
        "c5706f903c04c7bed9c74844f8e24521b01bc07b8dbf597621cceeeb3afc1d0c"],
        ["a1faf85aa942ba30b9f0511141fcab3218c00953d046680d36e09c35c04be905",
        "7b6b1b6fb23e0ee5ea43c2498ea60f4fcf62f70c7e0e905eb4d9afa1d0a18800"],
        ["785d0993a70f1c2f0ac33c1f7632d64e34dd730d1d8a2fb0606f5770ed633506",
        "e12777c49ffc3f6c35d27a9ccb3d9b8fed7f0864a880f7bae7399e334207280e"],
        ["ab31972bf1d2f904d6b0bf18f4664fa2b16a1fb2644cd4e6278b63ade87b6d09",
        "1efb04fe9a75c01a0fe291d0ae00c716e18c64199c1716a086dd6e32f63e0a07"],
        ["a6f4e21a27bf8d28fc81c873f63f8d78e017666adbf038da0b83c2ad04ef6805",
        "c02103455f93c2d7ec4b7152db7de00d1c9e806b1945426b6773026b4a85dd03"]], 
        "cc": "d5ac037bb78db41cf924af713b7379c39a4e13901d3eac017238550a1a3b910a"
      \}],
    "pseudoOuts": [ "b313c1ae9ca06213684fbdefa9412f4966ad192bc0b2f74ed1731381adb7ab58",
    "7148e7ef5cfd156c62a6e285e5712f8ef123575499ff9a11f838289870522423"]
  \}
\}
\end{Verbatim}



\section*{Campi di una Transazione}

\begin{itemize}
    \item (linea 2) - il comando {\tt print\_tx} riporta l'altezza del blocco in cui è stata trovata la transazione, riportato qui a scopo dimostrativo.
	\item {\tt version} (linea 4) - Formato/era della versione della transazione; `2' corrisponde a RingCT.
	\item {\tt unlock\_time} (linea 5) - Impedisce che gli output di una transazione vengano spesi fino al termine del tempo specificato. Può essere un'altezza di blocco o un timestamp UNIX, se il numero è maggiore dell'inizio del tempo UNIX. Il valore predefinito è zero quando non viene specificato alcun limite.
	\item {\tt vin} (linea 6-23) - Lista di input (composta da due input in questo caso).
	\item {\tt amount} (linea 8) - Campo deprecato (legacy), era dedicato all'importo della transazione (per transazioni di tipo 1).
	\item {\tt key\_offset} (linea 9) - Questo campo consente ai verificatori di trovare le chavi appartenenti all'anello e i commitment nella blockchain, al fine di constatare la legittimità di questi membri. Il primo offset è un indice assoluto rispetto alla cronologia della blockchain, mentre i successi sono indici relativi al primo offset. Ad esempio, con offset reali (assoluti) \{7,11,15,20\}, su blockchain verranno registrati come \{7,4,4,5\}. I verificatori possono calcolare l'ultimo offset sommando tutti gli indici (7+4+4+5 = 20) (Sezione \ref{subsec:space-and-ver-rcttypefull}).
	\item {\tt k\_image} (linea 12) - Immagine chiave \(\tilde{K_j}\) trattata nella Sezione \ref{sec:MLSAG}, dove $j = 1$ dato che è il primo input.
	\item {\tt vout} (linee 24-35) - Lista di output (composta da due output in questo caso).
	\item {\tt amount} (linea 25) - Campo deprecato dedicato all'importo delle transazioni di tipo 1.
	\item {\tt key} (linea 27) - Chiave dell'indirizzo one-time di destinazione per l'output $t = 0$ come descritto nella Sezione \ref{sec:one-time-addresses}
	\item {\tt extra} (linee 36-39) - Dati aggiuntivi\marginnote{src/crypto- note\_basic/ tx\_extra.h}, inclusa la {\em chiave pubblica della transazione}, ovvero il segreto condiviso $r G$ della Sezione  \ref{sec:one-time-addresses}, e il payment ID cifrato della Sezione \ref{sec:integrated-addresses}. Tipicamente funziona come segue: Ogni numero rappresenta un byte (può assumere un valore compreso tra 0 e 255), e ogni tipo di elemento che può comparire nel campo ha un proprio \emph{tag} (etichetta) e una \emph{length} (lunghezza). Il \emph{tag} indica quale tipo di informazione segue, mentre \emph{length} specifica quanti byte occupa quell'informazione. Il primo numero è sempre un tag. In questo caso, `1' indica una `chiave pubblica della transazione' (transaction public key). Le chiavi pubbliche delle transazioni sono sempre lunghe 32 byte, quindi non è necessario specificarne la lunghezza. Trenta-due byte dopo troviamo un nuovo tag `2', che indica una `extra nonce', la cui lunghezza è `9'; il byte successivo è `1', che segnala un payment ID cifrato a 8 byte (l’extra nonce può contenere campi al suo interno, per qualche ragione). Seguono otto byte, e questo segna la fine del campo extra. Per ulteriori dettagli, si veda \cite{extra-field-stackexchange}. (Nota: nella specifica originale di Cryptonote, il primo byte indicava la dimensione del campo. Monero non utilizza questa convenzione). \cite{tx-extra-field}
	\item {\tt rct\_signatures} (linee 40-50) - Prima parte della firma.
	\item {\tt type} (linea 41) - Tipologia di firma; {\tt RCTTypeBulletproof2} tipo 4. I tipi RingCT {\tt RCTTypeFull} e {\tt RCTTypeSimple} sono deprecati, ovvero tipo 1 e 2 rispettivamente. Le transazioni di mining usano {\tt RCTTypeNull}, ovvero il tipo 0.
	\item {\tt txnFee} (linea 42) - Commissioni della transazione in chiaro, in queso caso 0.00003246 XMR.
	\item {\tt ecdhInfo} (linee 43-47) - ‘Informazione sulla curva ellittica Diffie-Hellman’: Importi offuscati per ogni output $t \in \{0, ..., p-1\}$; in questo caso $p = 2$.
    \item {\tt amount} (linea 44) - Campo {\sl ammontare} dell'output $t = 0$ come descritto nella Sezione \ref{sec:pedersen_monero}.
    \item {\tt outPk} (linee 48-49) - Commitment per ogni output, Sezione \ref{sec:ringct-introduction}.

    \item {\tt rctsig\_prunable} (linee 51-132) - Seconda parte della firma.
    \item {\tt nbp} (linea 52) - Numero di prove di intervallo (range proof) Bulletproof in questa transazione.
    \item {\tt bp} (linee 53-80) - Prove Bulletproof (Si ricorda che i Bulletproof non sono stati trattati in questo documento)\vspace{.175cm}
    \[\Pi_{BP} = (A, S, T_1, T_2, \tau_x, \mu, \mathbb{L}, \mathbb{R}, a, b, t)\]
    \item {\tt MGs} (linee 81-129) - Firme MLSAG.
    \item {\tt ss} (linee 82-103) - Componenti \(r_{i,1}\) e \(r_{i,2}\) derivanti dalla firma MLSAG del primo output\vspace{.175cm}
    \[\sigma_j(\mathfrak{m}) = (c_1, r_{1, 1}, r_{1, 2}, ..., r_{v+1, 1}, r_{v+1, 2})\]
    \item {\tt cc} (linea 104) - Componente \(c_1\) della sopracitata firma MLSAG.
    \item {\tt pseudoOuts} (linee 130-131) - Commitment pseudo-output $C'^a_j$, come descritto nella Sezione  \ref{sec:pedersen_monero}. Si ricorda che la somma di questi commitment è pari alla somma dei commitment dei due output di questa transazione (più i commitment delle commissioni della transazione $f H$).
\end{itemize}




\chapter{Contenuto dei Blocchi}
\label{appendix:block-content}

In questo appendice viene illustrata la struttura di un blocco della catena, in particolare il 1582196\textsuperscript{esimo} dopo il blocco genesi. Il blocco in questione ha 5 transazioni che sono state registrate sulla blockchain al timestamp 2018-05-27 21:56:01 UTC (come riportato dal miner del blocco).

\begin{Verbatim}[commandchars=\\\{\}, numbers=left]
print_block 1582196
timestamp: 1527458161
previous hash: 30bb9b475a08f2ea6fe07a1fd591ea209a7f633a400b2673b8835a975348b0eb
nonce: 2147489363
is orphan: 0
height: 1582196
depth: 2
hash: 50c8e5e51453c2ab85ef99d817e166540b40ef5fd2ed15ebc863091ca2a04594
difficulty: 51333809600
reward: 4634817937431
\{
  "major_version": 7,
  "minor_version": 7,
  "timestamp": 1527458161,
  "prev_id": "30bb9b475a08f2ea6fe07a1fd591ea209a7f633a400b2673b8835a975348b0eb",
  "nonce": 2147489363,
  "miner_tx": \{
    "version": 2,
    "unlock_time": 1582256,
    "vin": [ \{
        "gen": \{
          "height": 1582196
        \}
      \}
    ],
    "vout": [ \{
        "amount": 4634817937431,
        "target": \{
          "key": "39abd5f1c13dc6644d1c43db68691996bb3cd4a8619a37a227667cf2bf055401"
        \}
      \}
    ],
    "extra": [ 1, 89, 148, 148, 232, 110, 49, 77, 175, 158, 102, 45, 72, 201, 193,
    18, 142, 202, 224, 47, 73, 31, 207, 236, 251, 94, 179, 190, 71, 72, 251, 110, 
    134, 2, 8, 1, 242, 62, 180, 82, 253, 252, 0
    ],
    "rct_signatures": \{
      "type": 0
    \}
  \},
  "tx_hashes": [ "e9620db41b6b4e9ee675f7bfdeb9b9774b92aca0c53219247b8f8c7aecf773ae",
                 "6d31593cd5680b849390f09d7ae70527653abb67d8e7fdca9e0154e5712591bf",
                 "329e9c0caf6c32b0b7bf60d1c03655156bf33c0b09b6a39889c2ff9a24e94a54",
                 "447c77a67adeb5dbf402183bc79201d6d6c2f65841ce95cf03621da5a6bffefc",
                 "90a698b0db89bbb0704a4ffa4179dc149f8f8d01269a85f46ccd7f0007167ee4"
  ]
\}
\end{Verbatim}


\section*{Campi di un Blocco}

\begin{itemize}
	\item (linee 2-10) - Informazioni raccolte dal nodo (monerod) che non appartengono propriamente alla struttura del blocco.
    \item {\tt is orphan} (linea 5) - Indica se il blocco è orfano. Di solito, i nodi della rete conservano tutti i rami durante una situazione di biforcazione della blockchain, e scartano i rami non necessari quando emerge il ramo con difficoltà cumulativa più alta, lasciando di conseguenza dei blocchi orfani.
    \item {\tt depth} (linea 7) - In una copia di una blockchain, la profondità di un blocco è la distanza relativa al blocco più recente aggiunto alla catena.
    \item {\tt hash} (line 8) - L'identificativo del blocco (block ID).
    \item {\tt difficulty} (linea 9) - La difficoltà della rete non è registrata in un blocco, dato che gli utenti possono dedurre la difficoltà di un blocco dal relativo timestamp secondo le modalità descritte nella Sezione \ref{sec:difficulty}.
    \item {\tt major\_version} (linea 12) - Indica la versione del protocollo utilizzata per validare questo blocco.
    \item {\tt minor\_version} (linea 13) - In origine era usato per un meccanismo di voti dedicato ai miner, adesso segue lo stesso scopo di {\tt major\_version}. Dal momento che il campo non occupa molto spazio, probabilmente gli sviluppatori hanno pensato che lo sforzo di eliminare questo campo non valesse l'eventuale beneficio tratto in termini di risparmio di memoria.
    \item {\tt timestamp} (linea 14) - La rappresentazione del timestamp UTC del blocco sotto forma di numero interno, come riportato dal miner del blocco.
    \item {\tt prev\_id} (linea 15) - L'identificativo del blocco precedente. Qui risiede l'essenza della blockchain di Monero.
    \item {\tt nonce} (linea 16) - Il nonce utilizzato dal miner di questo blocco per raggiungere il suo obiettivo di difficoltà (difficulty target). Chiunque può ricalcolare la Proof of Work e verificare che il nonce sia valido.
    \item {\tt miner\_tx} (linee 17-40) - La transazione del miner del blocco.
    \item {\tt version} (linea 18) - Formato/era della versione della transazione; `2' corrisponde a RingCT.
    \item {\tt unlock\_time} (linea 19) - Gli output della transazione del miner non possono essere spesi fino al blocco 1582256\nth, ovvero dopo l'estrazione di 59 ulteriori blocchi (corrisponde ad un tempo di blocco pari a 60, dato che non possono essere spesi prima del passare di un'intervallo di tempo pari a 60 blocchi, ovvero $2*60 = 120$ minuti).
    \item {\tt vin} (linee 20-25) - Input della transazione del miner. Ovviamente non ci sono, dato che le transazioni dei miner sono usate per generare le ricompense dei blocchi e raccogliere le commissioni di rete.
    \item {\tt gen} (linea 21) - Abbreviazione di `generate'.
    \item {\tt height} (linea 22) - L'altezza della blockchain da cui si è generata la ricompensa al miner di questo blocco. Ogni blocco può generare una sola ricompensa e una sola volta.
    \item {\tt vout} (linee 26-32) - Output della transazione del miner.
    \item {\tt amount} (linea 27) - Ammontare distribuito dal miner della transazione, contenente la ricompensa del blocco e le commissioni derivanti dalle transazioni registrate, in unità atomiche.
    \item {\tt key} (linea 29) - Indirizzi one-time che conferiscono la proprietà della ricompensa generata al miner.
    \item {\tt extra} (linee 33-36) - Informazioni extra sulla transazione del miner, inclusa la chiave pubblica della transazione.
    \item {\tt type} (linea 38) - Tipologia della transazione, in questo caso `0' per {\tt RCTTypeNull}, indicando una transazione miner.
    \item {\tt tx\_hashes} (linee 41-46) - Tutti gli ID delle transazioni incluse in questo blocco, eccetto l'ID della transazione del miner, ovvero:\\ {\tt 06fb3e1cf889bb972774a8535208d98db164394ef2b14ecfe74814170557e6e9}
\end{itemize}




\chapter{Blocco Genesi}
\label{appendix:genesis-block}

In questo appendice viene illustrata la struttura del blocco genesi di Monero. Il blocco non ha alcuna transazione (ed invia semplicemente la ricompensa a thankful\_for\_today \cite{bitmonero-launched}). Il fondatore di Monero non ha aggiunto un timestamp, forse come retaggio di Bytecoin, la moneta da cui il codice di Monero deriva, i cui creatori avrebbero apparentemente cercato di nascondere un'ampia quantità di monete pre-estratte \cite{monero-history} e che potrebbero essere coinvolti nella gestione di una rete poco trasparente di software e servizi legati alle criptovalute \cite{bytecoin-network}.

Il blocco 1 è stato aggiunto alla blockchain al timestamp 2014-04-18 10:49:53 UTC (come riportato dal miner del blocco), di conseguenza possiamo assumere che il blocco genesi è stato creato nello stesso giorno, e ciò corrisponde con la data di lancio annunciata da thankful\_for\_today \cite{bitmonero-launched}.

\begin{Verbatim}[commandchars=\\\{\}, numbers=left]
print_block 0
timestamp: 0
previous hash: 0000000000000000000000000000000000000000000000000000000000000000
nonce: 10000
is orphan: 0
height: 0
depth: 1580975
hash: 418015bb9ae982a1975da7d79277c2705727a56894ba0fb246adaabb1f4632e3
difficulty: 1
reward: 17592186044415
\{
  "major_version": 1,
  "minor_version": 0,
  "timestamp": 0,
  "prev_id": "0000000000000000000000000000000000000000000000000000000000000000",
  "nonce": 10000,
  "miner_tx": \{
    "version": 1,
    "unlock_time": 60,
    "vin": [ \{
        "gen": \{
          "height": 0
        \}
      \}
    ],
    "vout": [ \{
        "amount": 17592186044415,
        "target": \{
          "key": "9b2e4c0281c0b02e7c53291a94d1d0cbff8883f8024f5142ee494ffbbd088071"
        \}
      \}
    ],
    "extra": [ 1, 119, 103, 170, 252, 222, 155, 224, 13, 207, 208, 152, 113, 94, 188, 
    247, 244, 16, 218, 235, 197, 130, 253, 166, 157, 36, 162, 142, 157, 11, 200, 144, 
    209
    ],
    "signatures": [ ]
  \},
  "tx_hashes": [ ]
\}
\end{Verbatim}



\section*{Campi del Blocco Genesi}

Dal momento in cui il comando per stampare a schermo il blocco genesi e il blocco dell'Appendice \ref{appendix:block-content} è lo stesso, la struttura apparirà molto simile. Teniamo dunque conto delle differenze:

\begin{itemize}
	\item {\tt difficulty} (linea 9) - La difficoltà del blocco genesi è indicata con un valore pari a 1, e ciò significa che qualsiasi {\tt nonce} potrebbe funzionare.
	\item {\tt timestamp} (linea 14) - Il blocco genesi non ha un timestamp significativo.
	\item {\tt prev\_id} (linea 15) - Per convenzione vengono utilizzati 32 bytes vuoti per l'ID del blocco precedente.
	\item {\tt nonce} (linea 16) - $n = 10000$ per convenzione.
	\item {\tt amount} (linea 27) - Corrisponde esattamente alla ricompensa del primo blocco (17.592186044415 XMR) calcolata come nella Sezione \ref{subsec:block-reward}.
	\item {\tt key} (linea 29) - I primissimi Moneroj estratti furono distribuiti al fondatore di Monero thankful\_for\_today.
	\item {\tt extra} (linee 33-36) - Utilizzo della codifica affrontata nell'Appendice \ref{appendix:RCTTypeBulletproof2}, Il campo {\tt extra} della transazione del miner del blocco genesi contiene soltanto la chiave pubblica della transazione.
	\item {\tt signatures} (linea 37) - Non ci sono firme nel blocco genesi. Questo è soltanto un artefatto del comando {\tt print\_block}. Lo stesso vale per il campo {\tt tx\_hashes} linea 39.
\end{itemize}


\end{appendices}