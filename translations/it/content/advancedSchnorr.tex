\chapter{Advanced Schnorr-like Signatures}
\label{chapter:advanced-schnorr}

A basic Schnorr signature has one signing key. As it happens, we can apply its core concepts to create a variety of progressively more complex signature schemes. One of those schemes, MLSAG, will be of central importance in Monero's transaction protocol.



%-same signing key across multiple bases
\section{Prove knowledge of a discrete logarithm across multiple bases}
\label{sec:proofs-discrete-logarithm-multiple-bases}

It is often useful to prove the same private key was used to construct public keys on different `base' keys. For example, we could have a normal public key $k G$, and a Diffie-Hellman shared secret $k R$ with some other person's public key (recall Section \ref{DH_exchange_section}), where the base keys are $G$ and $R$. As we will soon see, we can prove knowledge of the discrete log $k$ in $k G$, prove knowledge of $k$ in $k R$, {\em and} prove that $k$ is the same in both cases (all without revealing $k$).


\subsection*{Non-interactive proof}

Suppose we have a private key $k$, and $d$ base keys $\mathcal{J} = \{J_1,...,J_d\}$. The corresponding public keys are $\mathcal{K} = \{K_1,...,K_d\}$. We make a Schnorr-like proof (recall Section \ref{sec:schnorr-fiat-shamir}) across all bases.\footnote{While we say `proof', it can be trivially made a signature by including a message $\mathfrak{m}$ in the challenge hash. The terminology is loosely interchangeable in this context.} Assume the existence of a hash function \(\mathcal{H}_n\) 
mapping to integers from 0 to $l-1$.\footnote{In Monero, the hash function $\mathcal{H}_n(x) = \textrm{sc\textunderscore reduce32}(\mathit{Keccak}(x))$ where $\mathit{Keccak}$ is the basis of SHA3 and sc\textunderscore reduce32() puts the 256 bit result in the range 0 to $l-1$ (although it should really be 1 to $l-1$).} 

\begin{enumerate}
	\item Generate random number $\alpha \in_R \mathbb{Z}_l$, and compute, for all $i \in (1,...,d)$, $\alpha J_i$.
	\item Calculate the challenge,\vspace{.175cm}
	\[c = \mathcal{H}_n(\mathcal{J},\mathcal{K},[\alpha J_1],[\alpha J_2],...,[\alpha J_d])\]
	\item Define the response $r = \alpha - c*k$.
	\item Publish the signature $(c, r)$.
\end{enumerate}


\subsection*{Verification}

Assuming the verifier knows $\mathcal{J}$ and $\mathcal{K}$, he does the following.

\begin{enumerate}
	\item Calculate the challenge:\vspace{.175cm}
	\[c' = \mathcal{H}(\mathcal{J},\mathcal{K},[r J_1 + c*K_1],[r J_2 + c*K_2],...,[r J_d + c*K_d])\]
	\item If $c = c'$ then the signer must know the discrete logarithm across all bases, and it's the same discrete logarithm in each case (as always, except with negligible probability).
\end{enumerate}


\subsection*{Why it works}

If instead of $d$ base keys there was just one, this proof would clearly be the same as our original Schnorr proof (Section \ref{sec:schnorr-fiat-shamir}). We can imagine each base key in isolation to see that the multi-base proof is just a bunch of Schnorr proofs connected together. Moreover, by using only one challenge and response for all of those proofs, they must have the same discrete logarithm $k$. To get a single response that works for multiple keys the challenge would need to be known before defining an $\alpha$ for each key, but $c$ is a function of $\alpha$!



%-multiple signing keys on their own unique bases (or they can be the same e.g. G)
\section{Multiple private keys in one proof}
\label{sec:multiple_private_keys_in_one_proof}

Much like a multi-base proof, we can combine many Schnorr proofs that use different private keys. Doing so proves we know all the private keys for a set of public keys, and reduces storage requirements by making just one challenge for all proofs.


\subsection*{Non-interactive proof}

Suppose we have $d$ private keys $k_1,...,k_d$, and base keys $\mathcal{J} = \{J_1,...,J_d\}$.\footnote{There is no reason $\mathcal{J}$ can't contain duplicate base keys here, or for all base keys to be the same (e.g. $G$). Duplicates would be redundant for multi-base proofs, but now we are dealing with different private keys.} The corresponding public keys are $\mathcal{K} = \{K_1,...,K_d\}$. We make a Schnorr-like proof for all keys simultaneously.

\begin{enumerate}
	\item Generate random numbers $\alpha_i \in_R \mathbb{Z}_l$ for all $i \in (1,...,d)$, and compute all $\alpha_i J_i$.
	\item Calculate the challenge,\vspace{.175cm}
	\[c = \mathcal{H}_n(\mathcal{J},\mathcal{K},[\alpha_1 J_1],[\alpha_2 J_2],...,[\alpha_d J_d])\]
	\item Define each response $r_i = \alpha_i - c*k_i$.
	\item Publish the signature $(c, r_1,...,r_d)$.
\end{enumerate}


\subsection*{Verification}

Assuming the verifier knows $\mathcal{J}$ and $\mathcal{K}$, he does the following.

\begin{enumerate}
	\item Calculate the challenge:\vspace{.175cm}
	\[c' = \mathcal{H}(\mathcal{J},\mathcal{K},[r_1 J_1 + c*K_1],[r_2 J_2 + c*K_2],...,[r_d J_d + c*K_d])\]
	\item If $c = c'$ then the signer must know the private keys for all public keys in $\mathcal{K}$ (except with negligible probability).
\end{enumerate}



\section{Spontaneous Anonymous Group (SAG) signatures}
\label{SAG_section}

Group signatures are a way of proving a signer belongs to a group, without necessarily identifying him. Originally (Chaum in \cite{Chaum:1991:GS:1754868.1754897}), group signature schemes required the system be set up, and in some cases managed, by a trusted person in order to prevent illegitimate signatures, and, in a few schemes, adjudicate disputes. These relied on a {\em group secret} which is not desirable since it creates a disclosure risk that could undermine anonymity. Moreover, requiring coordination between group members (i.e. for setup and management) is not scalable beyond small groups or inside companies.

Liu {\em et al.} presented a more interesting scheme in \cite{Liu2004} building on the work of Rivest {\em et al.} in \cite{rivest-leak-secret}. The authors detailed a group signature algorithm called LSAG characterized by three properties: {\em anonymity, linkability,} and {\em spontaneity}. Here we discuss SAG, the non-linkable version of LSAG, for conceptual clarity. We reserve the idea of linkability for later sections.
\\

Schemes with anonymity and spontaneity are typically referred to as `ring signatures'. In the context of Monero they will ultimately allow for unforgeable, signer-ambiguous transactions that leave currency flows largely untraceable.


\subsection*{Signature}

Ring signatures are composed of a ring and a signature. Each {\em ring} is a set of public keys, one of which belongs to the signer and the rest of which are unrelated. The {\em signature} is generated with that ring of keys, and anyone verifying it would not be able to tell which ring member was the actual signer.

Our Schnorr-like signature scheme in Section \ref{sec:signing-messages} can be considered a one-key ring signature. We get to two keys by, instead of defining $r$ right away, generating a decoy $r'$ and creating a new challenge to define $r$ with.
\\

Let \(\mathfrak{m}\) be the message to sign, \(\mathcal{R} = \{K_1, K_2, ..., K_n\}\) a set of distinct public keys (a group/ring), and \(k_\pi\) the signer's private key corresponding to his public key \(K_\pi \in \mathcal{R}\), where $\pi$ is a secret index.

\begin{enumerate}
	\item Generate random number \(\alpha \in_R \mathbb{Z}_l\) and fake responses  \(r_i \in_R \mathbb{Z}_l\) for \(i \in \{1, 2, ..., n\}\) but excluding \(i = \pi\).

	\item Calculate
	\[c_{\pi+1} = \mathcal{H}_n(\mathcal{R}, \mathfrak{m}, [\alpha G])\]

	\item For \(i = \pi+1, \pi+2, ..., n, 1, 2, ..., \pi-1\) calculate, replacing \(n + 1 \rightarrow 1\),\vspace{.175cm}
	\[  c_{i+1} = \mathcal{H}_n(\mathcal{R}, \mathfrak{m}, [r_i G + c_i K_i])\] 

	\item Define the real response $r_\pi$ such that \(\alpha = r_\pi + c_\pi k_\pi \pmod l\).
\end{enumerate}

The ring signature contains the signature \(\sigma(\mathfrak{m}) = (c_1, r_1, ..., r_n) \), and the ring $\mathcal{R}$.


\subsection*{Verification}

Verification means proving $\sigma(\mathfrak{m})$ is a valid signature created by a private key corresponding to a public key in $\mathcal{R}$ (without necessarily knowing which one), and is done in the following manner:

\begin{enumerate}
	\item For \(i = 1, 2, ..., n\) iteratively compute, replacing \(n + 1 \rightarrow 1\),\vspace{.175cm}
	\begin{align*}
	c'_{i+1}   = \mathcal{H}_n(\mathcal{R}, \mathfrak{m}, [r_i G + c_i {K_i}])
	\end{align*}

	\item If \(c_1 = c'_1\) then the signature is valid. Note that $c'_1$ is the last term calculated.
\end{enumerate}

In this scheme we store (1+$n$) integers and use $n$ public keys.


\subsection*{Why it works}

We can informally convince ourselves the algorithm works by going through an example. Consider ring $R = \{K_1, K_2, K_3\}$ with $k_\pi = k_2$. First the signature:
\begin{enumerate}
    \item Generate random numbers: $\alpha$, $r_1$, $r_3$
\begin{align*}
    \intertext{\item Seed the signature loop:}	c_3 &= \mathcal{H}_n(\mathcal{R}, \mathfrak{m}, [\alpha G])
    \intertext{\item Iterate: \vspace{-.2cm}}
        c_1 &= \mathcal{H}_n(\mathcal{R}, \mathfrak{m}, [r_3 G + c_3 K_3])\\
        c_2 &= \mathcal{H}_n(\mathcal{R}, \mathfrak{m}, [r_1 G + c_1 K_1])
\end{align*}
    \item Close the loop by responding: $r_2 = \alpha - c_2 k_2 \pmod{l}$
\end{enumerate}

We can substitute $\alpha$ into $c_3$ to see where the word ‘ring' comes from:\vspace{.175cm}
\begin{alignat*}{3}
    c_3 &= \mathcal{H}_n(\mathcal{R}, \mathfrak{m}, [(r_2 + c_2 k_2) G &&])\\
    c_3 &= \mathcal{H}_n(\mathcal{R}, \mathfrak{m}, [r_2 G + c_2 K_2 &&])
\end{alignat*}\vspace{.05cm}

Then verification using $\mathcal{R}$, and $\sigma(\mathfrak{m}) = (c_1, r_1, r_2, r_3)$:
\begin{enumerate}
    \item We use $r_1$ and $c_1$ to compute\vspace{.175cm}
    \begin{align*}
c'_2 &= \mathcal{H}_n(\mathcal{R}, \mathfrak{m}, [r_1 G + c_1 K_1])
    \intertext{\item From when we made the signature, we see $c'_2 = c_2$. With $r_2$ and $c'_2$ we compute\vspace{.175cm}}
c'_3 &= \mathcal{H}_n(\mathcal{R}, \mathfrak{m}, [r_2 G + c'_2 K_2])
    \intertext{\item We can easily see that $c'_3 = c_3$ by substituting $c_2$ for $c'_2$. Using $r_3$ and $c'_3$ we get\vspace{.175cm}}
c'_1 &= \mathcal{H}_n(\mathcal{R}, \mathfrak{m}, [r_3 G + c'_3 K_3])
    \end{align*}
\end{enumerate}
\quad No surprises here: $c'_1 = c_1$ if we substitute $c_3$ for $c'_3$.\vspace{-.3cm}



\section{Back's Linkable Spontaneous Anonymous Group (bLSAG) signatures}
\label{blsag_note}

The ring signature schemes discussed here on out display several properties that will be useful for producing confidential transactions.\footnote{Keep in mind that all robust signature schemes have security models which contain various properties. The properties mentioned here are perhaps most relevant to understanding the purpose of Monero's ring signatures, but are not a comprehensive overview of linkable ring signature properties.} Note that both `signer ambiguity' and `unforgeability' also apply to SAG signatures.

\begin{description}
	\item[Signer Ambiguity]
	An observer should be able to determine the signer must be a member of the ring (except with negligible probability), but not which member.\footnote{\label{anonymity_note}Anonymity for an action is usually in terms of an `anonymity set’, which is `all the people who could have possibly taken that action’. The largest anonymity set is `humanity’, and for Monero it is the ring size, or e.g. the so-called `mixin level' $v$ plus the real signer. Mixin refers to how many fake members each ring signature has. If the mixin is $v$ = 4 then there are 5 possible signers. Expanding anonymity sets makes it progressively harder to track down real actors.} Monero uses this to obfuscate the origin of funds in each transaction.

	\item[Linkability]
	If a private key is used to sign two different messages then the messages will become linked.\footnote{\label{linkability_note}The linkability property does not apply to non-signing public keys. That is, a ring member whose public key has been used in different ring signatures will not cause linkage.} As we will show, this property is used to prevent double-spending attacks in Monero (except with negligible probability).

	\item[Unforgeability]
    No attacker can forge a signature except with negligible probability.\footnote{\label{unforgeability_note}Certain ring signature schemes, including the one in Monero, are strong against adaptive chosen-message and adaptive chosen-public-key attacks. An attacker who can obtain legitimate signatures for chosen messages and corresponding to specific public keys in rings of his choice cannot discover how to forge the signature of even one message. This is called {\em existential unforgeability}; see \cite{MRL-0005-ringct} and \cite{Liu2004}.} This is used to prevent theft of Monero funds by those not in possession of the appropriate private keys.
\end{description}

In the LSAG signature scheme \cite{Liu2004}, the owner of a private key could produce one anonymous unlinked signature per ring.\footnote{\label{lsag_linkability_note}In the LSAG scheme linkability only applies to signatures using rings with the same members and in the same order, the `exact same ring.’ It is really ``one anonymous signature per ring member per ring.” Signatures can be linked even if made for different messages.} In this section we present an enhanced version of the LSAG algorithm where linkability is independent of the ring’s decoy members.\footnote{LSAG was discussed in the first edition of this report. \cite{ztm-1}}

The modification was unraveled in \cite{MRL-0005-ringct} based on a publication by Adam Back \cite{AdamBack-ring-efficiency} regarding the CryptoNote \cite{cryptoNoteWhitePaper} ring signature algorithm (previously used in Monero, and now deprecated; see Section \ref{subsec:proofs-input-creation-spendproof}), which was in turn inspired by Fujisaki and Suzuki's work in \cite{Fujisaki2007}.


\subsection*{Signature}

As with SAG, let \(\mathfrak{m}\) be the message to sign, \(\mathcal{R} = \{K_1, K_2, ..., K_n\}\) a set of distinct public keys, and \(k_\pi\) the signer's private key corresponding to his public key \(K_\pi \in \mathcal{R}\), where $\pi$ is a secret index. Assume the existence of a hash function \(\mathcal{H}_p\), which maps to curve points in EC.\footnote{It doesn’t matter if points from $\mathcal{H}_p$ are compressed or not. They can always be decompressed.}\footnote{Monero uses a hash function\marginnote{src/ringct/ rctOps.cpp {\tt hash\_to\_p3()}} that returns curve points directly, rather than computing some integer that is then multiplied by $G$. $\mathcal{H}_p$ would be broken if someone discovered a way to find $n_x$ such that $n_x G = \mathcal{H}_p(x)$. See a description of the algorithm in \cite{hashtopoint-writeup}. According to the CryptoNote whitepaper \cite{cryptoNoteWhitePaper} its origin was this paper: \cite{hashtopoint-original-paper}.}

\begin{enumerate}
	\item Calculate key image \(\tilde{K} = k_\pi \mathcal{H}_p(K_\pi)\).\footnote{In Monero it's important to use the hash to point function for key images instead of another base point so linearity doesn't lead to linking signatures created by the same address (even if for different one-time addresses). See \cite{cryptoNoteWhitePaper} page 18.}

	\item Generate random number \(\alpha \in_R \mathbb{Z}_l\) and random numbers \(r_i \in_R \mathbb{Z}_l\) for \(i \in \{1, 2, ..., n\}\) but excluding \(i = \pi\).

	\item Compute
	\[c_{\pi+1} = \mathcal{H}_n(\mathfrak{m}, [\alpha G], [\alpha \mathcal{H}_p(K_\pi)])\]

	\item For \(i = \pi+1, \pi+2, ..., n, 1, 2, ..., \pi-1\) calculate, replacing \(n + 1 \rightarrow 1\),\vspace{.175cm}
	\[c_{i+1} = \mathcal{H}_n(\mathfrak{m}, [r_i G + c_i K_i], [r_i \mathcal{H}_p(K_i) + c_i \tilde{K}])\]

	\item Define \(r_\pi = \alpha - c_\pi k_\pi \pmod l\).
\end{enumerate}

The signature will be \(\sigma(\mathfrak{m}) = (c_1, r_1, ..., r_n)\), with key image $\tilde{K}$ and ring $\mathcal{R}$.


\subsection*{Verification}

Verification means proving $\sigma(\mathfrak{m})$ is a valid signature created by a private key corresponding to a public key in $\mathcal{R}$, and is done in the following manner:

\begin{enumerate}
    \item Check $l \tilde{K} \stackrel{?}{=} 0$.
	\item For \(i = 1, 2, ..., n\) iteratively compute, replacing \(n + 1 \rightarrow 1\),\vspace{.175cm}
	\begin{align*}
	c'_{i+1} = \mathcal{H}_n(\mathfrak{m}, [r_i G + c_i {K_i}], [r_i \mathcal{H}_p(K_i) + c_i \tilde{K}])
	\end{align*}

	\item If \(c_1 = c'_1\) then the signature is valid.
\end{enumerate}

In this scheme we store (1+$n$) integers, have one EC key image, and use $n$ public keys.

We\marginnote{src/crypto- note\_core/ cryptonote\_ core.cpp {\tt check\_tx\_ inputs\_key- images\_do- main()}} must check $l \tilde{K} \stackrel{?}{=} 0$ because it is possible to add an EC point from the subgroup of size $h$ (the cofactor) to $\tilde{K}$ and, if all $c_i$ are multiples of $h$ (which we could achieve with automated trial and error using different $\alpha$ and $r_i$ values), make $h$ unlinked valid signatures using the same ring and signing key.\footnote{We are not concerned with points from other subgroups because the output of $\mathcal{H}_n$ is confined to $\mathbb{Z}_l$. For EC order $N = h l$, all divisors of $N$ (and hence, possible subgroups) are either multiples of $l$ (a prime) or divisors of $h$.} This is because an EC point multiplied by its subgroup's order is zero.\footnote{In Monero's early history this was not checked for. Fortunately, it was not exploited before a fix was implemented in April 2017 (v5 of the protocol) \cite{key-image-bug}.}

To be clear, given some point $K$ in the subgroup of order $l$, some point $K^h$ with order $h$, and an integer $c$ divisible by $h$:
\begin{align*}
    c*(K + K^h) &= cK + cK^h\\
                &= cK + 0
\end{align*}

We can demonstrate correctness (i.e. `how it works') in a similar way to the more simple SAG signature scheme.

Our description attempts to be faithful to the original explanation of bLSAG, which does not include $\mathcal{R}$ in the hash that calculates $c_i$. Including keys in the hash is known as `key prefixing'. Recent research \cite{key-prefix-paper} suggests it may not be necessary, although adding the prefix is standard practice for similar signature schemes (LSAG uses key prefixing).


\subsection*{Linkability}

Given two valid signatures that are different in some way (e.g. different fake responses, different messages, different overall ring members),\vspace{.1cm}
\begin{align*}
	\sigma(\mathfrak{m})   &= (c_1, r_1, ..., r_n)\textrm{ with } \tilde{K}\textrm{, and}\\
	\sigma'(\mathfrak{m}')  &= (c_1', r'_1, ..., r'_{n'})\textrm{ with } \tilde{K}'\textrm{,}
\end{align*}
\quad if \(\tilde{K} =  \tilde{K}'\) then clearly both signatures come from the same private key.% because $\tilde{K}= k_{\pi} \mathcal{H}_p(\mathcal{R})$.

While an observer could link $\sigma$ and $\sigma'$, he wouldn’t necessarily know which $K_i$ in $\mathcal{R}$ or $\mathcal{R}'$ was the culprit unless there was only one common key between them. If there was more than one common ring member, his only recourse would be solving the DLP or auditing the rings in some way (such as learning all $k_i$ with $i \neq \pi$, or learning $k_\pi$).\footnote{\label{lsag_unforgeable_note}LSAG, which is quite similar to bLSAG, is unforgeable, meaning no attacker could make a valid ring signature without knowing a private key. If he invents a fake $\tilde{K}$ and seeds his signature computation with $c_{\pi+1}$, then, not knowing $k_\pi$, he can’t calculate a number $r_\pi = \alpha - c_\pi k_\pi$ that would produce $[r_\pi G + c_\pi K_\pi] = \alpha G$. A verifier would reject his signature. Liu {\em et al.} prove forgeries that manage to pass verification are extremely improbable \cite{Liu2004}.}



\section{Multilayer Linkable Spontaneous Anonymous Group (MLSAG) signatures}
\label{sec:MLSAG}

In order to sign transactions, one has to sign with multiple private keys. In \cite{MRL-0005-ringct}, Shen Noether {\em et al.} describe a multi-layered generalization of the bLSAG signature scheme applicable when we have a set of \(n \cdot m\) keys; that is, the set\vspace{.175cm}
\[\mathcal{R} = \{K_{i,j}\}  \quad \textrm{for} \quad  i \in \{1, 2, ..., n\} \quad \textrm{and} \quad j \in \{1, 2, ..., m\}\]

where we know the $m$ private keys \(\{k_{\pi, j}\}\) corresponding to the subset \(\{K_{\pi, j}\}\) for some index \(i = \pi\). Such an algorithm would address our needs if we generalize the notion of linkability.
\begin{description}
	\item[Linkability] If any private key \(k_{\pi, j}\) is used in 2 different signatures, then those signatures will be automatically linked.
\end{description}


\subsection*{Signature}

\begin{enumerate}
	\item Calculate key images \(\tilde{K_j} = k_{\pi, j} \mathcal{H}_p(K_{\pi, j})\) for all \(j \in \{1, 2, ..., m\}\).

	\item Generate random numbers  \(\alpha_j \in_R \mathbb{Z}_l\), and \(r_{i, j} \in_R \mathbb{Z}_l\) for \(i \in \{1, 2, ..., n\}\) (except \(i = \pi\)) and \(j \in \{1, 2, ..., m\}\).

	\item Compute\footnote{Monero\marginnote{src/ringct/ rctSigs.cpp {\tt MLSAG\_Gen()}} MLSAG uses key prefixing. Each challenge contains explicit public keys like this (adding the $K$ terms absent from bLSAG; key images are included in the message signed):\vspace{-.25cm}
	\[c_{\pi+1} = \mathcal{H}_n(\mathfrak{m}, K_{\pi, 1}, [\alpha_1 G], [\alpha_1 \mathcal{H}_p(K_{\pi, 1})], ..., K_{\pi, m}, [\alpha_m G], [\alpha_m \mathcal{H}_p(K_{\pi, m})])
	\]}
	\[c_{\pi+1} = \mathcal{H}_n(\mathfrak{m}, [\alpha_1 G], [\alpha_1 \mathcal{H}_p(K_{\pi, 1})], ..., [\alpha_m G], [\alpha_m \mathcal{H}_p(K_{\pi, m})])\]

	\item For \(i = \pi+1, \pi+2, ..., n, 1, 2, ..., \pi-1\) calculate, replacing \(n + 1 \rightarrow 1\),\vspace{.175cm}
	\[ c_{i+1} = \mathcal{H}_n(\mathfrak{m}, [r_{i, 1} G + c_i K_{i, 1}], [r_{i, 1} \mathcal{H}_p(K_{i, 1}) + c_i \tilde{K}_1], 
	..., [r_{i, m} G + c_i K_{i, m}], [r_{i, m} \mathcal{H}_p(K_{i, m}) + c_i \tilde{K}_m])\]

	\item Define all \(r_{\pi, j} = \alpha_j - c_\pi k_{\pi, j} \pmod l\).
\end{enumerate}

The signature will be \(\sigma(\mathfrak{m}) = (c_1, r_{1, 1}, ..., r_{1, m}, ..., r_{n, 1}, ..., r_{n, m}) \), with key images $(\tilde{K}_1, ...,  \tilde{K}_m)$.

%One way to think about MLSAG is that there are $m$ sub-loops of size $n$, and in each sub-loop we know a private key at index $i = \pi$ ($m \cdot n$ total public keys). The signature algorithm ties together a ‘stack’ of keys at each stage $c$, composed of one key from each sub-loop. bLSAG is the special case where $m = 1$.


\subsection*{Verification}

Verification of a signature is done in the following manner:

\begin{enumerate}
    \item For all $j \in \{1,...,m\}$ check $l \tilde{K}_j \stackrel{?}{=} 0$.
	\item For \(i = 1, ..., n\) compute, replacing \(n + 1 \rightarrow 1\),\vspace{.175cm}
	\begin{align*}
	c'_{i+1} = \mathcal{H}_n(\mathfrak{m}, [r_{i, 1} G + c_i K_{i, 1}], [r_{i, 1} \mathcal{H}_p(K_{i, 1}) + c_i \tilde{K}_1], 
	..., [r_{i, m} G + c_i K_{i, m}], [r_{i, m} \mathcal{H}_p(K_{i, m}) + c_i \tilde{K}_m])
	\end{align*}

	\item If \(c_1 = c'_1\) then the signature is valid.
\end{enumerate}


\subsection*{Why it works}

Just as with the SAG algorithm, we can readily observe that

\begin{itemize}
    \item If \(i \ne \pi \), then clearly the values \(c'_{i + 1}\) are calculated as described in the signature algorithm.

    \item If \(i = \pi\) then, since \(r_{\pi, j} = \alpha_j - c_\pi k_{\pi, j} \) closes the loop,\vspace{.175cm}
    \begin{alignat*}{6}
        r_{\pi, j} G + c_\pi K_{\pi,j} &= (\alpha_j - c_\pi k_{\pi, j}) G + c_\pi K_{\pi,j} = \alpha_j G\\
        \intertext{and}
        r_{\pi, j} \mathcal{H}_p(K_{\pi, j}) + c_\pi \tilde{K}_j &= (\alpha_j - c_\pi k_{\pi, j}) \mathcal{H}_p(K_{\pi, j}) + c_\pi \tilde{K}_j = \alpha_j \mathcal{H}_p(K_{\pi, j})\\
    \end{alignat*}
    In other words, it holds also that \(c'_{\pi + 1} = c_{\pi+1}\).
\end{itemize}


\subsection*{Linkability}

If a private key \(k_{\pi, j}\) is re-used to make any signature, the corresponding key image \(\tilde{K}_j\) supplied in the signature will reveal it. This observation matches our generalized definition of linkability.\footnote{As with bLSAG, linked MLSAG signatures do not indicate which public key was used to sign it. However, if the linking key image's sub-loops' rings have only one key in common, the culprit is obvious. If the culprit is identified, all other signing members of both signatures are revealed since they share the culprit's indices.}


\subsection*{Space requirements}

In this scheme we store (1+$m*n$) integers, have $m$ EC key images, and use $m*n$ public keys.



\section{Concise Linkable Spontaneous Anonymous Group (CLSAG) signatures}
\label{sec:CLSAG}

CLSAG \cite{MRL-0011-CLSAG}\footnote{The paper this section is based on is a pre-print being finalized for external review. CLSAG is promising as a replacement for MLSAG in future protocol versions, but has not been implemented, and might not be in the future.} is sort of half-way between bLSAG and MLSAG. Suppose you have a `primary' key, and associated with it are several `auxiliary' keys. It is important to prove knowledge of all private keys, but linkability only applies to the primary. This linkability retraction allows smaller, faster signatures than afforded by MLSAG.

As with MLSAG, we have a set of \(n \cdot m\) keys ($n$ is the ring size, $m$ is the number of signing keys), and the primary keys are at index 1. In other words, there are $n$ primary keys, and the $\pi$\nth such key and its auxiliaries will sign.\vspace{.175cm}
\[\mathcal{R} = \{K_{i,j}\}  \quad \textrm{for} \quad  i \in \{1, 2, ..., n\} \quad \textrm{and} \quad j \in \{1, 2, ..., m\}\]

We know the private keys \(\{k_{\pi, j}\}\) corresponding to the subset \(\{K_{\pi, j}\}\) for some index \(i = \pi\).


\subsection*{Signature}

\begin{enumerate}
	\item Calculate key images \(\tilde{K_j} = k_{\pi, j} \mathcal{H}_p(K_{\pi, 1})\) for all \(j \in \{1, 2, ..., m\}\). Note the base key is always the same, and so key images with $j>1$ are `auxiliary key images'. For notational simplicity we call them all $\tilde{K}_j$.

	\item Generate random numbers  \(\alpha \in_R \mathbb{Z}_l\), and \(r_{i} \in_R \mathbb{Z}_l\) for \(i \in \{1, 2, ..., n\}\) (except \(i = \pi\)).

    \item Calculate aggregate public keys $W_i$ for \(i \in \{1, 2, ..., n\}\), and aggregate key image $\tilde{W}$\footnote{The CLSAG paper says to use different hash functions for domain separation, which we model by prefixing each hash with a tag \cite{MRL-0011-CLSAG}, e.g. $T_1 =$ ``CLSAG\_1", $T_c =$ ``CLSAG\_c", etc. Domain separated hash functions have different outputs even with the same inputs. We also use key prefixing here (including $\mathcal{R}$, which has all the keys, in the hash). Domain separating is a new policy for Monero development, and will likely be done with all applications of hash functions added in the future (v13+). Historical uses of hash functions will probably be left alone.}%an actual implementation may use $j$ flags, instead of the index itself
    \begin{align*}
    W_i &= \sum^{m}_{j=1} \mathcal{H}_n(T_j, \mathcal{R}, \tilde{K}_1,...,\tilde{K}_{m})*K_{i,j}\\
    \tilde{W} &= \sum^{m}_{j=1} \mathcal{H}_n(T_j, \mathcal{R}, \tilde{K}_1,...,\tilde{K}_{m})*\tilde{K}_j
    \end{align*}{}
    where $w_{\pi} = \sum_j \mathcal{H}_n(T_j,...)*k_{\pi,j}$ is the aggregate private key.

	\item Compute
	\[c_{\pi+1} = \mathcal{H}_n(T_c, \mathcal{R}, \mathfrak{m}, [\alpha G], [\alpha \mathcal{H}_p(K_{\pi, 1})])\]

	\item For \(i = \pi+1, \pi+2, ..., n, 1, 2, ..., \pi-1\) calculate, replacing \(n + 1 \rightarrow 1\),\vspace{.175cm}
	\[c_{i+1} = \mathcal{H}_n(T_c, \mathcal{R}, \mathfrak{m}, [r_i G + c_i W_i], [r_{i} \mathcal{H}_p(K_{i,1}) + c_i \tilde{W}])\]

	\item Define \(r_{\pi} = \alpha - c_\pi w_\pi \pmod l\).
\end{enumerate}

Therefore \(\sigma(\mathfrak{m}) = (c_1, r_1, ..., r_n) \), with primary key image $\tilde{K}_1$, and auxiliary images $(\tilde{K}_2,...,\tilde{K}_{m})$.


\subsection*{Verification}

The verification of a signature is done in the following manner:

\begin{enumerate}
    \item For all $j \in \{1,...,m\}$ check $l \tilde{K}_j \stackrel{?}{=} 0$.\footnote{In Monero we would only check $l*\tilde{K}_1 \stackrel{?}{=} 0$ for the primary key image. Auxiliary keys would be stored as $(1/8)*\tilde{K}_j$, and during verification multiplied by 8 (recall Section \ref{elliptic_curves_section}), which is more efficient. The method discrepancy is an implementation choice, since linkable key images are very important and so shouldn't be messed with aggressively, and the other method was employed in prior protocol versions.}

    \item Calculate aggregate public keys $W_i$ for \(i \in \{1, 2, ..., n\}\), and aggregate key image $\tilde{W}$\vspace{.175cm}
    \begin{align*}
    W_i &= \sum^{m}_{j=1} \mathcal{H}_n(T_j, \mathcal{R}, \tilde{K}_1,...,\tilde{K}_{m})*K_{i,j}\\
    \tilde{W} &= \sum^{m}_{j=1} \mathcal{H}_n(T_j, \mathcal{R}, \tilde{K}_1,...,\tilde{K}_{m})*\tilde{K}_j
    \end{align*}{}

	\item For \(i = 1, ..., n\) compute, replacing \(n + 1 \rightarrow 1\),\vspace{.175cm}
	\[c_{i+1} = \mathcal{H}_n(T_c, \mathcal{R}, \mathfrak{m}, [r_i G + c_i W_i], [r_{i} \mathcal{H}_p(K_{i,1}) + c_i \tilde{W}])\]

	\item If \(c_1 = c'_1\) then the signature is valid.
\end{enumerate}


\subsection*{Why it works}

The biggest danger in concise signatures like this is key cancellation, where the key images reported aren't legitimate, yet still sum to a legitimate aggregate value. This is where the aggregation coefficients $\mathcal{H}_n(T_j, \mathcal{R}, \tilde{K}_1,...,\tilde{K}_{m})$ come into play, locking down each key to its expected value. We leave tracing out the circular repercussions of faking a key image as an exercise to the reader (perhaps start by imagining those coefficients don't exist). Auxiliary key images are an artifact of proving the primary image is legitimate, since the aggregate private key $w_{\pi}$, which contains all the private keys, is applied to base point $\mathcal{H}_p(K_{\pi,1})$.


\subsection*{Linkability}

If a private key \(k_{\pi, 1}\) is re-used to make any signature, the corresponding primary key image \(\tilde{K}_1\) supplied in the signature will reveal it. Auxiliary key images are ignored, as they only exist to facilitate the `Concise' part of CLSAG.


\subsection*{Space requirements}

We store (1+$n$) integers, have $m$ key images, and use $m*n$ public keys.

%NOTE: in Monero CLSAG key prefixing is a bit different.
%\[c_{i+1} = \mathcal{H}_n(T_c, one-time addresses, output commitments, pseudo output commitment, \mathfrak{m}, [r_i G + c_i W_i], [r_{i} \mathcal{H}_p(K_{i,1}) + c_i \tilde{W}])\]