% this file is called up by main.tex
% content in this file will be fed into the main document

% ---------------------------------------------------------------------------

Criptografía. Puede parecer que solo los matemáticos e informáticos tienen acceso a este tema misterioso, esotérico, poderoso y elegante. De hecho, muchos tipos de criptografía son lo suficientemente sencillos como para que cualquiera pueda aprender sus conceptos fundamentales.
\\ \newline
Casi todo el mundo sabe que la criptografía se usa para proteger las comunicaciones, ya sean cartas codificadas o interacciones digitales privadas. También se aplica en lo que se conoce como criptomonedas. Estas monedas digitales utilizan la criptografía para asignar y transferir la propiedad de los fondos. Para garantizar que ninguna unidad de dinero pueda duplicarse o crearse a voluntad, las criptomonedas suelen basarse en `cadenas de bloque', que son libros de contabilidad públicos y distribuidos que contienen registros de transacciones monetarias que pueden ser verificadas por terceros \cite{Nakamoto_bitcoin}.
\\ \newline
A primera vista, podría pensarse que las transacciones deben enviarse y almacenarse en formato de texto plano para que se puedan verificar públicamente. No obstante, es posible ocultar a los participantes de una transacción, así como las cantidades implicadas, utilizando herramientas criptográficas que, sin embargo, permiten que las transacciones sean verificadas y consensuadas por observadores \cite{cryptoNoteWhitePaper}. Esto se ejemplifica en la criptomoneda Monero.
\\ \newline
Aquí no solo nos esforzamos en enseñar a cualquiera que sepa álgebra básica y conceptos sencillos de informática como la `representación en bits' de un número cómo funciona Monero a un nivel detallado y exhaustivo, sino también lo útil y hermosa que puede ser la criptografía.
\\ \newline
Para nuestros lectores experimentados: Monero es una criptomoneda de cadena de bloque de gráfico acíclico distribuido unidimensional estándar \cite{Nakamoto_bitcoin} en la que las transacciones se basan en criptografía de curva elíptica usando la curva Ed25519 \cite{Bernstein2008}, las entradas de transacciones son firmadas con firmas de grupo anónimas multicapa enlazables espontáneas de tipo Schnorr \cite{MRL-0005-ringct} y las cantidades de salida (comunicadas a los destinatarios a través de ECDH \cite{Diffie-Hellman}) se ocultan con compromisos de Pedersen \cite{maxwell-ct} y se prueban en un rango legítimo con Bulletproofs \cite{Bulletproofs_paper}. En gran medida, la primera parte de este informe se dedica a explicar estas ideas.
