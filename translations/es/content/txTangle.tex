\chapter{Joint Monero Transactions (TxTangle)}
\label{chapter:txtangle}

There are a number of unavoidable transaction graph heuristics created by the nature of different entities and scenarios. In particular, the behavior of miners, pools (Section 5.1 of \cite{AnalysisOfLinkability}), escrowed marketplaces, and exchanges have clear patterns open to analysis even within Monero's ring signature-based protocol.

We describe here TxTangle, analogous to Bitcoin's CoinJoin \cite{coinjoin-wiki}, one method to confuse those heuristics.\footnote{This chapter constitutes the proposal for a joint transaction protocol. No such protocol has been implemented as of this writing. A previous proposal, named MoJoin and created by pseudonymous co-author Monero Research Lab (MRL) researcher Sarang Noether, required a trusted dealer to function. Such a dealer seems to conflict with the Monero project's basic commitment to privacy and fungibility, and hence MoJoin was not pursued further.} In essence, several transactions are squashed into one transaction, making the behavior patterns of each participant blend together. 

To accomplish that obfuscation, it must be unreasonably difficult for observers to use the information contained in a joint transaction to group inputs and outputs, and associate them with individual participants, nor know how many participants there actually were.\footnote{Since in Bitcoin amounts are clearly visible, it is often possible to group CoinJoin inputs and outputs based on amount sums. \cite{coinjoin-sudoku}} Moreover, even the participants themselves should not be aware of the number of participants, or be able to group the inputs and outputs of other participants unless they control all but one participant's grouping.\footnote{Maliciously polluting joint transactions is a potential attack on this method, first identified for CoinJoin. \cite{coinjoin-pollution}} Finally, it should be possible to construct joint transactions without relying on a central authority \cite{exa-blockchain-analysis}. Fortunately, all these requirements can be achieved in Monero.



\section{Building joint transactions}
\label{sec:building-txtangle}

In a normal transaction, inputs and outputs are tied together using the proof that amounts balance. From Section \ref{sec:commitments-and-fees}, the sum of pseudo output commitments equals the sum of output commitments (plus the fee commitment).
\[\sum_j C'^a_{j} - (\sum_t C^b_{t} + f H) = 0\]

A trivial joint transaction could take all the content of multiple transactions, and stick them in one. MLSAG messages would sign all the sub-transactions' data, and amount balancing would quite obviously work (0 + 0 + 0 = 0).\footnote{Since Bulletproofs-style range proofs are actually aggregated into one (Section \ref{sec:range_proofs}), participants would have to collaborate to some extent even in the trivial case.} However, input and output groupings could just as trivially be identified based on testing if input/output subsets have balancing amounts.\footnote{Participants could try to divide up the fee in a fancy manner to confuse observers, but it would fail in the face of brute force since fees are not that large (around 32 bits or less).}

We can easily get around this by computing shared secrets between each pair of participants, then adding these offsets to their pseudo output commitments' masks (Section \ref{sec:ringct-introduction}). In each pair, one participant adds the shared secret to one of his pseudo output commitments, and the other participant subtracts it from one of {\em his} pseudo commitments. When summed together, the secrets cancel out, and since each pair of participants has a shared secret the amount balance only appears after all commitments are combined.\footnote{We offset the pseudo output commitments instead of output commitments since output commitment masks are constructed from the recipient's address (Section \ref{sec:pedersen_monero}).}

Shared secrets may hide input/output groupings in the immediate sense, but participants must learn about all the inputs and outputs somehow, and the easiest way is if they each communicate their individual input/output groupings. Clearly this violates the initial premise, and in any case implies participants know the number of participants.


\subsection{$n$-way communication channel}
\label{subsec:n-way-channel}

The maximum number of participants to a TxTangle is either the number of outputs or inputs (whichever is lower). We model that by each real participant pretending to be a different person for each output he is sending. This is primarily for the purpose of setting up a group communication channel with other would-be participants, without revealing how many participants there are.

Imagine $n$ ($2 \leq n \leq 16$, although at least 3 is recommended)\footnote{Currently, a transaction may have at most 16 outputs.} supposedly unrelated people gather at apparently random intervals in a chatroom, scheduled to open at time $t_0$ and close at $t_1$ (only 16 people can be in a chatroom at once, and the chatroom has conditions like fee priority, base fee per byte [to simplify consensus around the current median and block reward], and range of acceptable transaction types since e.g. currently transactions from the beginning of Monero can't be directly spent in a RingCT transaction \cite{pre-ringct-outputs-like-coinbase-research-issue-59}). At $t_1$ all mock-members signal desire to proceed by publishing a public key, and the room is converted into an $n$-way communication channel by constructing a shared secret amongst all the mock-members.\footnote{The multisig method from Section \ref{sec:m-of-n} is one way, extending M-of-N all the way to 1-of-N.} This shared secret is used to encrypt message contents, while mock-members sign input-related messages using SAG signatures (Section \ref{SAG_section}) so it's never clear who sent a given message, and output-related messages with a bLSAG (Section \ref{blsag_note}) on the set of mock-member public keys so actual outputs are dissociated from mock-members.\footnote{Each separate set of TxTangle bLSAGs should use the same key images, since everything related to a given output is linked together.}% can create their key images using a different hash-to-point algorithm, or more simply by tagging the hash with a string, e.g. $\tilde{K}_t = \mathcal{H}_p(``TxTangle\_bLSAG\_2",K_t)$.}


\subsection{Message rounds to construct a joint transaction}
\label{subsec:message-rounds-txtangle}

After the channel is set up, TxTangle transactions can be constructed in five rounds of communication, where the next round may only begin after the previous is finished, and each round is given a timed communication interval within which messages should be randomly published. These intervals are intended to prevent message clusters that would reveal input/output groupings.
\begin{enumerate}
    \item Each mock-member privately generates a random scalar for each intended output, and signs them with bLSAGs. A sorted list of these scalars is used to determine output indices (recall Section \ref{sec:multi_out_transactions}; the smallest scalar gets index $t = 0$).\footnote{Output index selection should match other implementations of transaction construction to avoid fingerprinting different software. We use this effectively random approach to align with the core implementation, which also randomizes outputs.} They publish those bLSAGs, and also SAGs which sign the transaction version numbers of intended inputs. After this round participants can calculate the transaction weight based on number of inputs and outputs, and accurately estimate the fee required.\footnote{If it turns out there must be only two real participants, based on comparing the number of inputs and outputs to one's own input/output count, the TxTangle can be abandoned. It's recommended for each participant to have at least two inputs and two outputs, in case of malicious actors who don't abandon TxTangles even when they realize it's just two participants. This recommendation is open to debate, since using more inputs and outputs is not heuristically neutral.}\footnote{TxTangle transactions should not have extraneous information stored in the extra field (e.g. no encrypted payment ID unless it's only a 2-output TxTangle which should have at least a dummy encrypted payment ID).}\footnote{Fee estimation should be based on a standardized approach, so each participant calculates the same thing. Otherwise outputs may be clustered based on fee calculation method. This same fee standard should be implemented outside of TxTangle, to promote TxTangle transactions looking the same as normal transactions.}
    \item Each mock-member uses the list of public keys to construct a shared secret with each other member for offsetting their pseudo output commitments, and decides who will add or subtract based on each pair's smaller public key.\footnote{Since points are compressed (Section \ref{point_compression_section}), just interpret the keys as 32-byte integers. The smaller key's owner adds, and larger key's owner subtracts, by convention.} Each mock-member must pay for 1/$n$\nth of the fee estimate (using integer division). The mock-member with lowest output index is given the responsibility of paying for the remainder after dividing (it will be a truly infinitesimal amount, but must be taken into account to prevent fingerprinting TxTangle transactions). They privately generate transaction public keys for each of their outputs (not to be sent to other members just yet), and construct their output commitments, encoded amounts, and Part A partial proofs to be used for the aggregate Bulletproof range proof, signing all of it with bLSAGs (one commitment, one encoded amount, and one partial proof per bLSAG message, and the key image links these to the original list of random scalars that was used to specify output indices). Pseudo output commitments are generated as normal (Section \ref{sec:ringct-introduction}), then offset with the shared secrets, and signed with SAGs. After the bLSAGs and SAGs are published, and assuming participants estimated the total fee in the same way, they may now verify that amounts balance overall.\footnote{We don't publish the separate fee amounts paid for in case a participant calculated it wrong, which may reveal an output cluster due to a collection of non-standard fee amounts. If amounts don't balance properly, the TxTangle transaction may be abandoned.}
    \item If amounts balance properly we begin a small additional round for building the aggregate Bulletproof that proves all output amounts are in range. Each mock-member uses the previous round's Part A partial proofs and output commitments, and privately computes the aggregate challenge A. They use it to construct their Part B partial proof, which they send to the channel with a bLSAG.
    \item Participants begin filling in the message to be signed by MLSAGs (recall the footnote in Section \ref{full-signature}). Published in random order over the communication interval are two kinds of messages. Each input's ring member offsets and key images are signed with a SAG and associated with the correct pseudo output commitment. Each output's one-time address, transaction public key, and Part C partial proof (computed based on the Part B partial proofs and an aggregate challenge B) are signed with a bLSAG (these can also include a random base transaction public key component, which as we will see can be used for fake Janus mitigation).
    \item Participants use all the partial proofs to complete the aggregate Bulletproof, and privately apply a logarithmic inner product technique to compress it for the final proof to be included in transaction data. Once all the information to be signed by MLSAGs is collected, each participant completes their inputs' MSLAGs and randomly sends them (with a SAG for each) to the channel over the communication interval. Any participant may submit the transaction as soon as they have all the pieces.
\end{enumerate}{}

\subsubsection*{Transaction public keys and the Janus mitigation}

If every participant to a TxTangle knows the transaction private key $r$ (Section \ref{sec:one-time-addresses}), then any of them may test the others' one-time output addresses against a list of known addresses. For this reason it is necessary to construct TxTangle transactions as if there were a subaddress recipient (Section \ref{sec:subaddresses}), including different transaction public keys for each output.

To match with possible implementation of a mitigation for the Janus attack related to subaddresses, in which one additional `base' transaction public key is included in the extra field \cite{janus-mitigation-issue-62}, TxTangles should also have a fake `base' key composed of a sum of random keys generated by each mock-member.\footnote{Key cancellation (Section \ref{subsec:drawbacks-naive-aggregation-cancellation}) should not be a problem, since it's just a fake key and should ideally be randomly indexed within the list of transaction public keys.}

Many TxTangle participants sending money to a subaddress will likely have at least two outputs, one of which diverts change back to the participant. This means any TxTangle participant can still enable Janus mitigation by making their change's transaction public key also the `base' key for the subaddress recipient.\footnote{If sending to your own subaddress, there is no need for Janus mitigation. Wallets enabled for Janus mitigation should recognize the amount spent in a TxTangle equals the amount received to your subaddress, so they don't erroneously notify the user of a problem.} The subaddress recipient might realize the transaction is a TxTangle, and that the `base' key probably corresponds with the sender's change output.\footnote{This is assuming transaction public keys are 1:1 with the outputs, as is apparently the case today. If it was standard for transaction public keys to be in random or sorted order within the extra field, then TxTangle and non-TxTangle transactions would be largely indistinguishable for subaddress recipients. There are niche cases where TxTangle participants are unable to include a `base' key (e.g. when all their outputs are to subaddresses), or where it is clearly non-TxTangle since the subaddress recipient receives most or all of the outputs. Note that since TxTangle transactions would generally have a lot more outputs than a typical transaction, this heuristic can be used to differentiate TxTangles from normal subaddress'd transactions.}


\subsection{Weaknesses}
\label{subsec:weaknesses-txtangle}

Malicious actors have two primary ways to defeat the purpose of TxTangle, which is to hide input/output groupings from potential adversaries/analysts. They may pollute the transactions, such that the subset of honest participants is smaller (or even non-existent) \cite{coinjoin-pollution}. They may also cause TxTangle attempts to fail, and use subsequent attempts by the same participants to estimate input/output groupings.

The former case is not easy to mitigate, especially in the very decentralized case where no participant has a reputation. One possible application of TxTangle is with collaborating pools, who may hide which pool their miners belong to among a collection of pools. Such pools would know the input/output groupings, but since the purpose is helping their connected miners it would behoove them to keep the information secret. Moreover, such TxTangles would not permit bad actors, assuming the pools are somewhat honest.

The latter case can be defended against by only attempting to TxTangle a few times before taking a break, and by always regenerating most random elements of a transaction for new attempts. These elements include the transaction public keys, pseudo commitment masks, range proof scalars, and MLSAG scalars. In particular, the set of ring decoys for each input should remain the same to prevent cross-comparisons revealing the true input. If possible, different true inputs should be used for different TxTangle attempts. Since this weakness is inevitable, it makes the next section's concept more palatable.



\section{Hosted TxTangle}
\label{sec:hosted-txtangle}

Truly decentralized TxTangle has some open questions. How are the timed rounds initiated and enforced? How are chatrooms created in the first place, for participants to find each other? The most straightforward way is with a TxTangle host, who generates and manages those chatrooms.

Such a host would seem to defy the goal of obfuscated participation, since each individual must connect, and send it messages that could be used to correlate input/output groupings (especially if the host participates and knows the message contents). We can use a network like I2P\footnote{The Invisible Internet Project (\url{https://geti2p.net/en/}).} to make each message received by the host appear as if from a unique individual.


\subsection{Basic communication with a host over I2P, and other features}
\label{subsec:txtangle-host-communication}

With I2P, users make so-called `tunnels' that pass heavily encrypted messages through other users' clients before reaching their destination. From what we understand, these tunnels can transport multiple messages before being destroyed and recreated (e.g. there appears to be a 10-minute timer on tunnels). It is essential for our use case to carefully control when new tunnels are created, and which messages may come out of the same tunnel.\footnote{In I2P there are `outbound' and `inbound' tunnels (see \url{https://geti2p.net/en/docs/how/tunnel-routing}). Everything received through an inbound tunnel looks like it's from the same source even if from multiple sources, so on the surface it would appear TxTangle users don't need to create different tunnels for all their usecases. However, if the TxTangle host makes himself the entry point for his own inbound tunnel, then he gains direct access to the outbound tunnels of TxTangle participants.}%\footnote{For the sake of absolutely minimal information leaks, what we describe here is probably incredibly inefficient, especially since I2P is already very inefficient compared to the `clear' web.} is it?
\begin{enumerate}
    \item {\em Applying for TxTangles}: In our original $n$-way proposal (Section \ref{subsec:n-way-channel}) participants gradually add their mock-members to available TxTangle rooms before they are scheduled to close. However, if a large enough volume of users try to TxTangle concurrently, there is likely to be a high rate of failure as users try to randomly put all their intended outputs in the same TxTangle `room', but then the rooms get full too soon so they have to back out. It would be quite a mess.

    We can make an impactful optimization by telling the host how many outputs we have (e.g. giving him a list of our mock-member public keys), and letting him assemble each TxTangle's participants. Since we still retain the bLSAG and SAG messaging protocol, the host won't be able to identify output groupings in the final transaction. All he knows is the number of participants, and how many outputs each had. Moreover, in this scenario observers can't monitor open TxTangle rooms to deduce information about the participants, an important privacy improvement. Note that the host's power to pollute TxTangles isn't significantly different from the non-host design, so this change is neutral to that attack vector.
    \item {\em Communication method}: Since the host is already acting as the locus of message transport, it is simplest for him to manage TxTangle communication. During each round the host collects messages from mock-members (still at random over a communication interval), and at the end of a round there is a short data distribution phase where he sends all the collected data to each participant, with a buffer period before the next round to ensure the messages are received and given time to process.% If someone controls multiple mock-members, it's possible that some of those messages just fizzle out, or perhaps that person receives duplicate messages (e.g. the host is given a different destination for each mock-member). The host should not realize how many real people he is distributing to, nor which mock-member has a given output.
    \item {\em Tunnels and input/output groupings}: Once a TxTangle has been initiated, users should dissociate their mock-member identities from the actual outputs that get created. This means creating new tunnels for bLSAG-signed messages, and each such tunnel may only transmit messages related to a specific output (it is fine to transmit multiple such messages through the same tunnel, since obviously information about the same output comes from the same source). They should also create new tunnels for SAG-signed messages related to specific inputs.
    \item {\em Threat of host MITM attack}: The host might fool a participant by pretending to be other participants, since he controls sending out the mock-member list for constructing bLSAGs and SAGs. In other words, the list he sends participant A might contain participant A's mock-members, and all the rest are his own. Messages received by participant B are re-signed using A's list before being retransmitted to A. Since all messages signed by A's list belong to A, the host would have direct insight into A's input/output groupings!

    We can prevent the host from acting as MITM of honest participant interactions by modifying how transaction public keys are made. Participants send each other their intended transaction public keys as normal (with a bLSAG), then, much like robust key aggregation from Section \ref{sec:robust-key-aggregation}, the actual keys that get included in transaction data (and used to make output commitment masks, etc.) are prefixed with a hash of the mock-member list. In other words, $\mathcal{H}_n(T_{agg},\mathbb{S}_{mock},r_t G)*r_t G$ is the $t$\nth transaction public key. Baking the mock-member list into the transaction itself makes it very difficult to complete TxTangles without direct communication between all actual participants.\footnote{If Janus mitigation is implemented, this MITM defense should instead be done with the fake Janus base key. Each mock-member provides a random key $r_{mock} G$, then the actual base key is $\sum_{mock} \mathcal{H}_n(T_{agg},\mathbb{S}_{mock},r_{mock} G)*r_{mock} G$.}%Users must access the host's `eepsite/service' to discover available TxTangles, without revealing to the host how many prospective participants there are. If he sees five requests for available TxTangles, and then five outputs/mock-members appear, he shouldn't be able to conclude there are five actual participants, nor easily associate any given request with a specific mock-member. To accomplish this, while engaged in a TxTangle a user's client should at random intervals create a new tunnel to the host and through it send a (false) request for available TxTangles. Users should also create new tunnels for each intended output/mock-member, which are used to actually apply for a TxTangle.\footnote{In I2P there are `outbound' and `inbound' tunnels (see \url{https://geti2p.net/en/docs/how/tunnel-routing}). Everything received through a inbound tunnel looks like it's from the same source even if from multiple sources, so on the surface it would appear TxTangle users don't need to create different tunnels for all their usecases. However, if the TxTangle host makes himself the entry point for his own inbound tunnel, then he gains direct access to the outbound tunnels of TxTangle participants.}\footnote{If a large enough volume of users try to TxTangle concurrently, there is likely to be a high rate of failure as users try to randomly put all their intended outputs in the same TxTangle `room', but then the rooms get full too soon so they have to back out. It would be quite a mess. We can make an impactful optimization by telling the host how many outputs we have (e.g. giving him a list of our mock-member public keys), and letting him assemble each TxTangle's participants. Since we still retain the bLSAG and SAG messaging protocol, the host won't be able to identify output groupings in the final transaction. All he knows is the number of participants, and how many outputs each had. Moreover, in this scenario observers can't monitor open TxTangle rooms to deduce information about the participants, an important privacy improvement. Note that the host's power to pollute TxTangles isn't significantly affected, so this change is neutral to that attack vector. We can prevent the host from acting as MITM (`man in the middle') of honest participant interactions (he could do this by serving them fake mock-member lists where he pretends to be all the other participants) by modifying how transaction public keys are made. Participants send each other their intended transaction public keys as normal (with a bLSAG), then, much like robust key aggregation from Section \ref{sec:robust-key-aggregation}, the actual keys that get included in transaction data are prefixed with a hash of the mock-member list and list of original transaction public keys. In other words, $\mathcal{H}_n(T_{agg},\mathbb{S}_{mock},\mathbb{S}_{r-original},r_t*G)*r_t*G$ is the $t$\nth transaction public key. Baking the mock-member list into the transaction itself makes it is very difficult to complete TxTangles without direct communication between all actual participants.}
\end{enumerate}{}


\subsection{Host as a service}
\label{subsec:txtangle-host-service}

It's important for sustainability and continuous improvement that a TxTangle service operate for profit.\footnote{While a deployed TxTangle service may be for profit, the code itself could be open source. This would be important for auditing wallet software that interacts with a TxTangle service.} Rather than compromise user identities with an account-based model, the host may participate in each TxTangle as a lone output, and require the participants to fund it. When accessing the host's `eepsite'/service to apply for TxTangles, users are notified of the current hosting charge, which should be paid on a per-output basis.

Participants would be responsible for paying fractions of the fee {\em and} the host charge. This time, the smallest mock-member's public key (excluding the host's key) would take the remainder of both the fee and host charge.\footnote{We must use mock-member keys here since the host doesn't pay a fee, and his output index is unknown.} Since the host has no inputs, he has no pseudo output commitment to cancel his output's commitment mask. Instead, he creates shared secrets with the other mock-members as usual, then separates his real commitment mask into randomly sized chunks for each other mock-member and divides them by the shared secrets. He publishes a list of those scalars (corresponding them with each other mock-member based on their public key), signing with his mock-member key so participants know it's from the host. The appearance of this list signals the beginning of round 1 from Section \ref{subsec:message-rounds-txtangle} (e.g. the end of setup round `0'). Mock-members will multiply their host-scalar by the appropriate shared secret, and add that to their pseudo commitment mask. In this way, even the host's output cannot be identified by any participant in the final transaction without a full coalition against him.

To simplify calculations of the fee, the host may distribute the total fee to be used in the transaction at the end of round 1, since he will learn the transaction weight early. Participants can verify that amount is similar to the expected amount, and pay their fraction of it.

If participants collaborate to cheat, and not provide the hosting charge, then the host may terminate the TxTangle at round 3. He may also terminate if messages appear in the channel that should not be there or that aren't valid.

At the end of round 5 the host completes the transaction and submits it to the network for verification, as part of his service. He includes the transaction hash in the final distribution message.



\section{Using a trusted dealer}
\label{sec:dealer-txtangle}

There are some drawbacks to decentralized TxTangle. It requires all participants actively communicate within a strict timing schedule (and find each other to begin with), and is challenging to implement.

Adding a central dealer, who is responsible for collecting transaction information from each participant and obfuscating the input/output groupings, can simplify the proceedings. The cost is a higher bar of trust, since the dealer must (at minimum) learn those groupings.\footnote{This section is inspired by the MoJoin protocol outline.}


\subsection{Dealer-based procedure}
\label{subsec:dealer-procedure-txtangle}

The dealer will advertise that he is available to manage TxTangles, and collects applications from potential participants (consisting of the number of intended inputs [with their types] and outputs). He may participate with his own input/output set if he wishes.

After a grouping of almost 16 outputs is assembled (there should be two or more participants, and no participant may have all but one output or input), the dealer will start the first of five rounds. In each round he accumulates information from each participant, makes some decisions, and sends out messages that signify the beginning of a new round.
\begin{enumerate}
    \item To begin, the dealer generates, for each pair of participants, a random scalar, and decides which in each pair should have the positive or negative version. He uses the number and type of inputs and outputs to estimate the total fee required. He sums each participant's scalars together, and privately sends to each their sum, along with the fee fraction they are expected to pay for, and the (randomly chosen) indices of their outputs. These messages constitute a signal to participants that a TxTangle is beginning.
    \item Each participant constructs their sub-transaction as they normally would, generating separate transaction public keys for their outputs (with Janus mitigation as needed), calculating one-time output addresses and encoding output amounts, making pseudo output commitments that balance output commitments and the fee fraction, assembling a list of ring member offsets for use in MLSAG signatures along with the appropriate key images, and add to one of their pseudo output commitments the dealer-sent scalar (multiplied by $G$). They create Part A partial proofs for their outputs, and send all this information to the dealer. The dealer verifies that input and output amounts balance, and sends the complete list of Part A partial proofs to each participant.
    \item Each participant computes the aggregate challenge A, and generates Part B partial proofs which they send to the dealer. The dealer collects the partial proofs and distributes them to all the other participants.
    \item Each participant computes the aggregate challenge B, and generates Part C partial proofs which they send to the dealer. The dealer collects these, and applies the logarithmic inner product technique to compress them into the final proof. Assuming the proof verifies as it should, he generates a random fake Janus `base' transaction public key, and sends the message to be signed in MLSAGs to each participant.
    \item Each participant completes their MLSAGs and sends them to the dealer. Once he has all the pieces, he may finish constructing the transaction, and submit it to be included in the blockchain. He may also send the transaction ID to each participant so they can confirm it was published.
\end{enumerate}{}